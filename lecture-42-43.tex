\Lecture{Jayalal Sarma}{October, 19 2015}{42}{Chinese Remainder Theorem}{Vijitha}{$\alpha$}{TA}
In this lecture, we will see a factorization tool called Chinese Remainder Theorem (CRT), which in its basic form determines a number n that, when divided by some given divisors, leaves given remainders. For example, what is the lowest number n that, when divided by 3 leaves a remainder of 2, when divided by 5 leaves a remainder of 3, when divided by 7 leaves a remainder of 2? We'll see the formal version of CRT for ideals in this lecture.

\section{Notion of Relatively prime ideals}
Let $I$ and $J$ be two ideals in ring $R$. $I + J = \{ x + y ~|~ x \in I , y \in J \}$
\begin{definition}[Relatively prime ideals]
Two ideals $I$ and $J$ in ring $R$ are relatively prime if  $I + J = R$.
\end{definition}

Let us take a simple example. Let $a\Z$ and $b\Z$ be two ideals in $\Z$ such that gcd(a,b)=1.\newline $\implies$ a , b are relatively prime. \newline $\implies$ 1 \in a$\Z$ +  b$\Z$. \newline $\implies \exists$ c , d such that ca + db = 1. If an ideal contains 1, it contains $R$. \newline $\implies a$\Z$ + b$\Z$ = 1 \iff gcd(a,b) = 1. $

\begin{note}
	Enter date, lecture number, title and your name. 
\end{note}
\jsay{This is a sample comment}

\section{Sample section}
This is a sample section. Sections helps in dividing the notes to logically separated parts.


\section{Writing Math}
Here we will see how to write math. Normal math symbols : 
$\alpha\beta\gamma\epsilon\phi\Phi $. You can also use calligraphic letters : $\calC$
\begin{enumerate}
\item Avoid these : B=A2+B*ci, B $=$ A $+$ B $*$ c$_i$, phi:A -> N
\item Good math : $B=A^2 +~~~B \times c_{ij}$, $\phi: A \leftarrow N$. Note the space in the equation.
\end{enumerate}



\section{Writing Theorems and Proofs} \label{sec:rel}
\begin{theorem} \label{cl:relativity}
If $m$ is mass and $c$ is speed of light then, 
\begin{equation} \label{eq:relativity}
E = mc^2
\end{equation}

\end{theorem}
\begin{proof}
Trivial. 
\end{proof}

\begin{claim} 
Halting problem is undecidable
\end{claim}
\begin{proof}(Idea)
Set of languages is $\mathcal{P}(\Sigma^*)$ is uncountably infinite,
while set of all Turing machines which can be identified with 
$\Sigma^*$ is only countably infinite. 

If Halting problem is 
decidable, then every language in $\mathcal{P}(\Sigma^*)$ can be 
captured uniquely by Turing machines. This suggests existence of a 
bijection. But such a bijection between a countably infinite 
and uncountably infinite set cannot exits by Cantor's diagonalisation argument. 
Hence Halting problem is undecidable.
\end{proof}

\subsection{Environments available}
\begin{proposition}
This is a proposition.
\end{proposition}
\begin{corollary}
	This is a corollary
\end{corollary}
\begin{observation}
	This is an observation
\end{observation}
\begin{definition}
	This is a definition
\end{definition}
\begin{example}
	This is an example
\end{example}
\begin{exercise}
	This is an exercise
\end{exercise}
\begin{remark}
	This is a remark
\end{remark}
\begin{conjecture}
	A conjecture
\end{conjecture}

\begin{fact}
	A fact
\end{fact}


\subsection{Writing equations}
\begin{itemize}
\item Normal equations : 
	$\int_0^\infty e^{-x} x^{n-1} \mathrm{d}x = \Gamma(n)$.
\item Display math equation : 
	\begin{equation}
	\int_0^\infty e^{-x} x^{n-1} \mathrm{d}x = \Gamma(n)
	\end{equation}
\item Display math equation with no numbering : 
	\begin{equation*}
	\int_0^\infty e^{-x} x^{n-1} \mathrm{d}x = \Gamma(n)
	\end{equation*}
\item Writing sets and using $\langle$ $\rangle$ instead of $<$ and $>$
	\begin{equation}
	HP = \set{ \langle M,x \rangle ~|~ \text{$M$ on inputs $x$ halts} }
	\end{equation}
	\begin{equation}
	S =  \set{ i ~\left |  \prod_{d | i} i \text{ is even }, i > 0 \right . }
	\end{equation}
\end{itemize}

\subsection{Aligning equations, writing text in math mode}
\begin{align*}
\sum_{i=1}^n i & = \sum_{i=1}^{n-1} i + n \\
			   & = \frac{(n-1)\cdot n}{2} + n && [\text{By induction hypothesis}]\\
			   & = \frac{n(n+1)}{2}
\end{align*}




\section{Drawing tables}
\begin{center}
\begin{tabular}{||c|l|l||}
\hline 
\textbf{Type} & \textbf{Language} & \textbf{Machine} \\ 
\hline \hline
Type 3 & Regular & Finite Automata \\ 
\hline 
Type 2 & Context Free & Push Down Automata \\ 
\hline 
Type 1 & Context Sensitive & Linear Bounded Automata \\ 
\hline
Type 0 & Recursively Enumerable & Turing Machine \\ 
\hline 
\end{tabular} 
\end{center}

\section{Referring sections and theorems}
Recalling equation~\ref{eq:relativity} in claim~\ref{cl:relativity} from section~\ref{sec:rel},
it is possible to generate energy from nuclear reactions.




\Lecture{Jayalal Sarma}{*Month, XX 2015*}{43}{*Title of Lecture*}{*Student name*}{$\alpha$}{TA}

\section{Add details of next lecture}
