\Lecture{Jayalal Sarma}{Aug 31, Sep 1 2015}{15}{*Title of Lecture*}{Mitali Bafna}{$\alpha$}
Example 1:

Consider the complete binary tree of depth $k$. Since any isomorphism will send leaves to leaves, $G = Aut(X)$ acts on the leaves and we observe that given any 2 leaves we can get an isomorphism which sends one leaf to the other.\\

Example 2:

Consider $X = k$ traingles. $Aut(X)$ acts on the vertices and here too given any two vertices there is a permutation in $Aut(X)$ that sends one to the other. \\


These 2 examples motivate the following definition:

\begin{definition}{Transitivity:}
G acts on $\Omega$ transitively if there is only one orbit, i.e. $\forall \alpha,\beta \in \Omega$, $\exists g \in G$, such that $\alpha^g = \beta$.
\end{definition}


In example 2, any $g \in Aut(x)$ will send a triangle to itself or another triangle. Similarly for example 1, this property holds for the leaves taken in pairs, where the pair has all ancestors common.\\

\begin{definition}{Block:}
$\Delta$ is a said to be a block if $\forall g \in G, \Delta^g = \Delta$ or $\Delta^g \cup \Delta = \phi$.
\end{definition}


We observe that trivially $\Omega$ and the singletons form blocks. We say that the group action is primitive if there are no non-trivial blocks.\\


Examples of block structure:

1) $G = S_n$ 

Action is trivially transitive. No non-trivial blocks hence primitive.

2) $G \leq S_n$ acts on itself (right-multiplication, $\Omega = G$).

Transitive action. $g g' = h$

Not primitive: Consider a non-trivial subgroup $H$. $Hg = H$ or $Hg$ is a coset, i.e. $Hg \cup H = \phi$.

Converse holds.

\begin{definition}{Block System:}
A partition of $\Omega$ into sets such that each part is a block under the action of G.
\end{definition}

Claim 1: $\Delta^g$ is a block if $\Delta$ is a block. 

\begin{proof}
Suppose not. Implies $\exists h \in G, \Delta^{gh} \ne \Delta^g$ and $\Delta^{gh} \cap \Delta^g \ne \phi$.

Consider the action of $g' = (gh)^{-1}$ on $\Delta^{gh}, \Delta^g$. We have that $(\Delta^{gh})^{g'} = \Delta$.

Since $\Delta^{gh} \ne \Delta^g \Rightarrow \exists \beta \in \Delta^{gh}, \beta \notin \Delta^g$. Since any group element permutes $\Omega$, $x^g = y^g \Leftrightarrow x = y$. So $\beta^{g'} \in \Delta$ and $\notin \Delta^{gg'}$.

But $\Delta^{gh} \cup \Delta^g \ne \phi \Rightarrow \exists \alpha \in \Delta^{gh}, \in \Delta^g$. This means that $\alpha^{g'} \in \Delta$ and $\in \Delta^{gg'}$. 

Combining these two we get that $\Delta \ne \Delta^{gg'}$ and $\Delta \cap \Delta^{gg'} \ne \phi$, which condraticts the fact that $\Delta$ is a block.

\end{proof}


Claim 2: $|\Delta^g| = |\Delta|$

This follows from the fact that $g \in G$ permutes the elements of $\Omega$.
\\
Claim 3: $\bigcup_g \in G \Delta^g = \Omega$

Let $k \in \Delta$ and $k' \in \Omega$. Since the action is transitive $\exists g, k^g = k'$. So $k' \in \Delta^g$. Hence $\Omega \subseteq \bigcup_g \in G \Delta^g $. The other containment is trivial and hence they are equal.
\\
Claim 4: $\Delta^{g_1} \cap \Delta^{g_2} \ne \phi \Rightarrow \Delta^{g_1} = \Delta^{g_2}$. 

This is true because $\Delta^{g_1}$ is a block (Claim 1).
\\
Claim 4: For any block $\Delta , |\Delta|$ must divide $|\Omega|$.

The above claims prove that $\Omega$ is partitioned into subsets with equal cardinality, one of which is $\Delta$. So $|\Delta|$ must divide $|\Omega|$. 

Just an Observation: If $|\Omega |$ is prime, the group action is bound to be primitive. 

.........................................................................................................

\begin{definition}{Maximal Subgroup:}
$H \leq G$ is a maximal subgroup if $\not \exists H',H < H' < G$.
\end{definition}

\begin{lemma}
Let $G$ act transitively on $\Omega$. The action is primitive iff $\forall \alpha \in G, G_\alpha$ is a maximal sugroup of G. 
\end{lemma}

\begin{proof}
$\Leftarrow$ 

We will prove the contrapositive that is: Action is not primitive $\Leftarrow$ $\exists \alpha G_\alpha$ is not a maximal subgroup)

Suppose $G's$ action is not primitive $\Rightarrow \{\alpha\} \subsetneq \Delta \subsetneq \Omega$. 

Take $H = Setstab(\Delta)$. $H$ is a subgroup of G.

Let $g \in G_\alpha$. Since $\alpha^g = \alpha$, $\Delta^g = \Delta$ and $g \in H$. Hence $G_\alpha \leq H$.

Let $\beta \neq \alpha \in \Delta$ and $g, \alpha^g = \beta$. $g \notin G_\alpha$ but $g \in H$. 

Hence $G_\alpha \lneq H$ and $G_\alpha$ is not maximal. 

$\Rightarrow$

$G_\alpha \lneq H \lneq G$. To produce a $\Delta,\{\alpha\} \subsetneq \Delta \subsetneq \Omega$, such that $\Delta$ is a block.

Define $\Delta = \alpha^H$. 

$\alpha \subsetneq \Delta$

Since $G_\alpha \not \leq $...

$\Delta \not \subseteq \Omega$

$G = \bigcup_{g_\beta :\alpha \rightarrow \beta } G_\alpha g_\beta $.

Obs: $\beta \in \Delta \Leftrightarrow G_\alpha g_\beta \subseteq H$.

$\Delta$ is a block.

$\Delta ^g \cap \Delta \neq \phi$

$\exists h \in H,\alpha ^h \in \Delta ,\alpha^h$

\end{proof}


