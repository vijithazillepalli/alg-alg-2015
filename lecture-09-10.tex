\Lecture{Jayalal Sarma}{August, 17 2015}{9}{Set Stabilizers and Point-wise Stabilisers}{Aditi Raghunathan}{$\delta$}{Jayalal Sarma}

\section{Recap and Exercise}
In the previous class, we were looking at ways to solve $\cal{AUT}(X)$.  We looked at $4$ different problems in this context. Consider \emph{Problem 3} of finding the orbit. We are interested in computing the size of the orbit of  $\alpha \in G$. Since some of the students expressed difficulty in coming up with an algorithm, we decide to state the algorithm and leave the correctness proofs as an exercise.

\subsection{Orbit Computation} \label{sec:orbit-comp}
The following is an algorithm to calculate the orbit of $\alpha$

\begin{algorithm}[htp!]
\caption{Algorithm for Orbit Computation}\label{orbit}
\label{alg:orbit_alg}
\begin{algorithmic}[1]
\Procedure{Orbit Computation}{ Input : Generating set $S$ }
\State $\Delta = \lbrace \alpha \rbrace$

\Repeat{}
\State $\Delta^1 = \Delta$
\For {$g \in S$ and $\delta \in \Delta$}
\State $\Delta = \Delta^1 \cup \lbrace \delta^g \rbrace$
\EndFor
\Until {$\Delta^1 \neq \Delta$}
\EndProcedure
\end{algorithmic}
\end{algorithm}

The algorithm was developed as step by step in the lecture which also developed its correctness. However, the formal proofs are left as an exercise. To be precise, it is left as an exercise to the reader to do the following : (1)~Prove that Algorithm \ref{alg:orbit_alg} eventually terminates. (2)~Prove that Algorithm \ref{alg:orbit_alg} terminates with $\Delta$ = Orbit($\alpha$)
(3)~Calculate the running time of Algorithm \ref{alg:orbit_alg}.

%\subsection{Algorithm 2 to compute orbit}
The problem can be viewed in terms of the graph. Consider the following definition of a directed graph $X$. $V(X)$, the vertex set of the graph $G$ is equal to the set $\Omega$. 
$E(X) = \lbrace (\alpha, \beta) \mid \exists g \in S \text{ such that } \alpha^g = \beta \rbrace $

The key observation is that, if there is a path in $X$ from $\alpha$ to $\gamma$ then $\gamma$ is in the orbit of $\alpha$.
Finding the orbit of $\alpha$ is equivalent to computing the 
transitive closure of the graph $X$. It is left as an exercise to the reader to complete the details of the above algorithm including rigorous proof of correctness.

\section{Set Stabilisers Problem}

We recall the problem that we defined in the last lecture. Given group $G \leq S_n$ and a set $\Sigma \subseteq \Omega$, where $G$ acts on $\Omega$ and $|\Omega| = n$, we are interested in computing the generating set of the following group :
\begin{align}
\mathcal{SETSTAB} (\Sigma) = \lbrace g \in G \mid \Sigma^g = \Sigma \rbrace \text{ where }\\
\Sigma^g = \lbrace \alpha^g \mid \alpha \in \Sigma \rbrace 
\end{align}

As we saw in the last lecture, solving the set stabiliser problem gives an algorithm for obtaining $\mathcal{AUT}(X)$.

We now define a variant of the problem called the \textit{Point Stabiliser Problem}

\paragraph{Point-wise Stabiliser Problem}
\begin{align}
\mathcal{POINTSTAB} (\Sigma) = \lbrace g \in G \mid \forall \alpha \in \Sigma, \alpha^g = \alpha \rbrace
\end{align}
$\mathcal{POINTSTAB}$ is \emph{easier} to solve than $\mathcal{SETSTAB}$
The Point Stabilizer Problem is to obtain a generating set for the
group $\mathcal{POINTSTAB} (\Sigma)$. It is easy to check that $\mathcal{POINTSTAB} (\Sigma)$ is indeed a group.

In the reduction from $\mathcal{GA}$ to $\mathcal{GI}$, we defined
a tower of sub-groups  where
\begin{align}
G^{(i)} = \lbrace g \in G \mid \forall 1 \leq j \leq i j^g = j \rbrace 
\end{align}
Observe that each tower here is a point-wise stabiliser of $\lbrace 1, 2, \cdots i \rbrace$
For each $G^{(i)}$, we have \\
$\Omega = \lbrace 1, 2, \cdots n \rbrace$  and $\Sigma = \lbrace 1, 2, \cdots i \rbrace$ 

Here is a general strategy for solving point-wise stabilizer problem. Without loss of generality, as above, assume that $\Sigma$ is the first $i$ elements of $\Omega$. The problem to solve is precisely the following, given the generators $S$ of a group $G$, find the generators of $G^{(i)}$. A natural attack on the problem is stepwise, given the generating set of $G$, find that of $G^{(1)}$, and then using that to find the generating set of $G^{(2)}$ and so on. In $i$ levels of this recursion, we will be done.

The key problem that we identified for solving the order computation as well as point-stabilizer problem is the following. Given the generating set of $G^{(i-1)}$, find that of $G^{(i)}$. This is exactly what we will do in the next section.

\subsection{Schreier's Lemma}

Schrier gave a clever way of completing our task. If in addition to the generating set of $G^{i-1}$, we are also given the coset representatives of $G^{i}$ as a subgroup in $G^{(i-1)}$.

In the following, we will call $G$ to be the bigger group ($G^{(i-1)}$) and $H$ to be the subgroup $G^{(i-1)}$. Let $R$ be set of right coset representatives of the subgroup $H$ in $G$. 
The following lemma gives a direct way of writing the generating set for $H$. 

\begin{lemma}[\textbf{Schreier's Lemma}]
Let $S$ be the generating set of $G$ and $R$ be 
the set of coset representatives of $H$.
%\begin{align}
$S' = \lbrace r_1 g r_2^{-1} \mid r_1, r_2 \in R, g \in S \rbrace$ 
$S^\prime \cap H$ forms a generating set for $H$.
%\end{align}
\end{lemma}
%\begin{remark}
%\label{rem1}
Before we prove this, notice that $R$ contains the identity of the groups. Hence the Set $S \subseteq S'$. By taking $S' \cap H$, we are extracting the elements in $S'$ which are also in $H$. In order to do this, we do need a membership testing method for $H$. In our context, $H$ is $G^{(i)}$ and has an easy membership test given an element $g \in G^{(i-1)}$.
%\end{remark}

\begin{proof}

Define, 
$$RS = \lbrace rs \mid r \in R, s \in S \rbrace$$

Since each element in $RS$ can be written as $(r s r_1^{-1}) r_1$, we immediately get that,
\begin{equation}
RS \subseteq S^\prime R
\label{thm:obs1}
\end{equation}

Let $\gen{S^\prime}$ denotes the group generated by $S^\prime$. By definition, $S^\prime R \subseteq \gen{S^\prime}R$. From \ref{thm:obs1}, we have $RS \subseteq \langle S^\prime \rangle R$
Therefore, $RSS \subseteq \gen{S^\prime} RS$, and hence $RSS \subseteq \gen{S^\prime} RS$. Repeating this process, we have
$\forall t \geq 0, RS^t \subseteq \langle S^\prime \rangle R$

Since $R$ contains the identity, and $S$ generates $G$, $G \subseteq \bigcup\limits_{i=1}^k RS^t$ for some fixed finite $k$. In fact, we will see later that $k \le |G|$, but we do not need this bound here.

We argue that $H \le \gen{S^\prime}$. Sinec $G \subseteq \langle S^\prime \rangle R$, if $\langle S^\prime \rangle$ does not contain $H$, $\langle S^\prime \rangle R $ cannot cover $G$ Hence proved by contradiction. Therefore, $S^\prime \cap H$ generates $H$.
\jsay{To be completed, why should it exactly generate $H$?}
\end{proof}

The following is a different view of the above proof. This was not done in the lecture, but we record this for completeness of this notes and for future use. We derive an equivalent statement to the above, by observing that for every $r_1$ and $a$, there is a unique $r_2$ such that $r_1ar_2^{-1} \in H$. Call this $r_2$ as $\overline{r_1a}$. Hence the following is an equivalent restatement.

\begin{lemma}[\textbf{Schreier's Lemma restated}]
Let $H$ be a subgroup of $G$, $S$ be the generating set of $G$ and $R$ be the set of right coset representatives (containing identity), then,
$$T = \left\{ rs (\overline{rs})^{-1} \mid r \in R, s \in S \right\}$$
is a generating set for $H$, where $\overline{g}$ denotes the coset representative corresponding to the element $g$ (that is, the unique element in $Hg \cap R$).
\end{lemma}
\begin{proof}
First observe that the elements of $T$ are in $H$ (that is the way we redefined it to begin with). Hence, it cannot generate a larger group. It is enough to show that $T \cup T^{-1}$ generates all the elements of $H$.

Consider an arbitrary element $h \in H$. Since $H \le G$, it can be written as the product of elements from $S \cup S^{-1}$. Let $h = s_1s_2 \ldots s_k$. We will incrementally keep modifying this product representations from left to right to obtain a product representation of $h$ using elements of $T \cup T^{-1}$. We call this modification sequence to be $h_1, h_2 \ldots h_k$ where each $h_i = h$.
%
%Let $r_1 = 1$. The sequence goes as follows.
%
%$$h_1 = r_1 s_1 s_2 \ldots s_{i+1} s_{i+1} s_{i+3} \ldots s_k$$
%$$h_2 = t_1 r_1 s_2 \ldots s_{i+1} s_{i+1} s_{i+3} \ldots s_k$$
%$$h_3 = t_1 t_2 r_2 s_3 \ldots s_{i+1} s_{i+2} s_{i+3} \ldots s_k$$
%$$\vdots$$
%$$h_{i-1} = t_1 t_2 t_3 \ldots t_{i-1} t_{i+1} r_{i+2} s_{i+2} s_{i+2} \ldots 
%$$h_i = t_1 t_2 t_3 \ldots t_i t_{i+1} r_{i+2} s_{i+2} s_{i+2} \ldots s_k$$
%$$\vdots$$
%$$h_{k} = t_1 t_2 \ldots t_{i+1} s_{i+2} s_{i+3} \ldots s_k$$
In general, $$h_i = t_1 t_2 t_i r_{i+1} s_{i+1} s_{i+2} \ldots s_k$$
where $t_i \in T \cup T^{-1}$ and $r_{i+1} \in R$.

The initial element $h_0 = \prod_{i=1}^k s_i$ with $r_1 = 1$, which is exactly the original representation, and the final element $h_k = (\prod_{i=1}^k t_i)r_{k+1}$. Since $h_k = h \in H$, it must be that $r_{k+1}$ is the identity. This gives the representation for $h$ in terms of $T \cup T^{-1}$.

Now we complete the proof by defining $h_{i+1}$ from $h_i$. We just need to define $t_{i+1}$ and $r_{i+2}$. Take, $t_{i+1} = r_{i+1}s_{i+1}\overline{r_{i+1}s_{i+1}}^{-1}$ and $r_{i+2} = \overline{r_{i+1}s_{i+1}}$. Clearly, the product $h_{i+1}$ has the required form, $t_{i+1} \in T \cup T^{-1}$ and $r_{i+2} \in R$.
\end{proof}


\Lecture{Jayalal Sarma}{August, 18 2015}{10}{Using Schreier's Lemma - Reduce
Algorithm}{Aditi Raghunathan}{$\delta$}{Jayalal Sarma}

%%%%%%%%%%%%% Length of the generating sequence %%%%%%%%%%%%%%%

Let $G$ be a group generated by a set $S$. In the first part of this lecture, we went after a question from the previous lecture, about the value of $k$ such that $G \subseteq \cup_{t=0}^k S^t$. This is an important question to be answered for the following variant of the membership problem. Given a group $G \le S_n$ via the generating set $S$, and a $g \in S_n$, we asked the problem of testing whether $g \in G$. In the previous versions, we expected a yes or no answer to this question. However, it is also useful to ask the computational version of this problem. That is, if $g \in G$, then give a representation of $g$ as the product of elements in $S$. However, for this, the first question to ask is a bound on the length of any such representation.

\section{Bounds on length of generating sequence}

We start with the following easy bound.

\begin{proposition}
$k \leq |G|$
\end{proposition}
\begin{proof}
We prove the statement by contradiction. 

Let the smallest length of the generating sequence of 
some element $g_1 \in G$ be $|G| + m$, $m>0$.
\begin{align}
g_1 = \prod\limits_{i=1}^{|G|+m} a_i
\end{align}

Consider partial products of the sequence above : 
\begin{align}
S_l = \prod\limits_{i=1}^{l} a_i
\end{align}
There are $|G|+m$ partial products. There are only $G$ possible values for $S_i$ 
By Pigeon Hole Principle, we have the following :
\begin{align}
\label{eqn:php}
\exists l_1, l_2, ~~ l_1 < l_2 ~~ S_{l_1} = S_{l_2}
\end{align}

This means, we can remove the elements of the generating
sequence between $l_1$ and $l_2$, which will generate the
same element $g_1$ (From Equation ~\ref{eqn:php}). Hence we get a shorter generating sequence for $g_1$ which contradicts the original assumption. Thus, $k \leq |G|$
\end{proof}

The above bound is too weak for our purpose because, in our context, $|G|$ could be very large. Can we improve this bound in general?
Unfortunately we end up proving that the bound is tight.
%%%%%%%%%%%%%%%%%%% Lower bound on the size %%%%%%%%%%%%%%%%

\begin{proposition}
There is an $n \in N$, group $G \le S_n$ and a generating set $S$ of $G$, and a $g \in G$ such that the shortest product representation of 
sequence is of length $|G|$. \label{prop:repr-large}
\end{proposition}
\begin{proof}
Let $p_1, p_2, \cdots p_m$ be the first $m$ distinct prime numbers.
Choose $n = p_1p_2 \cdots p_m$. We will define a $G \le S_n$. The $G$ that we define will be a cyclic group generated by the following element $a$.
\begin{align}
a = (1, 2, \cdots p_1) (p_1+1, p_1+2 \cdots p_2) \cdots (\cdots)
\end{align}
%Let $G$ be the group generated by $\lbrace a \rbrace$. 
Let $M$ be the order of the element $a$. That is, $M$ is the smallest $M$ such that $a^M = id$. First observation is that, $M = \prod\limits_{i=1}^m p_i$, which is the lcm of the length of the above cycles. It is easy to see that the element $a^{M-1}$ cannot have a shorter representation than as the product of $a$, $M-1$ times. Notice that $M \ge 2^m$.

From the prime number theorem, $m \geq \Omega(\frac{p_m}{\log p_m})$.
Also notice that $n = \sum\limits_{i = 1}^m p_i \leq 1 + 2 + 3 \cdots + p_m \leq p_m^2$. Thus, $p_m \geq \sqrt n$.

Hence, $m \geq \Omega(\frac{\sqrt n}{\log \sqrt n})$. Thus, $M \geq 2^{\frac{c \sqrt n}{\log n }} - 1$.
\end{proof}

That is negative news. However, the above example does not rule out the existence of a polynomial length expression or computation for the element. For example, $a^{M-1}$ can be expressed in terms of
$a$ through repeated squaring.

But then, how do we formulate our membership problem? This is where the strong generating strikes again. If we are allowed to change the generating set $S$ to and $S'$ (precomputation before seeing the $g$) we can always compute a generating set such that we can output a short representation. 

%%%%%%%%%%%%%%%%%%%%%%%%%%%%%%%%%%%%%

\section{Need of a Reduction Step}

We now return to the algorithm to compute the generators of a sub group. Using Schreier's lemma, we concluded that it's sufficient to compute the set $S^\prime \cap H$. 

Before getting into the algorithm for the same, let us first
look at the size of the set that is generated. 

By the definition of the set $S^\prime$, we can only say
\begin{align}
|S^{\prime}| \leq |S| |R|^2 
\end{align}
If we have to implement this recursively, we see
that each level we are incurring a $|R|^2$ term. 
Hence 
\begin{align}
|S^{\prime}| \leq |S| l_1^2 \cdot l_2^2 \cdots 
\end{align}

We need to carefully control the size of the generating
set $S^\prime$ at each level.

\section{{\sc Reduce} algorithm}

How do we reduce the generating set to control its size? Consider $\pi, \psi \in S^\prime$ such that $1^\pi = 1^\psi = k$. Do we need to keep both of them? Yes, indeed, it may be that $2^\pi = k'$ and $2^\psi = k''$. We cannot afford to throw any of them away. 

However, here is an observation. Consider replacing $psi$ with 
$\pi^{-1} \psi$.

We can observe the following : 
\begin{enumerate}
\item We can still obtain all the elements $\psi$ can be obtained $\pi(\pi^{-1} \psi)$. Observe that we also do not add any new elements 
in the process as well. Hence by this change, the subgroup that is generated remains the same.
\item Doing the above process of replacement for every pair of elements that map $1$ to the same element, we end up with elements that all map $1$ to different elements (except those that fix $1$).
\item $\pi^{-1}\psi$ fixes $1$. Hence all the newly added elements fix $1$.
\item We can repeat the same procedure for $2,3,\ldots n$.
\end{enumerate}

The idea seems neat. At the end of the above procedure, for every pair $(i,j)$, we have exactly one $\pi$ such that $i^{\pi} = j$. This immediately gives an upper bound of $O(n^2)$ in the size of the sets.

However, there is a catch. When we repeat the same procedure for $2,3, \ldots n$, what is the guarantee that the property for $1$ will not be violated?

The ideas is that once we fix the elements which are mapping $1$ to $k$ where $k \ne 1$, all the extra elements which are mapping now $1$ to $1$ are pushed to the stage 2. In stage 2, the operations of the form $\pi^{-1} \psi$ are done on elements which are stabilizing 1. Hence they will continue to stabilize $1$. In the end, we may get several trivial elements, which we can simply discard.

To clarify this, we write this entire algorithm as a procedure.

\begin{algorithm}
\caption{\textsc{Reduce} Algorithm. Input : Generating set $S$, Output: Reduced Generating Set}
\label{alg:reduce}
\begin{algorithmic}[1]
\State{$B_0 = B$}
\State{$A[\ ][\ ]$, an empty $n\times n$ array}
\For{$i=0$ to $n-1$}
\ForAll{$\psi\in B_i$}
\State{$j = i^\psi$}
\If{$A[i][j]$ is empty}
\If{$j=i$}
\State{$B_{i+1} = B_{i+1} \cup \{\psi\}$}
\Else
\State{$A[i][j] = \psi$}
\EndIf
\Else
\State{$\pi = A[i][j]$}
\State{$B_{i+1} = B_{i+1} \cup \{\psi^{-1}\pi\}$}
\EndIf
\EndFor
\EndFor
\State discard all trivial elements from $\cup B_i$
\State {\bf return} $\cup B_i$
\end{algorithmic}
\end{algorithm}
The correctness proof for the algorithm is left as an exercise.
