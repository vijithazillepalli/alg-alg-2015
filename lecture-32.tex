\Lecture{Jayalal Sarma}{October 1 2015}{32}{Ideal Membership Problem And Grobner Basis}{Subhadra Nanda}{$\alpha$}{}

In the last lecture we discussed about multivariate polynomial division algorithm.

Given a set of polynomials $F = \{f_1,f_2,\ldots,f_k\}$ and $F, f \in \mathbb{F}[x_1, \ldots, x_n]$.

Division Algorithm produces $u_1,u_2,\ldots,u_k, r \in \mathbb{F}[x_1, \ldots, x_n]$, such that,
$$ f = u_1f_1 + u_2f_2 + \dots + u_kf_k + r $$ i.e., $r$ is reduced with respect to $F$.

The leading product of polynomial $f$ has following property:

$$lp(f) = \max_{1 \le i \le k} ( lp(u_if_i), lp(r)) $$
$$ ~~~~~= max ( \max_{1 \le i \le k} ( lp(u_i)\cdot lp(f_i)), lp(r))$$

So, it can not be that, a leading power product of $u_i$ is multiplied with a non-leading power product of $f_i$ and gives a leading power product of $u_if_i$. Because it contradicts the second property of term ordering.

In the last lecture, we stopped our discussion by posing a question on ideal membership problem. In this lecture we will continue that problem.

\section{Ideal membership problem}
\begin{problem}
Given an ideal $ I = \langle f_1,f_2,\ldots,f_k \rangle$ and  polynomials $f_1,\ldots,f_k, f \in \mathbb{F}[x_1,...,x_n]$.
Check whether the polynomial $f$ is a member of the ideal $I$.
\end{problem}

The proposed algorithm is following :
\begin{itemize}
\item Find $f {\mathop \rightarrow \limits^{F}}_{+} r$,
\item Return YES, if $ r = 0$,
\item Return NO otherwise.
\end{itemize}

We can observe that, the answer is definitely YES, if $r$ is zero. But, can we say NO, if $r$ is not zero? The answer is dependent on the order of the choices of $f_1, \ldots, f_k$.

We want to show that order of choices does really matter. Let's take an example.

\subsection{Example}
Let, polynomials $f,f_1,f_2 \in Q[x,y]$ and $y>x$ in degree lex ordering.
$$f = y^2x - x$$
$$f_1 = yx - y$$
$$f_2 = y^2 - x$$

Suppose, we choose $f_1$ first,then $f_2$, i.e., $f \mathop \rightarrow\limits^{f_1} h_1 \mathop \rightarrow\limits^{f_2} r$, division process will be as follows:



\begin{center}
\begin{tabular}{r}
	$y + 1 $ \hphantom{$\strut-x$} \\
	$\longdivide{f_1 = yx - y, f_2 = y^2 - x}{y^2x - x}$ \hphantom{$\strut-x$} \\
	{\underline{$y^2x - y^2$}}\hphantom{$\strut-x$}\\
	$y^2 - x$  \\
	\underline{$y^2 - x$} \\
	$0$
\end{tabular}
\end{center}

Observe that, in this case $f$ this reduces to $r = 0$.

But, if we choose $f_2$ first, then $f_1$, division process will be as follows :

\begin{center}
\begin{tabular}{r}
	$x $ \hphantom{$\strut-x$} \\
	$\longdivide{f_2 = y^2 - x, f_1 = yx - y}{y^2x - x}$ \hphantom{$\strut-x$} \\
	{\underline{$y^2x - x^2$}}\hphantom{$\strut-x$}\\
	$x^2 - x$
	
\end{tabular}
\end{center}

In this case, $f$ reduces to $ r = x^2 - x $, which cannot be reduced farther.

Therefore, $r$ is really dependent on the order of choices.

The same problem we have already seen in the univariate case also. In that case, we changed the generators of ideal by taking their $gcd$, so that, we had only one element as the generator of the ideal.

But, unfortunately, we can not do the same for multivariate polynomials, because the ideal with multiple variables cannot be generated by a single element.

To solve this problem, we are going to transform $\langle f_1,f_2, \ldots ,f_k \rangle$ to $\langle g_1,g_2, \ldots ,g_k \rangle$ in such way that, it follows a special property. We will describe that property now.

\section{Grobner Basis}
\begin{definition}
Let $I$ be an ideal in $\mathbb{F}[x_1, \ldots ,x_n]$
and a set of polynomials $G = \{g_1, \ldots ,g_k\} \subseteq I$.

$G$ is called Grobner basis, if $\forall f \in I$ such that $f \neq 0$, $\exists i \in {1, \ldots ,k}$ such that $lp(g_i)$ divides $ lp(f)$.

\end{definition}

That means, whenever $f$ is non-zero and is in ideal $I$, we can continue the reduction process until we get $f = 0$.

The algorithm will stop, if the polynomial $f$ is not in ideal $I$.

Now, if we collect all the leading terms of the ideal $I$, i.e., $lt(I) = \{ t | \exists f \in I s.t t = lt(f) \}$ and all the leading terms of $G$, i.e., $lt(G) = \{ lt(g_i) | g_i \in G \}$,

we can observe that the terms in $lt(I)$ are multiple of the terms in $lt(G)$.

\begin{observation}

The ideal generated by $lt(G)$ has all the combinations of leading terms with all possible multiples. So, $\langle lt(G) \rangle$ contains $lt(I)$.
$$ lt(I) \subseteq \langle lt(G) \rangle \Longrightarrow \langle lt(I) \rangle \subseteq \langle lt(G) \rangle$$
We can write $\langle lt(I) \rangle$ as $Lt(I)$. So, $Lt(I) \subseteq Lt(G)$.
\end{observation}

\section{Characterization Of Grobner Basis}
\begin{itemize}
\item[(1)] $G$ is Grobner basis.
\item[(2)] $ \forall f \not\equiv 0$, $ f \in I \Longleftrightarrow f {\mathop \rightarrow \limits^{G}}_{+} 0$
\item[(3)] $ f \in I \Longleftrightarrow f = \sum_{i =1}^{k} h_ig_i$ 

and $lp(f) = \max_{1 \le i \le k} ( lp(h_i)lp(g_i))$
\item[(4)] $Lt(G) = Lt(I)$ (Here, $Lt(I)$ is called initial ideal).
\item[(5)] For any polynomial $f \in \mathbb{F}[x_1 \ldots x_n], f {\mathop \rightarrow \limits^{G}}_{+} r$,  where, $r$ is unique.
\end{itemize}

\subsection{(1) $\Longrightarrow$ (2)}
\begin{proof}
Assume (1) is true, i.e., $G$ is Grobner basis.

Let, polynomial $f \ne 0$ and $ f \in I \Longleftrightarrow f {\mathop \rightarrow \limits^{G}}_{+} r$.

We have to prove that, $ f \in I \Longleftrightarrow r = 0$.

\begin{observation}
$f \in I \Longleftrightarrow r \in I$, because $\exists u_1 \ldots u_k$ such that, $f = u_1g_1 + u_2g_2 + \ldots + u_kg_k + r$. We know that, $f, g_1,g_2 \ldots g_k \in I$. So, $r$ is also in Ideal $I$.
\end{observation}

Suppose, $r \neq 0$. So, $\exists i$ such that $lp(g_i)$ divides $lp(r)$, as $r \in I$.

That means $r$ is not reduced, because $r$ can be expressed as $r = u_ig_i + r'$.

Therefore, if $r$ is reduced from f, then $r = 0$.

\end{proof}

\subsection{(2) $\Longrightarrow$ (3)}
\begin{proof}
Suppose, (2) is true, i.e., $ \forall f \not\equiv 0$, $ f \in I \Longleftrightarrow f {\mathop \rightarrow \limits^{G}}_{+} 0$.

$f$ can be expressed as,  $f = h_1g_1 + h_2g_2 + \ldots + h_kg_k + r$. Here, $ r = 0$. So, $f = \sum_{i =1}^{k} h_ig_i$.

From division algorithm, we know that, $lp(f) = \max_{1 \le i \le k} ( lp(g_i)lp(f_i), lp(r))$.

So, here $lp(f) = \max_{1 \le i \le k} ( lp(g_i)lp(f_i))$, as $ r = 0 $.

\end{proof}

\subsection{(3) $\Longrightarrow$ (4)}
\begin{proof}
Assume, (3) is true.

We have to prove, $Lt(G) \subseteq Lt(I)$ and $Lt(I) \subseteq Lt(G)$.

To prove $Lt(G) \subseteq Lt(I)$, we have to show, if $f \in Lt(G)$ then, $f \in Lt(I)$.

If $f \in Lt(G)$, we can say $f = \sum_{i =1}^{k} h_i \cdot lt(g_i)$.

Observe that, each term in R.H.S(hence, in L.H.S) is divided by $lt(g_i)$ for some $i$, where $g_i \in I$.

Hence, $f \in Lt(I)$.

To prove $Lt(I) \subseteq Lt(G)$, we have to show if $f \in Lt(I)$ then, $f \in Lt(G)$.

If, $f \in Lt(I)$, then $f = \sum_{i =1}^{k} h_i \cdot lt(f_i)$, where $f_i \in I$.

As, $f_i \in I$ and $f_i \neq 0$ and $g_i \in G$, we can say that $lt(f_i)$ can be divided by $lt(g_i)$.

So, each term in $f$ is divided by $lt(g_i)$ for some $i$.

Hence, $f \in Lt(G)$

\end{proof}

\subsection{(4) $\Longrightarrow$ (1)}
\begin{proof}

Suppose, $Lt(G) = Lt(I)$.

Let, $f \in I$ and $f \neq 0$. To prove (1), we have to show $\exists i$ such that, $lp(g_i)|lp(f)$.

We know that, $lt(f) = \sum_{i =1}^{k} h_i \cdot lt(g_i)$.

Observe that, each term in R.H.S is divided by some $lt(g_i)$. That means, $lt(f)$ must be divided by some $lt(g_i)$.

Hence, $lp(g_i)|lp(f)$ for some $i$.
\end{proof}

\subsection{(1) $\Longrightarrow$ (5)}
\begin{proof}
Let, $f$ reduces to $r_1$ and also reduces to $r_2$ for different choices of order of $g_1, \ldots ,g_k \in G$.

So, $ f {\mathop \rightarrow \limits^{G}}_+ r_1$ and $ f {\mathop \rightarrow \limits^{G}}_+ r_2$.

We have to prove $r_1 = r_2$.

Let, $r_1 \neq r_2$ i.e., $r_1 - r_2 \neq 0$.
Observe that $r_1 - r_2$ is in ideal, because, $f$ can be expressed as in terms of $g_1, \ldots ,g_k$ and $r_1$, and can also be expressed as in terms of $g_1, \ldots ,g_k$ and $r_2$. Now, if we substract one $f$ from another $f$, we will get $r_1 - r_2$, which is in ideal.

As, $lt(r_1)$ and $lt(r_2)$ are not divisible by any terms of $G$, that means $r_1 - r_2$ is also reduced. But, $r_1 - r_2 \neq 0$ and $r_1 - r_2$ is in ideal. So, this is a contradiction.

Hence, $r_1 = r_2$.
\end{proof}

