\Lecture{Jayalal Sarma}{October 14 2015}{40}{Proof of $R$ is a UFD implies
$R[x]$ is a UFD}{K Dinesh}{$\beta$}{K Dinesh}

\Lecture{Jayalal Sarma}{October 16 2015}{41}{Characterization of irreducible
polynomials}{K Dinesh}{$\beta$}{K Dinesh}

We saw in last two lectures that the following problems are well defined.
\begin{problem}
	Given a polynomial $f \in \polyring$, 
	\begin{itemize}
		\item test if it is irreducible
		\item if it is not, factorise $f$ into irreducible factors.
	\end{itemize}
\end{problem}

In this lecture, we give a characterisation of irreducible polynomials over
$\F[x]$. Recall the following definitions.

\begin{definition}[Quotient Ring]
	For an ideal $I$ in a ring $R(+,\cdot)$, the quotient ring $(R/I, +
	,\cdot)$ is a ring defined as the collection of $\{a+I ~|~ a \in R\}$
	with operations defined as : for all $a+I, b+I \in R \ I$,
	$(a+I)+(b+I) = (a+b+ I)$ and $(a+I) \cdot (b+I) = a\cdot b+I$.
\end{definition}
The element of $R/I$ are the cosets of $I$ in $R$ and this forms a ring for
the operations defined.  Following lemma gives as a characterisation of when
do two elements in $R$ fall in the same coset.
\begin{lemma} \label{lem:mem-ideal}
	For an ideal $I$ of a ring $R$, for any $a,b \in R$ 
	\[ a+I = b + I \iff a-b \in I \]
\end{lemma}
\begin{proof}
	($\Longrightarrow$) Let $a+I = b+I$. In particular $a \in a+I$. Hence
	$a \in b+I$ and $a-b \in I$.

	($\Longleftarrow$) We show that if $a-b \in I$, then $a+I \subseteq
	b+I$. Let $a-b = \alpha$ for some $\alpha \in I$. Hence $a = b +
	\alpha$ which implies $a \in b+I$. Now for any $c \in a+I$, $c \in
	b+(a-b)+I$. Since $a-b = \alpha \in I$, $(a-b)+I = I$ giving $c \in
	b+I$. Now interchange the role of $a$ and $b$ in the above proof to
	get $b+I \subseteq a+I$. This completes the proof.
\end{proof}

Let us consider an example of quotient rings. As usual, consider the ring $\Z$
and let $a\Z$ for $a \in \Z$ be an ideal. Recall that $\Z$ is a PID and hence
every ideal is singly generated. Consider the quotient ring $\Z/a\Z$. 
By earlier lemma (Lemma~\ref{lem:mem-ideal}), we get that 
\begin{observation} \label{obs:mem-z}
	For any $\alpha + I$, $\beta + I \in \Z/a\Z$, $\alpha$ and $\beta$
	are in the same cosets if and only if $\alpha - \beta \in a\Z$. 
\end{observation}
This is essentially same as that of saying $\alpha = \beta \mod a$. 
Note that there cannot be two representatives for any coset. If $\alpha,
\beta$ represent the same coset $a\Z$ with $\alpha \ne \beta$, then
$\alpha-\beta \ne 0 \mod a$ which violates the above observation
(Observation~\ref{obs:mem-z}).

Also coset representatives is of the coset are those that are less than
$a$ in value. If not, then we can always go modulo $a$ and get a
representative smaller than $a$. Hence the set quotient ring $\Z/a\Z$ can be
looked as $\Z_a$.
\begin{claim}
	$\Z/a\Z$ is isomorphic to $\Z_a$
\end{claim}
\begin{proof}
	For any $\alpha + a\Z \in \Z/a\Z$, define the map $\phi$ which takes
	$\alpha + z\Z$ to $\alpha \mod a$. This is indeed a bijection.  
	Also the operation is preserved because,
	\begin{align*}
		\phi(\alpha + a\Z + \beta +b\Z) = & \phi(\alpha + \beta +a\Z)
		= (\alpha + \beta) \mod a \\ 
		& = (\alpha \mod a + \beta \mod a ) \mod a \\
		& = \phi(\alpha + a\Z) + \phi(\beta +a\Z)
	\end{align*}
\end{proof}

\section{Quotient rings over $\F[x]$}
For a polynomial $p(x) \in \F[x]$ with $deg(p) = d$, consider the ideal $I =
\langle p(x) \rangle$ and the quotient ring $\F[x]/I$. Similar to the integer
rings, we make the following observations.

\begin{observation}
	No two polynomials with degree strictly smaller than $deg(p)$ can be
	in the same coset. 
\end{observation}
This follows from Lemma~\ref{lem:mem-ideal} and the argument is similar to the
integer case. Hence every coset has at most one polynomial of degree strictly
smaller in it. The next observation says that every coset must have an element
of degree strictly smaller than $deg(p)$.

\begin{observation}
	Every coset in $\F[x]/I$ must have a polynomial of degree strictly
	smaller than $deg(p)$.
\end{observation}
\begin{proof}
	Let $g + I$ be a coset in $\F[x]/I$. By division algorithm, $g(x)$ can
	be written as $q(x)p(x) + r(x)$ with $deg(r) < deg(p)$ and $r$ is
	unique. Hence $r(x) = g(x) - p(x)q(x) \in g +I$.
\end{proof}

This tells us that corresponding to every polynomial of degree strictly
smaller than $deg(p)$, there is exactly one coset. Hence for a finite field
$\F$ of size $q$, $|\F[x]/I| = q^d$. 

We now give a characterisation of irreducibles.
\section{Characterisation of irreduciblity}
The characterisation essentially says that checking for irreducibilty biolds
down to checking if a ring is a field.
\begin{theorem}
	For an ideal $I$ in $\F[x]$ generated by $p(x) \in \F[x]$, 
	\[ \F[x]/I \text{ is a field} \iff p(x) \text{ is irreducible over }
	\F[x] \]
\end{theorem}
\begin{proof}
	($\Longleftarrow$)
\end{proof}
