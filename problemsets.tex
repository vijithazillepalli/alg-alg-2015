\documentclass[11pt,addpoints]{exam}

\usepackage{enumerate}
\usepackage{ifthen}
\usepackage{showlabels}
\newtheorem{theorem}{Theorem}
\newtheorem{exercise}[theorem]{Exercise}

\def\standalonepsflag{1} % 1 if standalone, 0 if in lecturenotes
\def\psnumber{1} 
%  -1 if only labelled one needs to be printed.
%   0 if exercises 
% > 0 if specific problem set number
\def\pslabel{} % ignored if psnumber is not 0.

\newcommand{\homework}[3]
{
	\ifthenelse {\equal{\psnumber}{-1}} 
	{
		\ifthenelse {\equal{\pslabel}{#2}}
		{
			\begin{exercise}
			\label{exercise:#2}
			#3 	
			\end	{exercise}
		}
	}
	{
	% If psnumber is not 0, include all psnumber matches.
	% check if it is standalone
		\ifthenelse {\equal{\psnumber}{#1}} 
		{
			\ifthenelse {\equal{\standalonepsflag}{1}}	
				{\question[7] #3} % standalone pset question.
				{\item #3} % in-lecturenote question.
		}
	}
}

\firstpageheader{\bf \large IITM-CS6842 : Algorithmic Algebra \\[2mm] Problem Set \#1 ~~ ($7 \times 5 =  35$ points)}{}{\bf \large Given on : Sep 5, 2015 \\[2mm] Due on : Sep 17, 2015}

\begin{document}

\ifthenelse {\equal{\standalonepsflag}{0}}
{
	\ifthenelse {\not\equal{\psnumber}{-1}}
	{
		\newpage
		\ifthenelse {\equal{\psnumber}{0}}
		{\section*{Exercises}}
		{\section*{Problem Set \#\psnumber}}		
		\begin{enumerate}[(\psnumber.1)]	
	}
}
{
	\ifthenelse {\not\equal{\psnumber}{-1}}
		{\begin{questions}}
}

	 
\homework{1}{graph-iso-variants}
{
We will reduce variants of graph isomorphism problem to the original problem.
\begin{enumerate}[(a)]
\item We defined the graph rigidity problem {\sc GR} as - given a graph $X$, test if $Aut(X)$ is trivial. Show that {\sc GR} $\le$ {\sc GI}.
\item Notice that the {\sc GI} is defined for undirected graphs. Define the graph isomorphism problem for directed graphs {\sc DirGI}. Show that {\sc DirGI} $\le$ {\sc GI}.
\end{enumerate}
}

\homework{1}{tree-isomorphism}
{
Design a linear-time algorithm that takes two rooted trees $T_1$ and $T_2$ as input and tests if they are isomorphic. Note that an isomorphism between $T_1$ and $T_2$ must preserve edges as well as parent-child relations.
}

\homework{1}{normalizer-normalclosure}
{
Let $G$ and $H$ be subgroups of $S_n$. Give a polynomial (in $n$) time algorithm which given the generators of a group $G$, and $H$, write down algorithms for:
\begin{enumerate}[(a)]
\item Checks if $H$ is a normal subgroup of $G$.
\item Normalizer of $H$ in $G$. That is the largest subgroup of $G$ in which $H$ is a normal subgroup.
\item Normal closure of $H$ in $G$, that is the smallest normal subgroups of $G$ that contains $H$.
\end{enumerate}
}

\homework{0}{newman-tightbound}
{
We stated in class that for any $n > 3$, every subgroup of $S_n$ can be generated by at most $\lfloor \frac{n}{2} \rfloor$ elements. Show that this bound is tight by giving an example of an $\Omega$ and a group $G \le S_{|\Omega|}$ acting on it such that $G$ requires $\frac{|\Omega|}{2}$ elements to generate it. (Hint : Consider $\Omega = \{a_1, a_2, \ldots, a_m, b_1, b_2 \ldots b_m \}$).
}

\homework{1}{primite-actions}
{
Let $G$ act transitively on $\Omega$. In class, we showed a characterization of primitive actions. Extend the argument to show the following. Let $\alpha \in \Omega$. Let $\Gamma$ be the set of all blocks $B$ which contain $\alpha$ and set $\mathcal{H}$ be a set of all subgroups of $H \le G$ that contain $G_\alpha$. Establish a bijection between $\Gamma$ and $\mathcal{H}$.
}

\homework{1}{commutator-subgroup}
{
For any group $G$, the commutator subgroup $G'$ of the groups, sometimes denoted by $[G,G]$, is defined as,
\[ [G,G] = \{g_1g_2g_1^{-1}g_2^{-1} \mid g_1, g_2 \in G \} \]
\begin{enumerate}[(a)]
\item Argue that $G/G'$ is an Abelian group. (That is $\forall a,b \in G/G'$, $ab = ba$)
\item Show that $G'$ is the smallest group such that (a) holds.
\item Give a polynomial time algorithm for the following problem : Given a generating set for $G$, compute that of $G'$. (Hint : Use one of the previous algorithmic problems in this set).
\end{enumerate}
}

\homework{0}{subgroup-cost-structure}
{
	Let $H' \lneq H \lneq G$. The cosets induced by $H$ and $H'$
	partitions $G$. Show that the partition induced by $H'$ is a
	refinement of the partition induced by $G$. That is, show that
	every cosets of $H'$ in $G$ must be contained in some coset of $H$ in
$G$.  }

\ifthenelse {\equal{\standalonepsflag}{0}}
{
	\ifthenelse {\not\equal{\psnumber}{-1}}
		{\end{enumerate}}
}
{
	\ifthenelse {\not\equal{\psnumber}{-1}}
	    	{\end{questions}}
}


\end{document}
