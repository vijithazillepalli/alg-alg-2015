\documentclass[11pt,addpoints]{exam}

\usepackage{enumerate,amssymb,amsmath}
\newcommand{\F}{\mathbb{F}}
\usepackage{ifthen}
\usepackage{showlabels}
\newtheorem{theorem}{Theorem}
\newtheorem{exercise}[theorem]{Exercise}

\def\standalonepsflag{1} % 1 if standalone, 0 if in lecturenotes
\def\psnumber{2} 
%  -1 if only labelled one needs to be printed.
%   0 if exercises 
% > 0 if specific problem set number
\def\pslabel{} % ignored if psnumber is not 0.

\newcommand{\homework}[3]
{
	\ifthenelse {\equal{\psnumber}{-1}} 
	{
		\ifthenelse {\equal{\pslabel}{#2}}
		{
			\begin{exercise}
			\label{exercise:#2}
			#3 	
			\end	{exercise}
		}
	}
	{
	% If psnumber is not 0, include all psnumber matches.
	% check if it is standalone
		\ifthenelse {\equal{\psnumber}{#1}} 
		{
			\ifthenelse {\equal{\standalonepsflag}{1}}	
				{\question #3} % standalone pset question.
				{\item #3} % in-lecturenote question.
		}
	}
}

\firstpageheader{\bf \large IITM-CS6842 : Algorithmic Algebra \\[2mm] Problem Set \#2 ~~ ($8+8+8+8+3 = 35$ points)}{}{\bf \large Given on : Oct 2, 2015 \\[2mm] Due on : Oct 14, 2015}

\begin{document}

\ifthenelse {\equal{\standalonepsflag}{0}}
{
	\ifthenelse {\not\equal{\psnumber}{-1}}
	{
		\newpage
		\ifthenelse {\equal{\psnumber}{0}}
		{\section*{Exercises}}
		{\section*{Problem Set \#\psnumber}}		
		\begin{enumerate}[(\psnumber.1)]	
	}
}
{
	\ifthenelse {\not\equal{\psnumber}{-1}}
		{\begin{questions}}
}

	 
\homework{1}{graph-iso-variants}
{
We will reduce variants of graph isomorphism problem to the original problem.
\begin{enumerate}[(a)]
\item We defined the graph rigidity problem {\sc GR} as - given a graph $X$, test if $Aut(X)$ is trivial. Show that {\sc GR} $\le$ {\sc GI}.
\item Notice that the {\sc GI} is defined for undirected graphs. Define the graph isomorphism problem for directed graphs {\sc DirGI}. Show that {\sc DirGI} $\le$ {\sc GI}.
\end{enumerate}
}

\homework{1}{tree-isomorphism}
{
Design a linear-time algorithm that takes two rooted trees $T_1$ and $T_2$ as input and tests if they are isomorphic. Note that an isomorphism between $T_1$ and $T_2$ must preserve edges as well as parent-child relations.
}

\homework{1}{normalizer-normalclosure}
{
Let $G$ and $H$ be subgroups of $S_n$. Give a polynomial (in $n$) time algorithm which given the generators of a group $G$, and $H$, write down algorithms for:
\begin{enumerate}[(a)]
\item Checks if $H$ is a normal subgroup of $G$.
\item Normalizer of $H$ in $G$. That is the largest subgroup of $G$ in which $H$ is a normal subgroup.
\item Normal closure of $H$ in $G$, that is the smallest normal subgroups of $G$ that contains $H$.
\end{enumerate}
}

\homework{0}{newman-tightbound}
{
We stated in class that for any $n > 3$, every subgroup of $S_n$ can be generated by at most $\lfloor \frac{n}{2} \rfloor$ elements. Show that this bound is tight by giving an example of an $\Omega$ and a group $G \le S_{|\Omega|}$ acting on it such that $G$ requires $\frac{|\Omega|}{2}$ elements to generate it. (Hint : Consider $\Omega = \{a_1, a_2, \ldots, a_m, b_1, b_2 \ldots b_m \}$).
}

\homework{1}{primite-actions}
{
Let $G$ act transitively on $\Omega$. In class, we showed a characterization of primitive actions. Extend the argument to show the following. Let $\alpha \in \Omega$. Let $\Gamma$ be the set of all blocks $B$ which contain $\alpha$ and set $\mathcal{H}$ be a set of all subgroups of $H \le G$ that contain $G_\alpha$. Establish a bijection between $\Gamma$ and $\mathcal{H}$.
}

\homework{1}{commutator-subgroup}
{
For any group $G$, the commutator subgroup $G'$ of the groups, sometimes denoted by $[G,G]$, is defined as,
\[ [G,G] = \{g_1g_2g_1^{-1}g_2^{-1} \mid g_1, g_2 \in G \} \]
\begin{enumerate}[(a)]
\item Argue that $G/G'$ is an Abelian group. (That is $\forall a,b \in G/G'$, $ab = ba$)
\item Show that $G'$ is the smallest group such that (a) holds.
\item Give a polynomial time algorithm for the following problem : Given a generating set for $G$, compute that of $G'$. (Hint : Use one of the previous algorithmic problems in this set).
\end{enumerate}
}

\homework{0}{subgroup-cost-structure}
{
	Let $H' \lneq H \lneq G$. The cosets induced by $H$ and $H'$
	partitions $G$. Show that the partition induced by $H'$ is a
	refinement of the partition induced by $G$. That is, show that
	every cosets of $H'$ in $G$ must be contained in some coset of $H$ in
$G$.  }

\homework{0}{properties-of-ideals}
{
Let $I$ and $J$ be ideals of $R$.
\begin{enumerate}
\item Show that $I \cap J$ is an ideal of $R$.
\item Show that the set $I+J = \{i+j \mid i \in I \land j \in J \}$ is an ideal of $R$.
\end{enumerate}
}

\homework{2}{principal-ideals}
{
We showed in class that every ideal of $\mathbb{F}[x]$ is principal (generated by one element). We will show two observations about it.
\begin{enumerate}[(a)]
\item This is special to the case of polynomials in one variable. Consider the ideal $\langle x,y \rangle \subset \mathbb{F}[x,y]$. Prove that $I$ is not a principal ideal. (Hint : If $x = fg$ where $f,g \in \mathbb{F}[x,y]$ prove that $f$ or $g$ is a constant.)
\item This is special to the case of fields. Let $I$ be a subset of $\mathbb{Z}[x]$ consisting of all polynomials with an even constant term, i.e. for $p(x) = a_0 + a_1x+a_2x^2+\ldots+a_nx^n \in \mathbb{Z}[x]$, $p \in I$ if and only if $a_0$ is even. Show that $I$ is an ideal of $\mathbb{Z}[x]$ but not a principal ideal.
\end{enumerate}
}

\homework{2}{ring-homomorphism}
{
Let $R$ and $R'$ be two commutative rings with identities. A map $\phi : R \to R'$ is called a \textit{ring homomorphism} if $\phi(1) = 1$ and,
\[ (\forall a,b) \left[ \phi(a+b) = \phi(a)+\phi(b) \textrm{ and } \phi(ab) = \phi(a)\phi(b)\right] \]
That is, $\phi$ respects (multiplicative and additive) identity, addition, and multiplication. The set of elements of $R$ which gets mapped to the additive identity of $R'$ by the homomorphism $\phi$ is called \textit{kernel} of the homomorphism.
\begin{enumerate}[(a)]
\item Show that the kernel of $\phi$ is always a subring of $R$. Is it also an ideal?. Give arguments.
\item For every ideal $I \subseteq R$, there is a ring homomorphism ($\phi : R \to R'$) such that $I$ is the kernel of $\phi$.
\item Show that image of the $\phi$ is a subring of $R'$. Is it also an ideal? Give arguments.
\end{enumerate}
}

\homework{2}{term-ordering}
{
Let $f \in \mathbb{F}[x_1,x_2, \ldots x_n]$ and $x_1 > x_2 > \ldots > x_n$ be the variable order.
\begin{enumerate}[(a)]
\item Work out the proof of the statement \textit{every term ordering is also a well-ordering} (Theorem 1.4.6, Page 21 of the textbook - half-page proof). Is the converse also true? Argue the following.
 Let $<$ be a total order on the set of terms. Assume that $<$ is a well-order and satisfies the second condition of the definition of term ordering. Prove that for the term $x^\alpha \ne 1$ satisfies $1 < x^\alpha$.
\item We call $f$ to be homogeneous if the total degree of every term is the same. Let the term ordering be degrevlex. Prove that $x_n$ divides $f$ if and only if $x_n$ divides $lt(f)$. Generalize your argument to show $f \in \langle x_i, \ldots x_n \rangle$ if and only if $lt(f) \in \langle x_i, \ldots x_n \rangle$.
\end{enumerate}
}

\homework{0}{}{
\begin{enumerate}[(a)]
\item Show that there are ${n+d-1 \choose d}$ monomial of total degree $d$ in $n$ variables.
\item Let $f_1, f_2, \ldots, f_k \in \F[x]$ be univariate polynomials.\\
Prove that $GCD(f_1, \ldots, f_k) = GCD(f_1, GCD(f_2, \ldots, f_k))$.
\end{enumerate}
}

\homework{0}{}{
Let $f,g \in \F[x_1, x_2, \ldots, x_n]$ be nonzero polynomials and let $multdeg(f)$ denote the mult-degree of $f$. Argue how the leading terms of monomials change under addition and mutliplication. Use it to show that:
\begin{enumerate}[(a)]
\item $multdeg(fg)$ = $multdeg(f)+multdeg(g)$.
\item If $f+g \ne 0$, then $multdeg(f+g) \le \max(multdeg(f),multdeg(g))$.
\end{enumerate}
}

\homework{0}{}{
The usual $<$ ordering of $\mathbb{Z}_{\ge 0}$ has some nice (trivial to state) properties.
\begin{itemize}
\item ~there are only finitely many integers between any two integers. 
\item ~$\alpha > 0$ for all nonzero $\alpha \in \mathbb{Z}$. 
\end{itemize}
Let $<$ be a total order on $\mathbb{Z}_{\ge 0}^n$ that satisifes: $\forall \alpha, \beta$,  $\alpha > \beta \implies \alpha+\gamma > \beta+\gamma$.
Check if any of the above conditions are necessary and sufficient $<$  to be a monomial ordering?
}

\homework{0}{}{ 
Let $f, g \in \F[x_1, x_2, \ldots x_n]$ and $x^\alpha$ and $x^\beta$ be monomials. Prove that 
\[ S(x^\alpha.f,x^\beta.g) = x^\gamma S(f,g) \]
where
\[ x^\gamma = \frac{LCM(x^\alpha LM(f),x^\alpha LM(g))}{LCM(LM(f),LM(g))} \]
}

\homework{2}{}{
Let $I \subseteq \F[x_1, x_2, \ldots x_n]$ be an ideal generated by a (possibly infinite) set of power products (such ideals are called {\em monomial ideals}).
\begin{enumerate}[(a)]
\item Prove the following stricter version of Hilbert basis theorem for monomial ideals : there exists $\alpha_1, \alpha_2, \ldots \alpha_m \in \mathbb{N}^n$ such that $I = \langle x^\alpha_1, x^\alpha_2, \ldots, x^\alpha_m \rangle$.\\ (Hint for one possible solution : Equivalently, you can show the following fact about natural numbers : for any $A \subseteq \mathbb{N}^n$, there exists $\alpha_1, \alpha_2, \ldots \alpha_m \in A$ such that $A \subseteq \bigcup_{i=1}^m(\alpha_i + \mathbb{N}^n)$. Show the equivalence if you are using the hint.)
\item Prove that every monomial ideal contain a unique minimal generating set. That is, prove that there is a subset $G \subseteq I$ generating $I$ such that for all subsets $F \subseteq I$ generating $I$, we have that $G \subseteq F$.
\end{enumerate}
}

\homework{2}{}{ 
Let $I$ be an ideal and $G$ be a Grobner basis of $I$. For any $f \in \F[x_1, x_2, \ldots x_n]$, let $\overline{f}^G$ denote the unique $r$ such that $f \rightarrow^G_{+} r$.
%\begin{enumerate}[(a)]
%\item Show that $\overline{f}^G = \overline{g}^G$ if and only if $f - g \in I$.
Use this to deduce that : $\overline{f+g}^G = \overline{f}^G + \overline{g}^G$.
%\end{enumerate}
}

\homework{0}{}{ 
Show that the result of applying the Euclidean Algorithm in $\F[x_1,x_2, x_n]$ to any pair of polynomials $f,g$ is a reduced Groebner basis for $\langle f,g \rangle$ (after dividing by a constant to make the leading coefficient equal to 1). Explan how the steps of the Euclidean Algorithm can be seen as special cases of the operations used in Buchberger's algorithm.
}


\ifthenelse {\equal{\standalonepsflag}{0}}
{
	\ifthenelse {\not\equal{\psnumber}{-1}}
		{\end{enumerate}}
}
{
	\ifthenelse {\not\equal{\psnumber}{-1}}
	    	{\end{questions}}
}


\end{document}
