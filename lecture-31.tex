\Lecture{Jayalal Sarma}{September, 29 2015}{31}{Multivariate Division Algorithm}{Subhadra Nanda}{$\alpha$}{K Dinesh}
\section{Recap}
In previous lectures,we saw there are two cases of polynomial rings, univariate case and linear or multivariate case. 

We also discussed about two key ingredients of polynomials, those are : 1. Term Ordering and 2. Division Algorithm.

In the last lecture we defined two properties of term ordering(definition 30.3) and showed that these properties follow well ordering principle, which helps to terminate the division algorithm.
We also talked about the three possible term ordering, 1. Lex ordering, 2.Degree Lex Ordering and 3. Degree Reverse Lex Ordering.

Now, we will see some examples on term ordering.

\section{Example of Term Ordering}

Let $f$ be a polynomial, $$ f = 2x^2yz + 3xy^3 - 2x^3 $$. Where, $x>y>z$ in terms of order.

Let's define some terminologies on polynomial $f$ :

$lc(f)$ : leading coefficient of $f$

$lp(f)$ : leading power product of $f$

$lt(f)$: leading term of $f$

\subsection{Using Lex ordering}

$$x^2yz \Rightarrow (2,1,1)$$
$$xy^3 \Rightarrow (1,3,0)$$
$$x^3 \Rightarrow (3,0,0)$$

So, $lt(f) = -2x^3$ as $(3,0,0)>(2,1,1)>(1,3,0)$ and $lc(f) = -2$ and $lp(f) = x^3$.

\subsection{Using Degree Lex ordering}

It sorts the terms using total degree first and then uses lex ordering.

$$x^2yz \Rightarrow total~degree~4 $$
$$xy^3 \Rightarrow total~degree~4$$
$$x^3 \Rightarrow total~degree~3$$

As $x^2yz$ and $xy^3$ have same total degree, sort them using lex ordering.

So,  $lt(f) = 2x^yz$ as $(2,1,1)>(1,3,0)$ and $lc(f) = 2$ and $lp(f) = x^2yz$.

\subsection{Using Degree Reverse Lex ordering}

It sorts the terms using total degree first and then uses reverse lex ordering.

As $x^2yz$ and $xy^3$ have same total degree, sort them using reverse lex ordering.

So,  $lt(f) = 3xy^3$ as $(2,1,1)>(1,3,0)$ and $lc(f) = 3$ and $lp(f) = xy^3$.

\section{Multivariate Division}
Let, polynomials $f,g,h \in  \mathbb{F}[x_1, \ldots ,x_n]$.

We will use new notations $f \mathop \rightarrow \limits^{g} h$, which implies "f reduces to h using g in one step",

and $f {\mathop \rightarrow \limits^{g}}_{+} h$, which implies "f reduces to h using g in multiple steps".

In case of univariate polynomial division, we did
$$ h = f - \frac{lt(f)}{lt(g)}\cdot g$$

By doing this we ensure that $h$ is descending in chain of term ordering, so that the division algorithm terminates.

Now, we will generalize it for  multivariate polynomials.

\begin{definition}
We say, $f \mathop \rightarrow \limits^{g} h$ if and only if $lp(g)$ divides one of the non-zero terms(say $X$) in the polynomial $f$.

$$ h = f - \frac{X}{lt(g)}\cdot g$$
\end{definition}

\begin{observation}
Term $X$ gets removed completely from the polynomial $f$.
\end{observation}
\begin{observation}
$h$ introduces new terms by $\frac{X}{lt(g)}\cdot g$, which are less than the term $X$.
\end{observation}

\begin{exercise} Proof of the above observations.
\end{exercise}

\section{examples}
Let, $f$ and $g$ are two polynomials.
$$ f = 6x^2y - x + 4y^3 -1$$
$$ g = 2xy + y^3$$

Assume, $x>y$ in lex ordering.

\subsection{Using Lex Ordering}
$lt(g) = 2xy$ and $lp(g) = xy$, and $xy$ divides the term $6x^2y$ of f. So, $ X = 6x^2y$ here.

Therefore, $$ h = 6x^2y - x + 4y^3 -1 - \frac{6x^2y}{2xy}\cdot {(2xy + y^3)}$$
$$ = -x + 4y^3 -1 -3xy^3$$
$$ = -3xy^3 - x + 4y^3 - 1$$

We can observe that, the term $6x^2y$ is removed from $h$ and a new term $3xy^3$ is introduced here, which is lesser than $6x^2y$.

\subsection{Using Degree Lex Ordering}
$lp(g) = y^3$ and it divides the term $4y^3$ of $f$.

Therefore, $$ h = 6x^2y - x + 4y^3 -1 - \frac{4y^3}{y^3}\cdot {(2xy + y^3)}$$
$$= 6x^2y - 8xy -x -1$$

Observe that, the term $4y^3$ is completely removed and a new term $8xy$ is introduced, which is lesser than $4y^3$.

\section{Comparison between multivariate division algorithm and the long division process}

Let's take an example to show the long division process.

Let, $f$ and $g$ are to multivariate polynomials.
 
$ f = y^2x + 4yx - 3x^2$

$ g = 2y + x + 1$

Let, $y>x$ in terms of lex order. Divide $f$ by $g$ :

\newcommand{\ldsym}{$\left.\mathstrut\right)$}% unbalanced )
\newlength{\ldwidth}
\newcommand{\longdivide}[2]% #1 = denominator, #2 = numerator
{\settowidth{\ldwidth}{\ldsym}
#1\,\raisebox{1.5pt}{\ldsym}\hspace*{-.65\ldwidth}\overline{
\mathstrut\hspace*{.35\ldwidth}\ #2}}

\begin{center}
\begin{tabular}{r}
	$\frac{yx}{2} - \frac{x^2}{4} + \frac{7x}{4}$  \hphantom{$\strut-3x^2$} \\
	$\longdivide{2y+x+1}{y^2x+4yx-3x^2\hphantom{\strut-3x^2}}$ \hphantom{$\strut-3x^2$} \\
	\underline{$y^2x+\frac{yx^2}{2}+\frac{yx}{2}$} \hphantom{$\strut-3x^2 $}  \hphantom{$\strut-3x^2$} \\
	$\frac{-yx^2}{2}+\frac{7yx}{2} - 3x^2$ \hphantom{$\strut-3x^2$} \\ 
	\underline{$\frac{-yx^2}{2}-\frac{x^3}{4} - \frac{x^2}{4}$} \hphantom{$\strut-3x^2$} \\
	$\frac{7yx}{2}-\frac{x^3}{4}-\frac{11x^2}{4}$  \\
	\underline{$\frac{7yx}{2}-\frac{7x^2}{4}+\frac{7x}{4}$} \\
	$\frac{x^3}{4}-\frac{9x^2}{2}-\frac{7x}{4}$ 
\end{tabular}
\end{center}

Here in the first step, we have chosen $y^2x$ as term $X$ to be divided by $lt(g) = 2y$.

Now, the question is, if we choose the term $X$ differently, will it end up with the same polynomial? Does the choice of $X$ really matter?

We will try to give the answer of this question later.

\section{Generalization of multivariate division algorithm}

Given, $F = \{f_1,f_2, \ldots ,f_k\}$ and $f,h,f_1,f_2, \ldots f_k \in \mathbb{F}[x_1,\ldots,x_n]$.

We can say, $f {\mathop \rightarrow \limits^{F}}_{+} h$, if $\exists i_1,i_2, \ldots ,i_t$ and $h_1,h_2, \ldots ,h_3$ such that,

$$f \mathop \rightarrow\limits^{f_{i_1}} h_1 \mathop \rightarrow\limits^{f_{i_2}} h_2 \mathop \rightarrow\limits^{f_{i_3}}\ldots \ldots \rightarrow h_{t-1} \mathop \rightarrow\limits^{f_{i_t}} h$$

\begin{observation}
The polynomial $h$ will have no term that can be divided by any leading term of any of those polynomials $f_1,...,f_k$.
\end{observation}
In the division process we saw that,

$$ f = h + \frac{X}{lt(g)}\cdot g  = h + u\cdot g$$ where, $u = \frac{X}{lt(g)}$.

Similarly, we can generalize it as
$$ f = u_{i_1}f_{i_1} + u_{i_2}f_{i_2} + \ldots + u_{i_t}f_{i_t} + h $$

We can write it as,

$$ f = u_1f_1 + u_2f_2 + \ldots + u_kf_k + h $$

\begin{observation}
Using the above process we can solve the Ideal Membership problem, where the ideal $ I = \langle f_1,f_2, \ldots ,f_k \rangle$ and a polynomial $f \in \mathbb{F}[x_1,...,x_n]$ is given. We have to check whether $f$ is a member of the ideal $I$.

\end{observation}


We will discuss on this ideal membership problem in the next lecture.

 \begin{exercise}
Let $f,g \in \F[x_1, x_2, \ldots, x_n]$ be nonzero polynomials and let $multdeg(f)$ denote the mult-degree of $f$. Argue how the leading terms of monomials change under addition and mutliplication. Use it to show that:
\begin{enumerate}[(a)]
\item $multdeg(fg)$ = $multdeg(f)+multdeg(g)$.
\item If $f+g \ne 0$, then $multdeg(f+g) \le \max(multdeg(f),multdeg(g))$.
\end{enumerate}
\end{exercise}


\begin{exercise}
Let $f, g \in \F[x_1, x_2, \ldots x_n]$ and $x^\alpha$ and $x^\beta$ be monomials. Prove that 
\[ S(x^\alpha.f,x^\beta.g) = x^\gamma S(f,g) \]
where
\[ x^\gamma = \frac{LCM(x^\alpha LM(f),x^\alpha LM(g))}{LCM(LM(f),LM(g))} \]
\end{exercise}


















