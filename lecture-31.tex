\Lecture{Jayalal Sarma}{September, 29 2015}{31}{Multivariate Division
Algorithm}{Subhadra Nanda}{$\gamma$}{K Dinesh} 

\section{Recap}
In previous lectures, we saw how to handle ideal membership question when the
set of polynomials are univariate or linear equations.  We also discussed
about two key ingredients (1) term ordering and (2) division algorithm that 
needs to be define carefully to generalize our earlier approach.

In the last lecture we defined two properties of term ordering
(Definition~\ref{def:term-order}) and showed that these properties 
implies that the ordering is well ordered. We want the ordering to be well
ordered as this helps us to argue termination of the division algorithm.  We 
also talked about the three term ordering : (1) Lex ordering (lex) (2) 
Degree lex ordering (deglex) and (3) Degree reverse lex ordering (degrevlex).
Now, we will see some examples on term ordering.

\section{An example demonstrating term orderings}

Let $f \in \polyring$, $f = 2x^2yz + 3xy^3 - 2x^3$. with the variable ordering
$x>y>z$.

Recall the following terminologies on polynomial $f$.
\begin{enumerate}
	\item $lc(f)$ : leading coefficient of $f$
	\item $lp(f)$ : leading power product of $f$
	\item $lt(f)$: leading term of $f$
\end{enumerate}
We give the value of $lc(f), lp(f)$ and $lt(f)$ for the three ordering
mentioned before.
\begin{center}
	\begin{tabular}{|l|c|c|c|}
		\hline 
		  & lex & deglex & degrevlex  \\ 
		\hline \hline
		$lc(f)$ & $-1$ & $2$ & $3$ \\ 
		\hline 
		$lp(f)$ & $x^3$ & $x^2yz$ & $xy^3$ \\ 
		\hline 
			$lt(f)$ & $-2x^3$ & $2x^2yz$ & $3xy^3$ \\ 
			\hline
		\end{tabular}
\end{center}
Following is the justification for the table entries.
\begin{itemize}
	\item (Lex ordering) The terms in $f$, $x^2y, xy^3, x^3$ can be
		represented as $(2,1,1), (1,3,0), (3,0,0)$. Hence
		$lt(f) = -2x^3$ as $(3,0,0)>(2,1,1)>(1,3,0)$. Also
		$lc(f) = -2$ and $lp(f) = x^3$.
	\item (Degree Lex ordering)
	It sorts the terms using total degree first and then uses lex
	ordering.  Total degrees of $x^2y, xy^3, x^3$ are $4, 4, 3$
	respectively.  As $x^2yz$ and $xy^3$ have same total degree,
	we sort them using lex ordering.  So, $lt(f) = 2x^yz$ as
	$(2,1,1)>(1,3,0)$ and $lc(f) = 2$ and $lp(f) = x^2yz$.  
	\item (Degree Reverse Lex ordering) It sorts the terms using
	total degree first and then uses reverse lex ordering.
	As $x^2yz$ and $xy^3$ have same total degree, we sort
	them using reverse lex ordering.  So, $lt(f) = 3xy^3$
	as $(2,1,1)>(1,3,0)$ and $lc(f) = 3$ and $lp(f) =
	xy^3$.
\end{itemize}

\section{Multivariate Division}
Let $f,g,h \in  \mathbb{F}[x_1, \ldots ,x_n]$. We denote $f \stackrel{g}{\longrightarrow} h$, to imply ``$f$ reduces to $h$ using $g$ in
one step''.  We denote $f \stackrel{g}{\longrightarrow}_{+} h$, to imply ``$f$
reduces to $h$ using $g$ in multiple steps''.
 
We generalize the ideas that we saw in univariate division.  In case of
univariate polynomial division, we did $ h = f - \frac{lt(f)}{lt(g)}\cdot g$.
By doing this we ensure that $h$ is smaller in the term ordering, so that the
division algorithm terminates.

Now, we will generalize it for  multivariate polynomials.

\begin{definition}
We say, $f \stackrel{g}{\longrightarrow} h$ if and only if $lp(g)$ divides
one of the non-zero terms (say $X$) in the polynomial $f$ and 
$$ h = f - \frac{X}{lt(g)}\cdot g$$
\end{definition}

\begin{exercise} 
	Use the properties of term ordering to show the following. For  $f
	\stackrel{g}{\longrightarrow}_{+} h$ with $h = f -
	\frac{X}{lt(g)}\cdot g$.
	\begin{enumerate}
		\item Term $X$ gets removed completely from the polynomial $f$.
		\item Any new terms generated in $\frac{X}{lt(g)}\cdot g$, is
			less than $X$ in the term ordering.
	\end{enumerate}
\end{exercise}

Let us consider an example with $ f = 6x^2y - x + 4y^3 -1 $ and $ 
g = 2xy + y^3$. Assume, $x>y$ in term ordering.  We demonstrate division for
lex ordering and degree lex ordering.
\begin{description}
	\item[ Using Lex Ordering ] We have $lt(g) = 2xy$ and $lp(g) = xy$,
		and $xy$ divides the term $6x^2y$ of $f$. So, $X = 6x^2y$. 
		Therefore,
		\begin{align*}
		 h & = 6x^2y - x + 4y^3 -1 - \frac{6x^2y}{2xy}\cdot
		{(2xy + y^3)}\\
		&  = -x + 4y^3 -1 -3xy^3  \\
		& = -3xy^3 - x + 4y^3 - 1
		\end{align*}
		We can observe that, the term $6x^2y$ is removed from $h$ and
		a new term $3xy^3$ is introduced here, which is lesser than
		$6x^2y$.
	\item[Using Degree Lex Ordering] We  have $lp(g) = y^3$ and it divides
		the term $4y^3$ of $f$.  Therefore, 
		\begin{align*}
		 h & = 6x^2y - x + 4y^3 -1- \frac{4y^3}{y^3}\cdot {(2xy +
		 y^3)}\\
		 & = 6x^2y - 8xy -x-1
		\end{align*}
		Observe that, the term $4y^3$ is completely removed and a new
		term $8xy$ is introduced, which is lesser than $4y^3$.
\end{description}

\noindent We now demonstrate multivariate polynomial division by an example
(instead of providing a pseudo code). Let, $f, g \in \polyring$  with $f =
y^2x + 4yx - 3x^2$ and $ g = 2y + x + 1$
with $y>x$ in the lex term order. We divide $f$ by $g$ via long division.

\newcommand{\ldsym}{$\left.\mathstrut\right)$}% unbalanced )
\newlength{\ldwidth}
\newcommand{\longdivide}[2]% #1 = denominator, #2 = numerator
{\settowidth{\ldwidth}{\ldsym}
#1\,\raisebox{1.5pt}{\ldsym}\hspace*{-.65\ldwidth}\overline{
\mathstrut\hspace*{.35\ldwidth}\ #2}}

\begin{center}
\begin{tabular}{r}
	$\frac{yx}{2} - \frac{x^2}{4} + \frac{7x}{4}$  \hphantom{$\strut-3x^2$} \\
	$\longdivide{2y+x+1}{y^2x+4yx-3x^2\hphantom{\strut-3x^2}}$ \hphantom{$\strut-3x^2$} \\
	\underline{$y^2x+\frac{yx^2}{2}+\frac{yx}{2}$} \hphantom{$\strut-3x^2 $}  \hphantom{$\strut-3x^2$} \\
	$\frac{-yx^2}{2}+\frac{7yx}{2} - 3x^2$ \hphantom{$\strut-3x^2$} \\ 
	\underline{$\frac{-yx^2}{2}-\frac{x^3}{4} - \frac{x^2}{4}$} \hphantom{$\strut-3x^2$} \\
	$\frac{7yx}{2}-\frac{x^3}{4}-\frac{11x^2}{4}$  \\
	\underline{$\frac{7yx}{2}-\frac{7x^2}{4}+\frac{7x}{4}$} \\
	$-\frac{x^3}{4}-\frac{9x^2}{2}-\frac{7x}{4}$ 
\end{tabular}
\end{center}

The long division gives us quotient as $\frac{yx}{2} - \frac{x^2}{4} +
\frac{7x}{4}$ and remainder $-\frac{x^3}{4}-\frac{9x^2}{2}-\frac{7x}{4}$.
In the division process above, we could have chosen $y^2x$ or $4yx$ as $X$ as
both of them are divisible\footnote{$2y$ can divide $y^2x$ if we choose
$\F$ as rational field.} by $lt(g) = 2y$. But we chose $y^2x$ as term
$X$.

So a natural question is : if we choose the term $X$ differently, will it end 
up with the same quotient and remainder (or at least the same remainder) ? 
Does the choice of $X$ really matter?

\section{Generalization of multivariate division algorithm}
Before, we try to answer this, let us first try to generalize multivariate
polynomial division further and then answer this question there.

\begin{definition}
Given, $F = \{f_1,f_2, \ldots ,f_k\}$ and $f,h,f_1,f_2, \ldots f_k \in
\mathbb{F}[x_1,\ldots,x_n]$.  We say, $f \stackrel{g}{\longrightarrow} h$, if
$\exists$ indices $i_1,i_2, \ldots ,i_t \in [k]$ and $h_1,h_2, \ldots ,h_{t-1}
\in \polyring$ such that, 
\[f \stackrel{f_{i_1}}{\longrightarrow} h_1\stackrel{f_{i_2}} {\longrightarrow}
h_2 \stackrel{ f_{i_3}}{\longrightarrow}\ldots \longrightarrow h_{t-1}
\stackrel{f_{i_t}}{\longrightarrow} h \]
\end{definition}

\begin{observation}
	Note that while performing the division, we need inverse of elements.
	Hence the fact that we are working over $\F$ helps.
\end{observation}
\begin{observation}
By division algorithm, the polynomial $h$ will have no term that can be
divided by leading term of any of those polynomials $f_1,\ldots ,f_k$.
\end{observation}
In the multivariate division process we saw that, $$ f = h +
\frac{X}{lt(g)}\cdot g  = h + u\cdot g$$ where, $u = \frac{X}{lt(g)}$.

Similarly, we can generalize this statement here as
$$ f = u_{i_1}f_{i_1} + u_{i_2}f_{i_2} + \ldots + u_{i_t}f_{i_t} + h $$
Since some of the $f_i$s can be the same, the above expression can be better
written as,
$$ f = u_1f_1 + u_2f_2 + \ldots + u_kf_k + h $$

\begin{observation}
Using the above process we can solve the Ideal Membership problem, where given
ideal $I = \langle f_1,f_2, \ldots ,f_k \rangle$ and a polynomial $f \in
\mathbb{F}[x_1,...,x_n]$ we check if $f \in I$. Now the problem is reduced to 
checking if $f \stackrel{F}{\longrightarrow}_{+} 0$ where $F = \{f_1,f_2,
\ldots ,f_k\}$.
\end{observation}

Let us now get back to the question we had asked. In ideal membership for
univariate case, it suffices to compute the GCD of the polynomials and check
if $f$ (whose membership in ideal is to checked) gives zero remainder when
divided by GCD. But is this a sufficient condition in multivariate case too ?
For instance, what if for some ordering of $F$, we get that the remainder is
$0$ while for others we get it to be non-zero ? Can we construct such 
an example ? We will discuss more on this problem and how to address it in the next lecture. 

Following are some exercise related to today's lecture.
\begin{exercise}
Let $f,g \in \F[x_1, x_2, \ldots, x_n]$ be nonzero polynomials and let $multdeg(f)$ denote the mult-degree of $f$. Argue how the leading terms of monomials change under addition and multiplication. Use it to show that:
\begin{enumerate}[(a)]
\item $multdeg(fg)$ = $multdeg(f)+multdeg(g)$.
\item If $f+g \ne 0$, then $multdeg(f+g) \le \max(multdeg(f),multdeg(g))$.
\end{enumerate}
\end{exercise}


\begin{exercise}
Let $f, g \in \F[x_1, x_2, \ldots x_n]$ and $x^\alpha$ and $x^\beta$ be monomials. Prove that 
\[ S(x^\alpha.f,x^\beta.g) = x^\gamma S(f,g) \]
where
\[ x^\gamma = \frac{LCM(x^\alpha lp(f),x^\alpha lp(g))}{LCM(lp(f),lp(g))} \]
\end{exercise}


