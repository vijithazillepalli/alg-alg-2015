\Lecture{Jayalal Sarma}{Aug 31, 2015}{15}{Transitivite Group Actions}{Mitali Bafna}{$\beta$}{K Dinesh}
\newtheorem{defn}{Defn}
\setlength{\parindent}{0pt}

Lets consider some examples of Group Action.

\begin{itemize}
\item
Consider the complete binary tree of depth $k$. Since any isomorphism will send leaves to leaves, $G = Aut(X)$ acts on the leaves and we observe that given any 2 leaves we can get an isomorphism which sends one leaf to the other.\\

\item
Consider $X = k$ triangles. $Aut(X)$ acts on the vertices and here too given any two vertices there is a permutation in $Aut(X)$ that sends one to the other. \\

\end{itemize}


These 2 examples motivate the following definition:

\begin{defn}{Transitivity:}
G acts on $\Omega$ transitively if there is only one orbit, i.e. $\forall \alpha,\beta \in \Omega$, $\exists g \in G$, such that $\alpha^g = \beta$.
\end{defn}


In example 2, any $g \in Aut(x)$ will send a triangle to itself or another triangle. Similarly for example 1, this property holds for the leaves taken in pairs, where the pair has all ancestors common.\\

\begin{defn}{Block:}
$\Delta$ is a said to be a block if $\forall g \in G, \Delta^g = \Delta$ or $\Delta^g \cap \Delta = \phi$.
\end{defn}

\begin{defn}{Primitive:}
We say that the group action is primitive if there are no blocks other than $\Omega$ and the singletons. 
\end{defn}

Examples of block structure:

\begin{itemize}
\item
$G = S_n$ 

Action is trivially transitive. No non-trivial blocks hence primitive.

\item
$G \leq S_n$ acts on itself (right-multiplication, $\Omega = G$).

This is a transitive action because given $g,h \in G, g' = g^{-1}h$ will take $g \rightarrow h$. If $G$ has a non-trivial subgroup $H$ then the action is not primitive. Namely the cosets of $H$ in $G$ form the non-trivial blocks. We will see in the next lecture that the converse holds too. 

({\sf Hint}: consider $SetStab(\Delta)$.)

\end{itemize}

\begin{defn}{Block System:}
A partition of $\Omega$ into sets such that each part is a block under the action of G.
\end{defn}

{\sf A Series of Claims:}

\begin{itemize}
\item
Claim 1: $\Delta^g$ is a block if $\Delta$ is a block. 

\begin{proof}
Suppose not. Implies $\exists h \in G, \Delta^{gh} \ne \Delta^g$ and $\Delta^{gh} \cap \Delta^g \ne \phi$.

Consider the action of $g' = (gh)^{-1}$ on $\Delta^{gh}, \Delta^g$. We have that $(\Delta^{gh})^{g'} = \Delta$.

Since $\Delta^{gh} \ne \Delta^g \Rightarrow \exists \beta \in \Delta^{gh}, \beta \notin \Delta^g$. Since any group element permutes $\Omega$, $x^g = y^g \Leftrightarrow x = y$. So $\beta^{g'} \in \Delta$ and $\notin \Delta^{gg'}$.

But $\Delta^{gh} \cup \Delta^g \ne \phi \Rightarrow \exists \alpha \in \Delta^{gh}, \in \Delta^g$. This means that $\alpha^{g'} \in \Delta$ and $\in \Delta^{gg'}$. 

Combining these two we get that $\Delta \ne \Delta^{gg'}$ and $\Delta \cap \Delta^{gg'} \ne \phi$, which contradicts the fact that $\Delta$ is a block.

\end{proof}

\item
Claim 2: $|\Delta^g| = |\Delta|$

This follows from the fact that $g \in G$ permutes the elements of $\Omega$.

\item

Claim 3: $\bigcup_{g \in G} \Delta^g = \Omega$

Let $k \in \Delta$ and $k' \in \Omega$. Since the action is transitive $\exists g, k^g = k'$. So $k' \in \Delta^g$. Hence $\Omega \subseteq \bigcup_{g \in G} \Delta^g $. The other containment is trivial and hence they are equal.

\item

Claim 4: $\Delta^{g_1} \cap \Delta^{g_2} \ne \phi \Rightarrow \Delta^{g_1} = \Delta^{g_2}$. 

This is true because $\Delta^{g_1}$ is a block (Claim 1).

\item

Claim 5: For any block $\Delta , |\Delta|$ must divide $|\Omega|$.

The above claims prove that $\Omega$ is partitioned into subsets with equal cardinality, one of which is $\Delta$. So $|\Delta|$ must divide $|\Omega|$. 

\item
Claim 6:

If $|\Omega |$ is prime, the group action is bound to be primitive. 

\end{itemize}

\Lecture{Jayalal Sarma}{Sep 1 2015}{15}{Characterisation of Primitivity}{Mitali Bafna}{$\beta$}{K Dinesh}

\begin{defn}{Maximal Subgroup:}
$H \leq G$ is a maximal subgroup if $\not \exists H',H < H' < G$.
\end{defn}

\begin{lemma}
Let $G$ act transitively on $\Omega$. The action is primitive iff $\forall \alpha \in G, G_\alpha$ is a maximal sugroup of G. 
\end{lemma}

\begin{proof}
The Backward Direction:

We will prove the contrapositive that is: Action is not primitive $\Rightarrow$ $\exists \alpha, G_\alpha$ is not a maximal subgroup.

Suppose $G's$ action is not primitive. Take any $\alpha \in \Omega$. $\{\alpha\} \subsetneq \Delta \subsetneq \Omega$. 

Take $H = Setstab(\Delta)$. $H$ is a subgroup of G.

Let $g \in G_\alpha$. Since $\alpha^g = \alpha$, $\Delta^g = \Delta$ and $g \in H$. Hence $G_\alpha \leq H$.

Let $\beta \neq \alpha \in \Delta$ and $g, \alpha^g = \beta$. $g \notin G_\alpha$ but $g \in H$. 

Hence $G_\alpha \lneq H$ and $G_\alpha$ is not maximal. \\

The Forward Direction:

We will again prove the contrapositive that is: $\exists \alpha, G_\alpha$ is not a maximal subgroup $\Rightarrow$ Action is not primitive.

We have that $G_\alpha \lneq H \lneq G$. We will produce a $\Delta$ such that $\Delta$ is a non-trivial block.


Define $\Delta = \alpha^H$. \\

From the Orbit-Stabiliser Theorem(Section 5.1) we have that each coset of $G_\alpha$ in $G$ corresponds to a different element in the orbit of $\alpha$. That is all the elements in a coset take $\alpha$ to the same element and if two cosets are different then their action on $\alpha$ is too. Mathematically, $g_1G_\alpha = g_2G_\alpha \Leftrightarrow \alpha^{g_1} = \alpha^{g_2}$.\\

Since $G_\alpha \lneq H,\exists$ a coset $C \neq G_\alpha \in H$. \\$G_\alpha \neq C \Rightarrow \alpha = \alpha^{G_\alpha} \not = \alpha^C = \beta$ But $\beta = \alpha^C \subseteq \alpha^H = \Delta$. So we have that $\{\alpha\} \subsetneq \Delta$. \\

Since $H \lneq G$ and $G_\alpha \lneq G, \exists$ a coset $C$ of $G_\alpha$ in $G$ which doesn't belong to $H$. Since different cosets take $\alpha$ to different elements $\alpha^C \notin \alpha^H$. So we have that $\Delta = \alpha^H \subsetneq \alpha^G = \Omega$. \\

We now prove that $\Delta$ is indeed a block. 

Suppose $\Delta^g \cap \Delta \neq \phi$. We get the following set of implications:

$\delta \in \Delta^g,\Delta \\
\Rightarrow \exists h',h \in H, \delta = \alpha^{h'g} = \alpha^h \\
\Rightarrow h'gh^{-1} \in G_\alpha \subseteq H \\
\Rightarrow g \in H \\
\Rightarrow \Delta^g = \alpha^{Hg} = \alpha^H = \Delta$. 

This means that $\Delta^g \cap \Delta \neq \phi \Rightarrow \Delta^g = \Delta$ or in other words $\Delta$ is a block.

\end{proof}





