\Lecture{Jayalal Sarma}{August 19, 2015}{11}{Pointwise Stabilizer, Membership
Testing and Group Intersection}{Samir Otiv}{$\gamma$}{K Dinesh}

\section{Recap}
We first attacked the Membership Problem. We saw that this reduces to the
problem of Order Computation. (See section~\ref{sec:comp-questions}) 

To solve this problem of Order Computation, we use the
Orbit-Stabiliser Lemma. This involved finding out the size orbit of an element
$\alpha$ (which we reduced to Graph Reachability), and then recursively
finding the size of $G_\alpha$ which is the stabiliser group of $\alpha$.
Product of these two gives the order of the group.
%We use the tower of subgroups to do this.

Note that in the above approach, $G$ is specified via a generating set. Hence
before we recurse, we need to obtain a generating set for
$G_\alpha$. For this purpose, we used Schreier's lemma to find a generating
set. We also observed that size of the generating set can grow very fast. Using
the reduction algorithm that Aditi very kindly explained in the previous
lecture, we could ensure that the size of the generating set always stayed 
within $O(n^2)$ (where $n = |\Omega|$).

\section{Finding Coset Representatives in the tower of subgroups}
\label{sec:find-coset}
Now that we have obtained a way of obtaining a generating set for the
subgroup, we are left with the problem of finding the orbit size. 
By Orbit-Stabiliser Lemma (Lemma~\ref{lem:os}), due to one-one correspondence
with cosets, we get that to estimate orbit size, it suffices to estimate the
number of cosets of the subgroup. In our setting this can be achieved by
finding the coset representatives in the tower of subgroups.

Denote $R$ as the set of right coset representatives of $G^{(i+1)}$  in $G^i$
for $i \ge 0$. Note that there are at most  $n-i$ cosets, since $i+1^{th}$
location can be taken to at most $n-i$ different places. Recall that the set of
locations where $i+1$ is taken to by $G^{(i)}$ is precisely $i+1$.
\[ (i+1)^{G^{(i)}}=\{ k \in \{i+1, \ldots, n\} | \exists g \in G^{(i)},
(i+1)^g = k\} \]

Let $X_i \subseteq \{i+1, \ldots, n\}$ be the orbit of $i+1$ in $G^{(i)}$ (We
have seen an algorithm to do it in Section~\ref{sec:orbit-comp}). 
For each $k \in X_i$, let $g_{i+1, k}$ be an element in $G^{(i)}$ that maps
$1$ to $k$. Note that such an element must exists in $G^{(i)}$. To find this
group element, it suffices to obtain a path from $i+1$ to $k$ in the graph of
generating set (see Section~\ref{sec:orbit-comp} for definition) and take the
product of edge labels on the path. The set $R$ is now the collection of all
$g_{i+1,k}$ for each $k \in X_i$.

\section{Algorithm for Pointwise Stabiliser Problem}
We now piece together the above ideas to show that for any
$\Sigma\subseteq\Omega$, the Generator set of $\pointstab(\Sigma) \le S_n$ can
be computed in $poly(|\Omega|)$ time. Let $i \ge 1$. Recall that
$\pointstab(\{1,2,\ldots,i\}) = G^{(i)}$. 

To compute a generating set for $G^{(i)}$
\begin{enumerate}
	\item Use the technique specified above to find coset representative for
		$G^{(1)}$ in $G$. We consider $G^{(0)} = S_n$ and use the
		generator set for $S_n$ as $\{ (1)(2)\ldots(n), (1~2), \ldots,
		(n-1~n)\}$ while constructing the graph for generating set.
	\item  Previous step gives us the set $R$ which consists of the right
		coset representatives of $G^{(1)}$ in $G^{(0)}$. Now we can
		apply Schreier's Lemma to obtain the generating set for
		$G^{(1)}$. 
	\item  Apply the {\sc Reduce} algorithm (Algorithm~\ref{alg:reduce})
		to obtain a smaller generator set.
	\item  Recurse by following the same procedure, and finding the
		generating set of $G^{(2)}$ given $G^{(1)}$, until the desired
		$G^{(i)}$ is found. 
\end{enumerate}
It can be seen the each step terminates in $\poly(|\Omega|)$ time and the
number of recursive calls is at most $i \le |\Omega|$. Hence overall runtime is
polynomial.


\section{Algorithm for Membership Testing}
Note that we already saw how to check membership of an element in a group
using an algorithm to compute order of the group. But this does not tell
anything about how the element is expressed in terms of its generators. Hence
we consider the following stronger version of Membership problem. 
\begin{problem}[Stronger Membership Test]
Given a generating set $A$ of $G$ with, $G \leq S_n$ and a $g\in S_n$, give a
representation in terms of any desired generating set, and the generating set 
itself.
\end{problem} 

\begin{algorithm}[htp!]
\caption{\textsc{MemberTest} : Algorithm for Membership Testing}
\label{alg:memtest}
\begin{algorithmic}[1]
	\Procedure{\textsc{MemberTest}}{Input : Element $g$, Generating set
$A$ for $G^{(i)}$, Index $i$}

\If{$g = id$} 
\State{return true}
\EndIf
\State{$X_i = (i+1)^{G^{(i)}}$ (Use Orbit Computation Algorithm~\ref{alg:orbit_alg})}
\State{Compute $R$, the set of coset representatives of $G^{(i+1)}$ 
in $G^{(i)}$ (see Section~\ref{sec:find-coset}) }
\State{$k = (i+1)^g$ (Image of $i+1$ on action of $g$)}
\If{$k \notin X_i$}
\State{return false}
\Else
\State{$B$ = Generating set of $G^{(i+1)}$ (Use Schreier's Lemma)}
\State{Apply Algorithm {\sc Reduce} to $B$ and denote
$A'$ for the reduced set obtained}
\State{Pick $g_{ik}$ from $R$ that maps $i$ to $k$}
\State{$g' = g \circ g_{ik}^{-1}$}
\State{return \textsc{MemberTest}$(g', A', i+1)$}
\EndIf
\EndProcedure
\end{algorithmic}
\end{algorithm}

Note that if insisting on giving the representation of $g$ in terms of $A$,
then the problem may not be solvable in polynomial time since there are groups
which require large size to represent in terms of its generators
(Proposition~\ref{prop:repr-large}). Now we call \textsc{MemberTest} 
on the input $(g,A,0)$ for checking if $g \in \langle A \rangle$ or not. It is
left as an exercise to show that the algorithm terminates in $poly(|\Omega|)$
time.

\section{Group Intersection Problem}
We consider the following two problem :
\begin{problem}[Subgroup Problem]
	Given groups $G$ and $H$ via their generating sets $A$ and $B$ 
	respectively, test if $H \leq G$
\end{problem}
\begin{problem}[Group Intersection Problem]
	Given groups $G$ and $H$ via their generating sets $A$ and $B$
	respectively, find the generating set of $G\cap H$\footnote{Verify
		that if $G$ and $H$ are groups then $G \cap H$ is also a
	group}.
\end{problem}

Since we have already given an algorithm for membership checking, Subgroup
problem has the following algorithm : for each $b\in B$ check if $b \in G$ 
using {\sc MemberTest} algorithm.

So how about Group Intersection problem ? We show that this problem is as hard
as the Set Stabilizer Problem. We show this by first reducing $\setstab$ to
$\groupintr$.
\begin{claim}
$\setstab \le \groupintr$
\end{claim}
\begin{proof}
	Given $\Sigma \subseteq \Omega$, generating set $K$ of $G$ we need to
	produce groups $J,H$ via generators $A,B$ such that $J \cap H =
	\setstab(\Sigma)$. Consider the group, $H = Sym(\Sigma) \times
	Sym(\Omega \setminus \Sigma)$ where $Sym(T)$  
	consists of all permutations defined over $T$ of length $|T|$. Hence
	$Sym(T)$ fixes $T$. Note that $H \le S_n$ and consists of all
	permutations that have $\Sigma$ mapped to $\Sigma$. Since
	$\setstab(\Sigma)$ talks about permutations in $G$ that fix $\Sigma$,
	$G \cap H$ is the $\setstab(\Sigma)$. Now it remains to give a
	generator for $Sym(\Sigma)$. It can be shown that if $\Sigma
	=\{1,2,\ldots,n\}$, then the generator set for $Sym(\Sigma)$ is
	$(1,\ldots, n), (1~2),(2~3),
	\ldots, (n-1~ n)$.
\end{proof}
In the next lecture, we will show $\groupintr$ reduces to $\setstab$ thereby
showing that they are equally hard.

\Lecture{Jayalal Sarma}{August 21, 2015}{12}{Group Intersection to Set
Stabiliser and Jerrum's Filter}{Samir Otiv}{$\gamma$}{K Dinesh}

In last lecture, we completed the discussion on algorithms for Pointwise
Stabilizer problem and Membership testing. We saw two new problems -- Subgroup
problem and Group Intersection problem and showed that Subgroup problem can be
reduced to Membership testing and hence polytime solvable while Set Stabilizer
problem reduces to Group Intersection problem. We continue with our discussion
on Group Intersection and Set Stabilizer problem.

\section{$\groupintr \le \setstab$}
We show that Group Intersection problem reduces to Set Stabilizer problem. 
Before we start, let us look at ways of combining groups.
\begin{definition}[Direct Product]
For groups $G,H$, define direct product of $G$ and $H$ denoted $G \times H$ as
the group 
\[ G \times H = \{(g,h) ~|~ g \in G,h \in H\} \]
with group operation for $(g_1, h_1), (g_2,h_2)$ defined as,
\[ (g_1,h_1)(g_2,h_2) = (g_1g_1,h_1h_2) \]

If $G,H$ are subgroups of $S_n$ acting on the set $\Omega$, then the group $G
\times H$ acts on $\Omega \times \Omega$ with the action defined as
$(\alpha,\beta)^{(g,h)}=(\alpha^g,\beta^h)$
where $\alpha, \beta \in \Omega$ and $(g,h) \in G \times H$.
\end{definition}

The reduction is as follows : given groups $G$ and $H$, consider the group $G
\times H$  and $\Sigma'=\{ (i,i)~|~ i \in \Omega \}$. To establish correctness,
we show that if we consider the action of $G \times H$ on $\Omega \times
\Omega$, then $\setstab(\Sigma')$ is $G \cap H$. More precisely,
\begin{claim}
$(G \cap H)_{duplicate} = \setstab(\Sigma')$ where $(G \cap H)_{duplicate} = \{
(g,g) ~|~ g\in G \cap H \}$ and Set Stabilizer computation is with respect to
the action of $G \times H$ on $\Omega \times \Omega$.
\end{claim}
\begin{proof} We show containment in both ways.

	 [$(G \cap H)_{duplicate} \subseteq
	\setstab(\Sigma')$] Consider $g \in G \cap H$.  Clearly
	$(g,g) \in G \times H$.  For $(i,i) \in \Sigma'$,
	$(i,i)^{(g,g)} = (i^g, i^g) \in \Sigma'$ by definition of
	$\Sigma'$. Hence $(g,g) \in \setstab(\Sigma')$.
	
	[$SetStab(\Sigma') \subseteq (G \cap H)_{duplicate}$]
	Let $(g,h) \in SetStab(\Sigma')$. By definition, $\forall
	(i,i) \in \Sigma'$ we have $(i,i)^{(g,h)} = (i^g,i^g) =(j,j)$
	for some $j \in \Omega$. In other words, $i^g = i^h$ for all
	$i$. Hence it must be that $g$ and $h$ are the same giving 
	$g= h  \in G \cap H$. Hence $(g,h) \in (G \cap
	H)_{duplicate}$.
\end{proof}

Now that we know that $\groupintr$ and $\setstab$ problems are equivalent to each other and are both seemingly harder than the graph automorphism problem which we want to solve. Our next aim would be to solve special cases of these problems.

Before we do the special cases, we would like to do one interesting optimization for the {\sc Reduce} algorithm.

\section{Jerrum's Filter: Obtaining generating set of size $n-1$}

\jsay{All the arguments are in place after the edit. A little more cleanup is to be done, will do that at the end of the semester.}

We know from Schreier's lemma how to obtain a strong generating set for $G \le
S_n$ of size $O(n^2)$. We also know that for any subgroup of $S_n$, there
exists a generating set of size $O(n\log n)$. A natural question is : can we
improve this bound further ?

Mark Jerrum showed that one can give a generator of size $n-1$ instead. The
process is algorithmic and is called as Jerrum's filter\footnote{It turns out that this can be improved to $\lfloor n/2 \rfloor$ (Neumann) and is tight.  We will not be discussing it in this lecture.}.

\subsection{Main ideas}
Given a generating set $S$ of $G$ and $G$ acts on a set $\Omega$, our aim is 
to obtain another generating set $A$ for $G$ of size $\le n-1$.

Given a set $S$ of group elements, we define a undirected graph $X_S(V,E)$ with
$V= \Omega$ and $E \subseteq V \times V$ is defined as for each $g \in S$, let
$i_g$ denote the least index moved by $g$ as $i_g$ i,e.  $\forall i \le i_g$
$i^g = i$.  Now we add the edge $(i_g,i_g^g)$ to $E$. We also 
label this edge by $g$. 

From the definition of $X_S$, the following can be observed.
\begin{observation}
	For a $g \in S$ $g^{-1}$ maps $i_g^g$ to $i_g$. Also 
	both $g$, $g^{-1}$ has the same least index. 
\end{observation}
The main idea is the following : we maintain a set $A$ of generators in such a
way that $X_A$ is acyclic. When we get a new $g \in S$, we add $g$ to $A$ and
compute $X_{A \cup \{a\}}$. Now we make some operation on this graph 
$X_{A \cup \{g\}}$ in such a way that the resultant graph has no cycles and
the updated set $A$ still generates $G$. This process is \emph{online},
since we update the graph as and when $g$ arrives and do not need the entire
$S$ a priori. Continuing in this manner, we get the $X_A$ to be a forest and
hence has $n-1$ edges.

Before explaining the details, we need the following measure.
\begin{definition}[Weight of $T$]
	Let $T \subseteq S_n$. Then \[ wt(X_T )=\sum_{g \in T} i_g \]
\end{definition}
Observe that for any $T$, $wt(T) \le |T|n$.

\subsection{The Algorithm}
The main idea of the algorithm is to maintain a set $A$ such that $X_A$ is
acyclic. In the construction, we ensure that we add edges
corresponding to element only in $A$ (as explained while we defined $X_A$) 
and remove those edges not in $A$.

At the end of the process, we get an acyclic graph which is a forest with
number of edges at most $n-1$. Hence $|A|$ is upper bounded by number of
edges in the graph which is at most $n-1$.

\begin{algorithm}[htp!]
\caption{\textsc{JerrumFilter} : Computing a generating set of size at most
$n-1$}
\label{alg:jerfilter}
\begin{algorithmic}[1] \Procedure{\textsc{JerrumFilter}}{Input :  Generating
	set $S$ for $G$ given in online fashion}
	\State{Let $A$ be generating set maintained. Initially $A = \emptyset$}
	\For{a given $g \in S$}
		\State{If $g \notin A$, then add $g$ to $A$ and 
		add $e_g = (i_g, i_g^g)$  to the graph $X_A$}
		\If{a cycle is created in $X_A$} \label{step:cyc}
		\State{Let $C$ be the unique cycle created of length $k$}
		\State{Let $i_0$  be the least index in the cycle $C$, $g_0$ 
		be an element corresponding to $i_0$}		  
		\State{Let $\{g_t\}_{t \in [k]}$ be the edge labels on the
	cycle $C$ with $g_0$ as label of edge incident to $i_0$}
		  \State{Denote $g_0 g_1^{\epsilon_1}, \ldots
		  g_k^{\epsilon_k}$ as $h$  where $\epsilon_i \in \{1,-1\}$ 
		  defines the  direction}
		  \State{Remove $g_0$ and add $h$ to the set $A$ and update
		  $X_A$ to reflect the changes}
		  \State{If $h$ is identity, ignore by not adding it to $A$}
		\EndIf
	\EndFor
\EndProcedure
\end{algorithmic}
\end{algorithm}

The algorithm is self explanatory till step~\ref{step:cyc}. Note that there is
exactly one cycle formed by adding of $e_g$. This is because, so far we have
maintained the invariant that $X_A$ is acyclic.
Let $i$ be the least index in the cycle $C$. We claim that
one of the edges incident to $i_0$ labelled by $g$ in $C$ must have $i =
i_0^g$ as its least index.  If not then the
neighbours of $i_0$ would not be the least moved element by the corresponding
group element.

Since $i_0$ is the least index in the cycle, it is fixed by all of 
$g_0, g_1,\ldots, g_k$ and therefore the least element moved by $h$ is
strictly more than $i$. Therefore, the weight of the graph increases by at
least on $1$ on performing this operation. 
Since the weight of $S$ is upper bounded by $\poly(|S|,n)$, the process 
terminates in in polynomially many steps. 
	
On repeating this with all elements, we have a generating set $A$ with an
acyclic graph $X_{A}$, and therefore $|A|\leq n-1$.

\printexercise{newman-tightbound}

