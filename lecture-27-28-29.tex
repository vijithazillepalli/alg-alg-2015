\Lecture{Jayalal Sarma}{Sep 21, 2015}{27}{Hilbert's Basis Theorem}{Manoj Kumar
Sure}{$\gamma$}{Ramya C}
We ended the last lecture by posing the following questions:
\begin{itemize}
\item Is $V$ = $\mathbb{V}(\mathbb{I}(V))$?
\item Is $I$ = $\mathbb{I}(\mathbb{V}(I))$?
\end{itemize}
The answer is \textbf{No} for both the questions.\\\\
Example for $I\neq\mathbb{I}(\mathbb{V}(I))$:\\

Consider $I = <x^2,y^2>$. Observe that $(0,0)$ is a root of both $x^2$ and $y^2$. Therefore $(0,0)\in\mathbb{V}(I)$ . Also observe that $(0,0)$ is also a root of polynomials $xy,\;x+y$. So they will be present in $\mathbb{I}(\mathbb{V}(I))$ but not in $I$, because the minimum degree of the polynomials in $I$ is 2. So, $I\neq\mathbb{I}(\mathbb{V}(I))$.\\\\
For $\mathbb{V}(\mathbb{I}(V))$, assuming the polynomials evaluate to 0 on 2 points, say $(0,0)$ and $(a,b)$.\\
The polynomials can be a straight line passing through those two points or $2^{nd}$ degree, $3^{rd}$ degree curves etc. Consider, for a straight line all the points on the straight line will be in $\mathbb{V}(\mathbb{I}(V))$.So, $V \neq \mathbb{V}(\mathbb{I}(V))$.\\\\
. 
\textbf{Question:} Given a set of points, Can we find a polynomial that has as zeroes exactly  these set of points?\\\\
If we are given only one point, then we can come up with a polynomial that is zero at exactly that point. Similarly, if we are given a finite set of points $(x_1,y_1), (x_2,y_2), \ldots,(x_n,y_n)$, we can find the polynomial that has exactly these points as zeroes.
$$P(x,y) = P_1(x,y)\times P_2(x,y)\times \ldots \times P_n(x,y)$$
where $P_i(x,y)$ has $(x_i,y_i)$ as the only zero.\\
But, for an infinite set of points, we may not be able to find a polynomial that vanishes exactly at these set of points. For instance, $X=\{(x,y)\;|\;y - \sin x=0\}$ .
\begin{definition}(Algebraic set of points)
A subset $V \subseteq \mathbb{F}^n$ is algebraic if there exists a finite set of polynomials $f_1,f_2,\ldots,f_k$ such that 
$$
	V = \{(a_1, a_2, \ldots, a_n) \in \mathbb{F}^n \;|\; \forall i\; f_i(a_1, a_2, \ldots, a_n) = 0\}
$$
\end{definition}
Any finite set of points is algebraic, but the points in $y-\sin x\;=\;0$ are not algebraic.\\
\begin{problem}(Ideal Membership Problem)

\end{problem}
Given an $I = <f_1,f_2, \ldots, f_k>$ and $g \in \mathbb{F}[x_1,x_2,\ldots,x_n]$, test whether $g\in I$ or not.


As an addition computational question we would also like to solve the following problem.


Given an $I = <f_1,f_2, \ldots, f_k>$ and $g \in \mathbb{F}[x_1,x_2,\ldots,x_n]$ if $g\in I$, can we find the representation $g_1,g_2,\ldots,g_k$ such that $$
	g\;=\;\sum_{i=1}^k g_if_i
$$

As an input to the ideal membership problem we are 
Given an ideal $I\subseteq\mathbb{F}[x_1,\ldots,x_n]$ generated by a finite set of polynomials $\{f_1,\ldots,f_k\}$. But it is natural to ask \\
Given an ideal $I\subseteq\mathbb{F}[x_1,\ldots,x_n]$ does there exists a finite set of polynomials $\{f_1,\ldots,f_k\}$ that generates whole of $I$ ? Hilbert answered this question in the affirmative.


\begin{theorem}(Hilbert's theorem)  \label{thm:ascending-chain-condition}
The following two statements are equivalent:
\begin{itemize}
\item In the ring $\mathbb{F}[x_1,x_2,\ldots,x_n]$, all the ideals are finitely generated. For every ideal $ I\subseteq \mathbb{F}[x_1,\ldots,x_n]$ there exists polynomials $f_1,f_2,\ldots,f_k\;\in\;\mathbb{F}[x_1,x_2,\ldots,x_n]$ 
such that ,$$I = <f_1,f_2,\ldots,f_k>$$
\item Any ascending chain of ideals in $\mathbb{F}[x_1,x_2,\ldots,x_n]$ is terminating. For $I_1 \subseteq I_2\subseteq\ldots I_N \subseteq \ldots$
there exists an $N > 0$ such that $I_n = I_N \forall n \geq N$.
\end{itemize}
\end{theorem}
The proof of this theorem will be discussed in the next lecture.
\Lecture{Jayalal Sarma}{Sep 21, 2015}{28}{Proof of Hilbert Basis
Theorem}{Manoj Kumar Sure}{$\gamma$}{K Dinesh}

In last class, we saw the difference between variety of ideals and ideal of
varieties. We had defined ideals based on a collection of finite set of
polynomials from the ring generating them. Now, we are interested in
ideals defined by a collection of points. We also some examples. This
naturally leads to the question of the existence of finite collection of
polynomials that can generate the ideal.

Hilbert's theorem answers this question in affirmative.
\begin{theorem}[Hilbert's Basis Theorem]
	Every ideal $I$ in $\F[x_1,\ldots,x_n]$ is finitely generated. That
	is, there exists a $k > 0$ and $f_1,f_2,\ldots,f_k \in
	\F[x_1,\ldots,x_n]$ such that $I = \langle f_1,f_2,\ldots,f_k \rangle
	$
	\label{thm:hbt}
\end{theorem}
The polynomials $f_1,\ldots,f_k$ are sometimes referred to as \emph{basis} or
\emph{generators} of ideal $I$. This theorem also says that in the Ideal
Membership Problem defined in last
% TODO : give backref
lecture, it makes sense to ask for a representation of the element $g$ in
terms of the generators. 

\section{An equivalent characterization of Hilbert's Theorem}
In last lecture, we stated two equivalent statements, one connecting ideals
being finitely generated and other on chains of ideals. We prove
the equivalence in this section.

\begin{proof}[Proof of equivalence]	% TODO : give backref
	[$1 \implies 2$] Suppose any ideal $I$ in $\F[x_1,\ldots,x_n]$ be
	finitely generated. Consider any ascending chain in
	$\F[x_1,\ldots,x_n]$. That is let $I_1,I_2,\ldots,I_k,\ldots$ be
	ideals in $\F[x_1,\ldots,x_n]$ such that $I_1 \subseteq I_2 \subseteq
	\ldots I_k \subseteq \ldots$. We need to show that this chain
	terminates.  Consider, $$ I = \bigcup_{i=1}^{\infty}I_i$$ We claim
	that $I$ is an ideal in $\F[x_1,\ldots,x_n]$ and this chain terminates
	in $I$. Note that $(I,+)$ is Abelian\footnote{This can be observed
	once we note that for any $a,b \in I$, there exists a $j\ge 1$ such that
	$a,b \in I_j$}. Now $RI = \cup_{i \ge 1} RI_i \subseteq \cup_{i \ge 1}
	I_i = I$. Hence $I$ is an ideal in $\F[x_1,\ldots,x_n]$. 
	
	By assumption that every ideal in $\F[x_1,\ldots,x_n]$ is finitely
	generated, $I$ is also finitely	generated and let $I =  \langle 
	f_1,f_2,\ldots,f_k \rangle$ where $k>0$ is finite. We now show that
	the chain terminates in $I$.

	Since $f_1 \in I$, there exists an $i$ such that $f_1\in I_i$ for
	$I_i$ in the chain. Let $i_1$ be the least integer such that $f_1
	\in I_{i_1}$. Similarly for $j \in [k]$, let $i_j$ be the least
	integer such that $f_j \in I_{i_j}$. Now choose, $N = \max_{j \in [k]}
	i_j$.  By choice of $N$, for every $i \le N$, $I_i \subseteq I_N$ and
	for every $n \ge N$, $I_n = I_N$.
	
	[$2 \implies 1$] We prove the contrapositive. Let there exist an ideal 
	$I$ that is not finitely generated in $R = \F[x_1,\ldots,x_n]$. We
	show that there exists a chain that is non-terminating. 

	Since $I$ is not finitely generated, there exists an infinite sequence 
	of polynomials, $f_1,f_2,\ldots$ which generates $I$.
	Consider, the following chain of ideals, 
	\begin{align*}
	I_1 &= \langle f_1 \rangle \\
	I_2& =\langle f_1,f_2 \rangle \\ 
	& \vdots \\
	I_k & = \langle f_1,f_2, \ldots, f_k \rangle  \\
	& \vdots 
	\end{align*}

	Clearly, $I_1\subseteq I_2\subseteq \ldots I_k \ldots$. This chain
	never terminates. If the chain
	is terminates say at $I_k$, then $f_1,\ldots,f_k$ generates $I$ 
	implying that $I$ will be finite generated which is a contradiction. 
\end{proof}

\section{Noetherian Rings and Proof of Hilbert's Theorem}
The rings that satisfy the condition (2) has a special name given after the
mathematician Emmy Noether\footnote{\url{https://en.wikipedia.org/wiki/Emmy_Noether}}.

\begin{definition}[Noetherian Rings]
	A ring $R$ is said to be Noetherian, if every ascending chain of
	ideals in $R$ is terminating.
\end{definition}
Noetherian rings have the following nice property.
\begin{lemma} \label{lem:noether}
If $R$ is Noetherian, then $R[x]$ is Noetherian.
\end{lemma}

Let us assume the above theorem and prove the main theorem of this lecture.
\begin{theorem}
Any ideal in $\mathbb{F}[x_1,\ldots,x_n]$ is finitely generated.
\end{theorem}
\begin{proof}(By induction on $n$)
Base case: $n$ = 0, the ring would be $\mathbb{F}$. It can be shown that
$\F$ has only two ideals : one is ${0}$ and the other is $\F$, both
of which can be verified to be generated by finite generating sets\footnote{
Proof is left as an exercise}. Hence, the base case holds true.

Assume that the given claim holds true for $n=k$, that is all ideals in 
$R = F[x_1,x_2,\ldots,x_k]$ are finitely generated. By definition, this implies
that, $R$ is a Noetherian ring. Using Lemma~\ref{lem:noether} on Noetherian
rings, we can say that $F[x_1,x_2,\ldots,x_k,x_{k+1}] =
(F[x_1,x_2,\ldots,x_k])[x_{k+1}]$ also must have its ideals finitely
generated. This concludes the proof.
\end{proof}

We now prove Lemma~\ref{lem:noether}.
\begin{proof}[Proof of Lemma~\ref{lem:noether}]
Let, $J$ be an ideal in $R[x]$. To show that $R[x]$ is Noetherian, it is 
sufficient to prove that $J$ is finitely generated. 
For $n \ge 0$, define 
\[ I_n = \{r\in R\;| r \text{ is the leading coefficient of a degree } n
\text{ polynomial in } J\} \cup \{0\} \]
Note that $I_0 \subseteq J$. Proof strategy is to show that for $n \ge 0$,
$I_n$ is an ideal in $R$ and show that $\{I_n\}_{n \ge 0}$ satisfy the
ascending chain condition. Now using the fact that $R$ is Noetherian, we get
that this chain must terminate. This observation helps us in getting a finite 
generator set for $J$.

We claim that $I_n$ is also an ideal of $R$. It can be seen that $(I_n,+)$ is
an Abelian group. Since, $J$ is an ideal, $(R[x]J)\subseteq J$.
To prove that $I_n$ is an ideal, we need to prove $$RI_n\subseteq I_n$$
Let, $t\in R$ and $r \in I_n$. We need to argue that $tr\in I_n$. 
To prove this, we need to show that there exists a degree $n$
polynomial\footnote{Polynomials in $J$ are treated as univariate polynomials
over $R$.} in $J$ such that $tr$ is its leading coefficient.
Since $r\in I_n$, there exists $p(x)$ with $deg(p)=n$ and $r$ as its leading 
coefficient. Consider the polynomial $t\cdot p(x)$. Note that 
the leading coefficient of this polynomial is $tr$. Since $J$ is an ideal,
$t\cdot p(x)\in J$. Also $deg(t\cdot p) = n$. Hence, by definition, $tr\in
I_n$. This shows that $I_n$ is an ideal in $R$.

To see that $I_n$ is contained in $I_{n+1}$, let $r \in I_n$ witnessed by
$h(x) \in J$. If we multiply $h$ with $x$, the resulting polynomial will have 
degree $n+1$ but have same leading coefficient. Hence $r \in I_{n+1}$ by
definition. This shows that $\forall~n \ge 0~ I_n\subseteq I_{n+1}$. 

Since, $R$ is Noetherian, the above chain must be terminating at the ideal
$I_N$ for an integer $N$. Also since $I_n$ are ideals in $R$ and chains in $R$
always terminates, the equivalent characterization gives that each of the
$I_n$ are finitely generated. 
% TODO : give backref to thm.
For $1 \le i \le N$, let $I_i$ be $ \langle
r_{(i,1)},r_{(i,2)},\ldots,r_{(i,t_i)} \rangle$ where $t_i$ is an integer
and each of the $r$'s belongs to $I_i$. Let $f_{ij}$ in $J$ whose degree 
is $i$ such that $r_{ij}$ is the leading coefficient of $F_{ij}$ 
We now claim that $J$ is finitely generated.
\begin{claim}
	Denote $J^*$ to be the ideal $\langle \{ f_{ij} ~|~ 0\leq i\leq N,
	1\leq j \leq t_i \}\rangle $. Then $J^* = J$
\end{claim}
\begin{proof}
	Since $f_{ij} \in J$, we have $J^*\subseteq J$. We now show that 
	$J\subseteq J^*$. We prove this by induction on the degree of 
	the polynomial $f$ in $J$.
	
	For base case, $n$ = 0. Hence degree of the polynomial is zero
	implying that all polynomials are constants. Hence $J^* = I_0
	\subseteq J$. 
	
	Assume, by inductive hypothesis, that any polynomial of degree $n-1$ in
	$J$ also appears in $J^*$. Let $f$ be a degree $n$ polynomial in
	$J^*$ and $r$ be its leading coefficient. Hence $r \in I_n$. This
	gives
	\[ r=\sum_{j=1}^{t_n} s_j r_{(i,j)} \]
	where $s_j \in R$. 
	Consider, the polynomial $g = \sum_{j=1}^{t_n}s_j f_{(i,j)} \in J^*$. 
	Note that the leading coefficient of $g$ is exactly $r$ which is same
	as the leading coefficient of $f$. Hence the degree of the polynomial 
	$f-g$ is $(n-1)$. Since $f,g \in J$\footnote{$g \in J$ since $g \in
	J^*$ and $J^* \subseteq J$.}, $f-g \in J$. Applying 
	induction hypothesis, $f-g$ is in $J^*$. Since $g \in J^*$, we get
	that $f \in J$. This proves that $J \subseteq J^*$. 
\end{proof}
The completes the proof that $R[x]$ is Noetherian if $R$ is Noetherian.
\end{proof}

\Lecture{Jayalal Sarma}{Sep 22, 2015}{29}{Computing the generating sets of
Ideals}{Manoj Kumar Sure}{$\gamma$}{K Dinesh}
In the previous lecture, we proved the Hilbert's theorem which says that every
ideal in $\mathbb{F}[x_1,\ldots,x_n]$ is finitely generated. Recall that
our aim of this study is a way to solve system of polynomial equations in
multiple variables. The above theorem guarantees the existence of basis
generating the ideal. So a natural question is can be compute this basis given
an ideal.

In this lecture, we show how to compute the finite generating sets for two
special cases where the system of polynomials is
\begin{enumerate}
\item Linear equations
\item Univariate
\end{enumerate}
For Linear equations case, we can do Gaussian elimination and find out the
generating set for the given ideal. Even though the idea is very simple, we
will see in subsequent lectures that this procedure will be generalized to
multivariate case also.

\begin{example} Consider the system of linear equations $f_1=0,f_2=0$ where ,
$f_1 = x+y-z$ and $f_2=  2x+3y+2z$. Let $I=\langle 
f_1,f_2 \rangle$. We can rewrite the above ideal using Gaussian elimination as 
$f'_1 \leftarrow f_1$ and $f'_2 \leftarrow (1/2)f_2-f_1$. Also $I= \langle 
f'_1,f'_2 \rangle$ remains unchanged.
\end{example}

\begin{example}
For univariate case, consider the following example $f_1 = f_2 = 0$ where
$f_1 = x^3-2x^2+2x+8$, $f_2 = 2x^2+3x+1$ and $I=\langle f_1,f_2\rangle $. 
We can write $f_1 = qf_2+r$, where $q=x/2 - 7/4\;and \; r=(27/4)x+(39/4)$
\end{example}
Note that our aim in both the cases were to somehow eliminate variables by
performing suitable operations. This naturally gives the following division
procedure for univariate polynomials.

\begin{lemma}[Division]
	For any $f,g \in \mathbb{F}[x]$, $\exists\;q,r \in \F[x]$ such 
	that $$ f=q \cdot g+r $$ such that $q,r$ are unique and 
	$0 \le deg(r) < deg(g)$
\end{lemma}
The values of $q$ and $r$ can be found using Division algorithm which we
outline below. We leave the proof of uniqueness of $q,r$ as exercises.
Let $LT(f),LC(f)$ represents the leading term of $f$ and leading coefficient 
of $f$ respectively. 
\begin{algorithm}[htp!]
\caption{Division algorithm for Univariate polynomials}\label{euclid}
\begin{algorithmic}[1]
\Procedure{Divide}{ Input : Polynomials $f$ and $g$ }
\State $r \leftarrow f$
\State $q \leftarrow 0$
\While {$r\neq0$ and $deg(g) \leq deg(r)$}
\State $q = q + \frac{LT(r)}{LT(g)}$
\State $r=r-\frac{LT(r)}{LT(g)}\cdot g$
\EndWhile
\State \textbf{return} $q,r$
\EndProcedure
\end{algorithmic}
\end{algorithm}
The division algorithm gives us some nice properties of $\F[x]$. 

\begin{definition}(Principal Ideal)
An ideal of a ring is a principal ideal, if it is generated by a single element.
\end{definition}
\begin{definition}(Principal Ideal Domain)
Principal Ideal Domain (PID) is an integral domain in which every ideal is generated by a single element.
\end{definition}
We show that $\F[x]$ is a principal idea domain.
\begin{theorem} \label{thm:pid-univariate}
If $I$ is an ideal in $\mathbb{F}[x]$, then $I$ is generated by a single element. 
\end{theorem}
\begin{proof} 
	Let $g$ be a polynomial in $I$ of least degree. We need to prove
	$\langle g\rangle =I$. Suppose, it is not, that is, there exists $f \in
	I,\; f \notin \langle g\rangle$.  By our choice  of $g$,
	$deg(f)\geq deg(g)$. Applying division algorithm for $f$ and $g$, we
	get $f=q.g+r$ with $q$ and $r$ as returned by division algorithm 
	and $r=0$ or $deg(r)<deg(g)$.
	
	If $r=0$, then $f$ is divisible by $g$, so, $f$ will be generated by
	$g$. If $deg(r)<deg(g)$, then $r = f-q.g$. Using the facts that $ f\in
	I,\;q\in\mathbb{F}[x],\;g\in I$, we get that $r\in I$ and
	$deg(r)<deg(g)$, which contradicts our choice of $g$. So, $I$ must
	be generated by $g$. Hence, $\mathbb{F}[x]$ is a Principal Ideal Domain.
\end{proof}

The above theorem essentially says, 
for a given set of univariate polynomials $f_1(x)=0,f_2(x)=0,\ldots,f_k(x)=0$,
for the ideal $I=\langle f_1,f_2,\ldots,f_k\rangle $, there exists a $g\in I$, 
such that $I=\langle g\rangle $. Hence to find a solution for system, it 
is sufficient to solve $g(x)=0$. 

Now, we need a way to find this $g$ given $f_1,\ldots,f_k$. Let us consider
the case for $k=2$.
\begin{problem}
	Given $f_1, f_2 \in \F[x]$ and $I=\langle f_1,f_2\rangle $, 
	find $g \in \F[x]$ such that $I=\langle g\rangle $
\end{problem}
The required $g$ is nothing but the largest degree polynomial which divides
$f_1$ and $f_2$. We formally define greatest common division (GCD) formally.
\begin{definition}[GCD of two polynomials]
	For $f_1,f_2 \in \F[x]$, $gcd(f_1,f_2)$ is a polynomial $g \in \F[x]$ 
	such that 
\begin{enumerate}
\item $g \mid f_1, g \mid f_2$
\item $\forall h \in \F[x], h \mid f_1 $ and $h \mid f_2$ implies $h \mid g$
\item leading coefficient of $g$ = 1 
\end{enumerate}
\end{definition}
%The last requirement (to have unique $g$)
We call a polynomial with its leading coefficient as $1$ as \emph{monic
polynomial.} 
\begin{observation}
	GCD of two polynomials $f_1,f_2 \in \F[x]$ has the following property.
$gcd(f_1, f_2) = gcd(f_1-qf_2,f_2)\;\;\forall q \in \mathbb{F}[x]$
\end{observation}

\begin{exercise}
%{}{
\begin{enumerate}[(a)]
\item Show that there are ${n+d-1 \choose d}$ monomial of total degree $d$ in $n$ variables.
\item Let $f_1, f_2, \ldots, f_k \in \F[x]$ be univariate polynomials.\\
Prove that $GCD(f_1, \ldots, f_k) = GCD(f_1, GCD(f_2, \ldots, f_k))$.
\end{enumerate}
\end{exercise}





In the next lecture, we discuss an algorithm for computing GCD of two
univariate polynomials.
%\begin{lemma}
%$$I = \langle f_1, f_2\rangle , f_1 \neq 0\;and\;f_2\neq 0$$
%then,$$I=\langle gcd(f_1,f_2)\rangle $$
%\end{lemma}
