\Lecture{Jayalal Sarma}{August 19, 2015}{11}{Pointwise Stabilizer, Membership
Testing and Group Intersection}{Samir Otiv}{$\gamma$}

\section{Recap}
We first attacked the Membership Problem. We saw that this reduces to the
problem of Order Computation. (See section~\ref{sec:comp-questions}) 

To solve this problem of Order Computation, we use the
Orbit-Stabiliser Lemma. This involved finding out the size orbit of an element
$\alpha$ (which we reduced to Graph Reachability), and then recursively
finding the size of $G_\alpha$ which is the stabiliser group of $\alpha$.
Product of these two gives the order of the group.
%We use the tower of subgroups to do this.

Note that in the above approach, $G$ is specified via a generating set. Hence
before we recurse, we need to obtain a generating set for
$G_\alpha$. For this purpose, we used Schreier's lemma to find a generating
set. We also observed that size of the generating set can grow very fast. Using
the reduction algorithm that Aditi very kindly explained in the previous
lecture, we could ensure that the size of the generating set always stayed 
within $O(n^2)$ (where $n = |\Omega|$).

\section{Finding Coset Representatives in the tower of subgroups}
\label{sec:find-coset}
Now that we have obtained a way of obtaining a generating set for the
subgroup, we are left with the problem of finding the orbit size. 
By Orbit-Stabiliser Lemma (Lemma~\ref{lem:os}), due to one-one correspondence
with cosets, we get that to estimate orbit size, it suffices to estimate the
number of cosets of the subgroup. In our setting this can be achieved by
finding the coset representatives in the tower of subgroups.

Denote $R$ as the set of right coset representatives of $G^{(i+1)}$  in $G^i$
for $i \ge 0$. Note that there are at most  $n-i$ cosets, since $i+1^{th}$
location can be taken to at most $n-i$ different places. Recall that the set of
locations where $i+1$ is taken to by $G^{(i)}$ is precisely $i+1$.
\[ (i+1)^{G^{(i)}}=\{ k \in \{i+1, \ldots, n\} | \exists g \in G^{(i)},
(i+1)^g = k\} \]

Let $X_i \subseteq \{i+1, \ldots, n\}$ be the orbit of $i+1$ in $G^{(i)}$ (We
have seen an algorithm to do it in Section~\ref{sec:orbit-comp}). 
For each $k \in X_i$, let $g_{i+1, k}$ be an element in $G^{(i)}$ that maps
$1$ to $k$. Note that such an element must exists in $G^{(i)}$. To find this
group element, it suffices to obtain a path from $i+1$ to $k$ in the graph of
generating set (see Section~\ref{sec:orbit-comp} for definition) and take the
product of edge labels on the path. The set $R$ is now the collection of all
$g_{i+1,k}$ for each $k \in X_i$.

\section{Algorithm for Pointwise Stabiliser Problem}
We now piece together the above ideas to show that for any
$\Sigma\subseteq\Omega$, the Generator set of $\pointstab(\Sigma) \le S_n$ can
be computed in $poly(|\Omega|)$ time. Let $i \ge 1$. Recall that
$\pointstab(\{1,2,\ldots,i\}) = G^{(i)}$. 

To compute a generating set for $G^{(i)}$
\begin{enumerate}
	\item Use the technique specified above to find coset representative for
		$G^{(1)}$ in $G$. We consider $G^{(0)} = S_n$ and use the
		generator set for $S_n$ as $\{ (1)(2)\ldots(n), (1~2), \ldots,
		(n-1~n)\}$ while constructing the graph for generating set.
	\item  Previous step gives us the set $R$ which consists of the right
		coset representatives of $G^{(1)}$ in $G^{(0)}$. Now we can
		apply Schreier's Lemma to obtain the generating set for
		$G^{(1)}$. 
	\item  Apply the {\sc Reduce} algorithm (Algorithm~\ref{alg:reduce})
		to obtain a smaller generator set.
	\item  Recurse by following the same procedure, and finding the
		generating set of $G^{(2)}$ given $G^{(1)}$, until the desired
		$G^{(i)}$ is found. 
\end{enumerate}
It can be seen the each step terminates in $\poly(|\Omega|)$ time and the
number of recursive calls is at most $i \le |\Omega|$. Hence overall runtime is
polynomial.


\section{Algorithm for Membership Testing}
Note that we already saw how to check membership of an element in a group
using an algorithm to compute order of the group. But this does not tell
anything about how the element is expressed in terms of its generators. Hence
we consider the following stronger version of Membership problem. 
\begin{problem}[Stronger Membership Test]
Given a generating set $A$ of $G$ with, $G \leq S_n$ and a $g\in S_n$, give a
representation in terms of any desired generating set, and the generating set 
itself.
\end{problem} 

\begin{algorithm}[htp!]
\caption{\textsc{MemberTest} : Algorithm for Membership Testing}
\label{alg:memtest}
\begin{algorithmic}[1]
	\Procedure{\textsc{MemberTest}}{Input : Element $g$, Generating set
$A$ for $G^{(i)}$, Index $i$}

\If{$g = id$} 
\State{return true}
\EndIf
\State{$X_i = (i+1)^{G^{(i)}}$ (Use Orbit Computation Algorithm~\ref{alg:orbit_alg})}
\State{Compute $R$, the set of coset representatives of $G^{(i+1)}$ 
in $G^{(i)}$ (see Section~\ref{sec:find-coset}) }
\State{$k = (i+1)^g$ (Image of $i+1$ on action of $g$)}
\If{$k \notin X_i$}
\State{return false}
\Else
\State{$B$ = Generating set of $G^{(i+1)}$ (Use Schreier's Lemma)}
\State{Apply Algorithm {\sc Reduce} to $B$ and denote
$A'$ for the reduced set obtained}
\State{Pick $g_{ik}$ from $R$ that maps $i$ to $k$}
\State{$g' = g.g_{ik}^{-1}$}
\State{return \textsc{MemberTest}$(g', A', i+1)$}
\EndIf
\EndProcedure
\end{algorithmic}
\end{algorithm}

Note that if insisting on giving the representation of $g$ in terms of $A$,
then the problem may not be solvable in polynomial time since there are groups
which require large size to represent in terms of its generators
(Proposition~\ref{prop:repr-large}). Now we call \textsc{MemberTest} 
on the input $(g,A,0)$ for checking if $g \in \langle A \rangle$ or not. It is
left as an exercise to show that the algorithm terminates in $poly(|\Omega|)$
time.

\section{Group Intersection Problem}
We consider the following two problem :
\begin{problem}[Subgroup Problem]
	Given groups $G$ and $H$ via their generating sets $A$ and $B$ 
	respectively, test if $H \leq G$
\end{problem}
\begin{problem}[Group Intersection Problem]
	Given groups $G$ and $H$ via their generating sets $A$ and $B$
	respectively, find the generating set of $G\cap H$\footnote{Verify that if $G$
	and $H$ are groups then $G \cap H$ is also a group}.
\end{problem}

Since we have already given an algorithm for membership checking, Subgroup
problem has the following algorithm : for each $b\in B$ check if $b \in G$ 
using {\sc MemberTest} algorithm.

So how about Group Intersection problem ? We show that this problem is as hard
as the Set Stabilizer Problem. We show this by first reducing $\setstab$ to
$\groupintr$.
\begin{claim}
$\setstab \le \groupintr$
\end{claim}
\begin{proof}
	Given $\Sigma \subseteq \Omega$, generating set $K$ of $G$ we need to
	produce groups $J,H$ via generators $A,B$ such that $J \cap H =
	\setstab(\Sigma)$. Consider the group, $H = Sym(\Sigma) \times
	Sym(\Omega \setminus \Sigma)$ where $Sym(T)$  
	consists of all permutations defined over $T$ of length $|T|$. Hence
	$Sym(T)$ fixes $T$. Note that $H \le S_n$ and consists of all
	permutations that have $\Sigma$ mapped to $\Sigma$. Since
	$\setstab(\Sigma)$ talks about permutations in $G$ that fix $\Sigma$,
	$G \cap H$ is the $\setstab(\Sigma)$. Now it remains to give a
	generator for $Sym(\Sigma)$. It can be shown that if $\Sigma
	=\{1,2,\ldots,n\}$, then the generator set for $Sym(\Sigma)$ is
	$(1,\ldots, n), (1~2),(2~3),
	\ldots, (n-1~ n)$.
\end{proof}


\Lecture{Jayalal Sarma}{August 21, 2015}{12}{Group Intersection to Set Stabiliser and Jerrum's Filter}{Samir Otiv}{$\beta$}

\section{GroupInter to SetStab}
Before we start, let's look at ways of combining groups.

\subsection{Direct Product}
$G*H=\{(g,h | g \in G,hH\}$
$(\alpha,\beta)^{(g,h )}=(\alpha^g,\beta^h)$

\subsection{The reduction is almost done}
Let $\Sigma=\{ (i,i)| i \in O\}$

\subsection{Claim}
$(G \cap H)_{duplicate} = SetStab(\Sigma)$\newline
where $(G \cap H)_{duplicate} = \{ (a,a) | a\in G \cap H \}$

\subsection{Proof}
Forward:\newline
	$a \in G \cap H$\newline
	$(a,a) \in S_n*S_n$\newline
	(i,i) $\in \Sigma, (i,i)^{(a,a) } \in \Sigma$\newline
	$\therefore (a,a) \in SetStab(\Sigma)$\newline\newline
Backward:\newline
	$(a,b) \in SetStab(\Sigma)$\newline
	$\therefore \forall (i,i) \in \Sigma:(i,i)^{(a,b)}=(j,j)$\newline
	$j=i^a$\newline
	$j=i^b$\newline
	$\therefore a=b \in$ G,H respectively.\newline
	$\therefore a \in G \cap H$\newline


\section{Jerrum's Filter: Reduce the size of the generating set to n-1}
So far we know a way of constructing it in $n^2$. Jerrum's Filter can do it in n-1 instead.
Another result: Neomann $ \lfloor n/2 \rfloor $, which we're not going to discuss further in this lecture.

\subsubsection{Jerrum's Filter}
S is a given set of permutations\newline
$G = <S>$\newline
G acts on $\Omega$

\subsubsection{Define a graph}
$X_S (V,E)$, where $V=\Omega$\newline
$\forall g \in S$, if $i_g \in \Omega$  is the smallest index moved by g, put an edge $(i_g,i_g^g ) \in E$\newline
View this as an undirected graph.

\subsubsection{Observation}
$g^{-1}$  maps $i_g^g$  to $i_g$

\subsubsection{Measure of weight}
$T \subseteq S_n$
$wt(X_T )=\Sigma _{g \in T}¦i_g$

\subsubsection{Crude bound on $wt(X_T )$}
$wt(X_T )\leq \|T\|n$

\subsubsection{The algorithm}
Maintain a set A such that $X_A$  is acyclic. This is the invariant of the algorithm. (If a new element g comes in, we modify the graph to maintain the acyclicness)\newline
Process each $g \in S$ one by one.\newline\newline
For a given g:

	If $g \notin  <A>$, then add g to A and add $e_g$  to the graph $X_A$.

	If there's no cycle created, there's nothing to be done. If there is, consider the cycle created. Let $i_0$  be the least index in the cycle.

	Now, both the edges in the cycle incident to $i_0$ must correspond to $i_0$. (If they didn't then the neighbour of $i_0$ wouldn't be the least moved element by the corresponding group element).
	
	Now consider the walk around the circle from $i_0$ to $i_0$. This corresponds to a product of group elements of the form $g_0 g_1^{\epsilon_1} . . g_k^{\epsilon_k}$. Now, just get rid of $g_0$, and insert h in its place. This will not change the group generated by S, since reachability is preserved.
	
	Now since $i_0$ is fixed by all of $g_0, g_1, . . g_k$, therefore the least element moved by h is $>$ i. Therefore, the weight of the graph increases on doing this operation.
	
	Therefore, in polynomially many steps, we will hit the upper bound, and will end up with an acyclic graph.
	
	On repeating this with all elements, we have a generating set A' with an acyclic graph $X_{A'}$, and therefore $|A|\leq n-1$.






