\documentclass[12pt]{report}
\usepackage{amsfonts,amsthm,amsmath}
\usepackage[margin=1.5in]{geometry}
\usepackage{complexity}
\usepackage{graphicx}
\usepackage{tocloft}
\usepackage{xcolor}
\usepackage[all]{xy}
\usepackage{scribe-book}
\usepackage{amssymb}
\usepackage{amsmath}
\usepackage{hyperref}
\input{preamble.tex}
\renewcommand{\E}{{\mathbb E}}

\author{}
%\newcommand{\N}{\mathbb{N}}
% Shows lecture titles and sections. Set 0 to list only chapters.
\setcounter{tocdepth}{2}

	
\begin{document}
\thispagestyle{empty}

\begin{flushright}

\vspace*{3cm}
\makeatletter 
\vskip 10\p@ 
\hrule height 2pt
\vskip 2\p@
\hrule height 1pt

{\huge Lecture Notes on}\\[3ex]
{\Huge \textbf{Algorithmic Algebra}} \\[2ex]

\end{flushright}

\makeatletter 
\vskip 10\p@ 
\hrule height 2pt
\vskip 2\p@
\hrule height 1pt
\vskip 60\p@
\makeatother


\begin{flushleft}
{\large Jayalal Sarma M.~N.\\ 
Department of Computer Science and Engineering\\
IIT Madras, Chennai 600036 \\[2ex]
Draft---\today\ and forever
}
\end{flushleft}

\newpage
\setcounter{page}{1}
\pagenumbering{roman}  % Roman numbering in intro portion.
\include{front-page}

\newpage
\listofscribe          % For automatic scribe list generation.

\newpage
\listofinstr

\newpage
\tableofcontents

\pagenumbering{arabic}  % Arabic page numbering for lectures.
%\tableofcontents
\newpage \chapter*{Preface}

This lecture notes are produced as a part of the course \textit{CS6842: Algorithmic Algebra} which was a course offered during August to November semester in 2013 and 2015 at the CSE Department of IIT Madras.

\section*{Acknowledgements}
Thanks to Alexander Shrestov and Markus Blaser for creating nice templates for lecture notes which are being combined and in this document.

\section*{Scribe status}
Each lecture has a field called {\bf status}. It tells which stage of the edit
pipeline is the document currently. Each scribe will be set to choose to
either Monday-Tuesday (MT) or Wednesday-Friday (WF) pair of lectures.
There are four versions.
\begin{enumerate}
	\item Alpha (deadline $X+2$): Rough draft. The MT Scribe is expected
		to turn-in the rough notes by X+2 (Thursday night 10pm) and WF
		scribe is expected to turn this in by Sunday night 10pm.
	\item Beta (deadline $X+4$) : Final draft. The MT scribe is expected
		to turn this in by Saturday night 10pm and WF scribe is
		expected to turn this in by Tuesday night 10pm.
	\item Gamma (deadline $X+5$) : Corrected draft. This is done by the
		TAs and given to the instructor.
	\item Delta (deadline $X+6$) : Final notes. The instructor puts out 
		this final version.
\end{enumerate}

%\part{Introduction, Motivation and the Language}
\newpage \setcounter{page}{1} 
\Lecture{Jayalal Sarma M.N.}{July 31, 2013}{1}{Introduction, Motivation and the Language}

\noindent
General Comments.

\section{Overview of the course. Administrative, Academic policies}
\newpage
\section{The Basis Objects - Polynomials.}
\newpage
\section{Three Representative Applications}
\subsection{From Graph Algorithms : Perfect Matching}
\newpage \newpage \newpage
\subsection{From Number Theory : Primality Checking}
\newpage \newpage \newpage
\subsection{From Robotics : Robot motion planning.}
\newpage \newpage \newpage
\section{Overview of the course}
\newpage \newpage



\newpage \setcounter{section}{0}
\Lecture{Jayalal Sarma}{Aug 4, 2015}{2}{Algebraic Approach to
Primality Testing}{K Dinesh}
In this lecture, we will see an algebraic approach to solving a fundamental
problem in Number Theory.
\section{Application to Number Theory}
Following is an algorithmic question that we are interested.
\begin{problem}
Given a number $n$ in its binary representation, check if it is a prime or not
in time $O(poly(\log N))$.
\end{problem}
\begin{note}
The trivial algorithms that we can think of will depend on $n$ and hence takes
time exponential in its input representation.
\end{note}

Consider the following property about prime number proved by Fermat which is
of interest in this context.
\begin{theorem}[Fermat's Little Theorem] If $N$ is a prime, then $\forall~a, 1
	\le a \le N-1$, \[ a^N = a \mod N \]
\end{theorem}
\begin{proof}
	Fix an $a \in \{1,2,\ldots N-1\}$. Now consider the sequence $a, 2a,
	\ldots, (N-1)a$. The question we ask is : can any two of the numbers
	in this sequence be the same modulo $N$. We claim that this cannot
	happen. We give a proof by contradiction : suppose that there are two
	distinct $r,s$ with $1 \le r < s \le N-1$ and $sa = ra \mod N$. Then
	clearly $N | (s-r)a$ which means $N | a$ or $N | (s-r)$. But both
	cannot happen as $a, s-r$ are strictly smaller than $N$.

	This gives that all the $N$ numbers in our list modulo $N$ are
	distinct. Hence all numbers from $1,2,\ldots, N-1$ appear in the list
	when we go modulo $N$. Taking product of the list and the list modulo
	$N$, we get 
	\[ (N-1)! a^{N-1} = (N-1)! \mod N \]
	By cancelling $(N-1)!$, we get that $a^{N-1} = 1 \mod N$.
\end{proof}
This tells that the above condition is necessary for a number to be prime. If
this test is also sufficient (i.e, if the converse of the above theorem is
true), we have a test for checking primality of a number. But it turns out
that this is not true due to the existance of Carmichael numbers which are not
prime numbers but satisfy the above test.

So one want a necessary and sufficient condition which can be used for
primality testing. For this, we need the notion of polynomials.
\begin{definition}
	A polynomial $p(x) = \sum_{i=0}^d a_ix^i$ with $a_d \ne 0$ denotes a
	polynomial in one variable $x$ of degree $d$. Here $a_i$s are called
	coefficients and each term excluding the coefficient is called a
	monomial. \textbf{A polynomial is said to be identically zero, if all the
	coefficients are zero.}
\end{definition}

A polynomial time algorithm for this problem has been found in 2002 by
Manindra Agarwal, Neeraj Kayal and Nitin Saxena (called as the AKS algorithm). 
In their result, they used the following polynomial characterisation for a
prime number.
\begin{theorem}[Polynomial formulation (Agarwal-Biswas 1999)] Let $N \ge 1$ be
	an integer. Define a polynomial 
	$p_N(z) = (1+z)^N - 1 - z^N$. Then \[ p_N(z) \equiv 0 (mod~ N)
	\iff N \text{ is prime} \]
\end{theorem}

Hence checking if $N$ is prime or not boils down to checking if $p_N(z)$ is
identically $0$ or not except for the fact that the underlying operations are
done modulo $N$.

Proof of the theorem is as follows.
\begin{proof}
	Note that $p_N(z) = \sum_{i=1}^{N-1} \binom{N}{i} z^i$ and
	$\binom{N}{i} = \frac{N(N-1)\ldots (N-i+1)}{1\cdot 2 \ldots i}$ with $2
	\le i \le N-1$. 

	If $N$ is prime, then $\binom{N}{i} = N \times k_i$ for some integer
	$k_i$ as none of $1, 2, \ldots i$ divides $N$. Hence $\binom{N}{i}
	\mod N = 0$ for every $i$ and $p_N(z) \equiv 0 \mod N$ since the
	polynomial has all coefficients as zero.

	If $N$ is composite, we need to show that $p_N(z) \not 
	\equiv 0 (mod~ N)$ which means that there is at least one non-zero
	coefficient $p_N(z) \mod N$. Suppose $N$ is
	composite. Hence there exists a prime $p$ such that $p ~|~
	N$.  Let $p^k | N$ where $k \ge 1$ is the largest exponent of $p$ in
	the prime factorisation of $n$. Hence $p^{k+1} \nmid N$.
	
	We first show that $p_N(z)$ is a non zero polynomial by showing that
	the coefficient of $z^i$ for $i = p$, which is $\binom{N}{p}$, is non
	zero modulo $p^k$ showing\footnote{Note that if
		$N|\binom{N}{i}$ then $p^k | \binom{N}{i}$. Taking
		contrapositive, we get that $\binom{N}{i} \not = 0 \mod p^k$
		implies $\binom{N}{i} \not = 0 \mod N$} $p_N(z)$ is not a zero
		polynomial modulo $N$. 
	Let $N = g \times p^k$. In $\binom{N}{p}$, $p$ of the denominator 
	divides $N$. Note that $p$ cannot divide $g$, for if it does, then
	$p^{k+1} | N$ which is not possible by choice of $p$ and $k$.
		\begin{equation}
		\binom{N}{p} = \frac{g\times p^k(N-1)\ldots (N-p+1)}
		{1\cdot 2 \ldots p} = \frac{g\times p^{k-1}(N-1)\ldots 
			\label{eq:ag-bis-proof}
		(N-p+1)}{1\cdot 2 \ldots p-1} \end{equation}
	Hence it must be that $p^{k-1}$ divides $\binom{N}{p}$. This is
	because, there is no $p$ left now in the denominator to divide $N$.
	Now $p^k \nmid \binom{N}{p}$ because, none of the terms $(N-1),
	\ldots,
	(N-p+1)$ can have $p$ as a factor since all these terms are obtained by
	subtracting at most $p-1$ times from $N$. Hence $\binom{N}{p} \ne 0
	\mod p^k$. This completes the proof.
\end{proof}

Note that Fermat's Little Theorem is a special case of Agarwal-Biswas
formulation of primality testing.
\begin{exercise}
Derive Fermat's Little Theorem from Agarwal-Biswas theorem. (Hint : Use
induction on $a$)
\end{exercise}

This shows that checking if the polynomial $p_N(z)$ is a zero polynomial is an
if and only if check for primality of $N$. There are two naive ways to do it.
One is to evaluate it at all the numbers from $1$ to $N$ or to expand out the
polynomial and see if it is identically zero. But both operations are costly.

So checking primality of a number now boils down to checking if a polynomial
is identically zero or not. This is a fundamental problem of polynomial
identity testing. We will be discussing about polynomial identity testing and 
AKS algorithm in our next theme.

%\part{Algorithmic Ideas Related to Polynomials}
\newpage \setcounter{section}{0}
\Lecture{Jayalal Sarma M.N.}{Aug 21, 2013}{3}{Algebraic Approach to
finding Perfect Matchings in Graphs}{Student}
\noindent
Preamble

\section{Application to Graph algorithms}
Consider the following problem of finding perfect matching.
\begin{definition}[Finding Perfect Matching]
Given a bipartite graph $G(V_1,V_2, E)$, we need to come up with an $E'
\subseteq E$ such that $\forall u \in V_1 \cup V_2$, there is exactly one edge
incident to it in $E'$.
\end{definition}

We shall give a polynomial formulation for the problem. Given $G(V_1, V_2, E)$
with vertex sets $|V_1| = |V_2| = n$. Define an $n\times n$ matrix $A$ where
$A(i,j) = 1$ if $(i,j) \in E$ and is $0$ otherwise for all $(i,j) \in V_1 \times
V_2$. Recall the determinant of $A$ given
by
\[ det(A) = \sum_{\sigma \in S_n} sign(\sigma) \prod_{i=1}^n A_{i,\sigma(i)}
\]

where \[
sign(\sigma) = \begin{cases} 
	  -1  & \text{ if } inv(\sigma) \text{ is even} \\
	  1 & \text{ if } inv(\sigma) \text{ is odd}
\end{cases}\]
and $inv(\sigma)$ is defined as 
$\left |\{(i,j) ~|~ i < j \text{ and } \sigma(i) > \sigma(j), 1 \le i<j \le n\}
\right |$\footnote{Every permutation can be decomposed to disjoint cycles. It
can also be shown that $sign(\sigma) = (-1)^{(n - \text{\# of odd cycles})}$}.

We denote $f(x) \equiv 0$ to denote that polynomial $f(x)$ is the zero
polynomial.
\begin{lemma}
For the matrix $A$ as defined as before, $det(A) \not \equiv 0 \implies
\text{$G$ has a perfect matching}$ \label{lem:det_pm_simple}
\end{lemma}
\begin{proof}
Let $det(A) \not \equiv 0$. Hence there exists a $\sigma \in
S_n$ such that $\prod_{i=1}^n A_{i, \sigma(i)} \ne 0$. Hence the edge set
$E' = \{(i, \sigma(i)) ~|~ 1\le i\le n\}$ exists in $G$ and since $\sigma(i) =
\sigma(j)$ iff $i=j$ for every $i,j$, $E'$ form a perfect matching.

\end{proof}
Note that converse of this statement is not true. For example, consider the
bipartite graph whose $A$ matrix is $A = 
\begin{bmatrix}
		1 & 1 & 0 & 0 \\
		0 & 1 & 1 & 0 \\
		0 & 0 & 1 & 1 \\
	        1 & 0 & 0 & 1
\end{bmatrix}
$. It can be verified that $det(A) = 0$ but the bipartite graph associated has
a perfect matching.

Hence the next natural question to ask is what kind of modification to the
converse of previous lemma (lemma~\ref{lem:det_pm_simple}) is true. In the
example considered, there were two perfect matchings in $G$ having opposite 
sign due to which the determinant became $0$. So the modification should
ensure that perfect matchings of opposite signs does not cancel off in the
determinant.

One way to achieve this is as follows.  Define a matrix $T$ as
\[ T(i,j) = \begin{cases}
		x_{ij} & (i, j) \in E \\
		      0 & \text{ otherwise}
	\end{cases}
\]
where $(i,j) \in V_1 \times V_2$.
Now, if we consider determinant of this matrix, we can see that the monomials
corresponding a $\sigma \in S_n$ can be a product of $n$ variables at most.
Hence $det(T)$ is a polynomial in $n^2$ variables with degree at most $n$.
We now give polynomial formulation for the problem of checking perfect
matching in a bipartite graph.
\begin{claim}
For the matrix $T$ as defined before, $det(T) \not \equiv 0 \iff \text{$G$
has a perfect matching}$
\end{claim}
\begin{proof}
($\Rightarrow$) Since $det(T) \not \equiv 0$, there exists a $\sigma \in S_n$
such that the monomial corresponding is non-zero and by definition of
deteminant it must be expressed as product of some $n$ set of variables
$\prod_i T(i,\sigma(i))$. The variable indices gives a matching and since
there are $n$ variables this is a perfect matching.

($\Leftarrow$) Suppose $G$ has a perfect matching given by a $\sigma$ and it
appears in $det(T)$ as $m$. Only way the term associated with this perfect
matching can vanishes from $det(T)$ is via another perfect matching which
appear as $-m$ in $det(T)$

But note that for any two different perfect matchings in $G$ if we look at the
permutations associated, they must differ in at least one edge. Hence the
associated terms must differ in some variable. Hence there does not exists
another matching which can vanish $m$ from $det(T)$ showing $det(T) \not \equiv
0$.
\end{proof}

Hence to check if the bipartite graph $G$ has a perfect matching or not, it
suffices to check if the polynomial $det(A)$ is identically zero or not.
Checking if a polynomial is identically zero or not is one of the fundamental
question in this area.

Note that this problem becomes easy if the polynomial is given as sum of 
monomial form. In most of the cases, the polynomial will not be given this way.
For example if we consider our problem, we are just given the $T$ matrix and
$det(T)$ is the required polynomial. Trying to expand $det(T)$ and simplifying
will involve dealing with $n!$ monomials which is not feasible.

So far we have spoken about finding perfect matching in bipartite graphs. A
similar result holds for general graphs also. 
\begin{theorem}
Given a graph $G(V,E)$ and for $i, j \in V$, $i \ne j$, define an $n \times n$
matrix $T$ as
\[T_{i,j} = \begin{cases}
		x_{ij} & (i,j) \in E, i < j\\
	       -x_{ji} & (i,j) \in E, i > j\\
   	        0 & \text{otherwise}
	    \end{cases}
    \]
	    Then, $det(T) \not \equiv 0 \iff \text{G has a perfect matching}$.
\end{theorem}



\newpage \setcounter{section}{0}
\Lecture{Jayalal Sarma M.N.}{Aug 7, 2015}{4}{Graphs, Groups and Generators}{Ramya C}

In this lecture we will pose three graph theoretic questions and find answers using approaches in algebra.

\section{Graph Isomorphism, Automorphism and Rigidity}

\begin{definition}(Graph Isomorphism.)
Let $G_1=(V_1,E_1)$ and $G_2=(V_2,E_2)$ be graphs. We say $G_1\stackrel{\sim}{=}G_2$ (read as $G_1$ is \em isomorphic to $G_2$) if there exists a bijection $\sigma : V_1\rightarrow V_2$ such that $\forall (u,v) \in V_1\times V_2$ we have

\begin{center}
$(u,v)\in E_1 \iff (\sigma(u),\sigma(v))\in E_2$
\end{center}
\end{definition}

In other words, we say a graph $G_1$ is isomorphic to $G_2$ if there exists a relabeling of the vertices in $G_1$  such that the the adjacency and non-adjacency relationships in $G_2$ is preserved. 
\begin{observation}
If $|V_1|\neq |V_2|$ ,then $G_1$ is not isomorphic to $G_2$.

\end{observation}

The graph isomorphism problem is stated as follows. 
\begin{center}
\fbox{
\begin{minipage}{10 cm}\textbf{PROBLEM : GRAPH ISOMORPHISM}\\
\textbf{Input} : $G_1=(V_1,E_1), G_2=(V_2,E_2)$\\
\textbf{Output} : Decide if $G_1\stackrel{\sim}{=}G_2$ or not.
\end{minipage}
}
\end{center}

A natural question to ask in this setting is that if there is an isomorphism from a graph $G$ to itself. 

Let $[n]=\{1,2,\ldots,n\}$. Let $S_n$ denote the set of all permuatations from the set $[n]$ to $[n]$.

\begin{definition}(Graph Automorphism.) 
Let $G=(V,E)$ be a graph. An automorphism of $G$ is a bijection $\sigma:V\rightarrow V$ such that $\sigma(G)= G$. Let
\begin{center}
$Aut(G)= \{\sigma | \sigma\in S_n \text{~and~} \sigma(G)= G \}$
\end{center}
be the set of all automorphisms of $G$. 
\end{definition}



The graph automorphism problem is stated as follows. 
\begin{center}
\fbox{
\begin{minipage}{10 cm}\textbf{PROBLEM : GRAPH AUTOMORPHISM}\\
\textbf{Input} : A graph $G=(V,E)$\\
\textbf{Output} : Construct $Aut(G)$.
\end{minipage}
}
\end{center}



\begin{observation}
Let $\tau:[n]\rightarrow[n]$ be the identity permuataion. That is, for all $i\in[n],\tau(i)=i$. Then by definition $\tau\in Aut(G)$ for any graph $G=(V,E)$. This identity permuataion $\tau$ is in $Aut(G)$. 
\end{observation}


Are there other permutations from $[n]$ to $[n]$ that are in the set $Aut(G)$ ? How large can the set $Aut(G)$ be ?

Formally, the graph rigidity problem is stated as follows.
\begin{center}
\fbox{
\begin{minipage}{10 cm}\textbf{PROBLEM : GRAPH RIGIDITY}\\
\textbf{Input} : A graph $G=(V,E)$\\
\textbf{Output} : Decide if $Aut(G)$ is trivial or not.
\end{minipage}
}
\end{center}

\begin{observation}
Let $G$ be the complete graph on $n$ vertices. Then $|Aut(G)|=n!$.  
\end{observation}

Is the set $Aut(G)$ just a set or does it have more algebraic structure ? 


\section{Groups and Generators}

\begin{definition}(Groups.)
A set $G$ together with a binary operation $*$ is said to be a group if the following four consitions are met
\begin{itemize}
\item \textbf{Closure} : $a,b\in G$, the element $a*b\in G$.
\item \textbf{Associative} : For any $a,b,c\in G$, we have $(a*b)*c = a*(b*c)$.
<<<<<<< HEAD
\item \textbf{Existence of Identity} : For any $a\in G$ there exists a unique element $e\in G$ such that $a*e =e*a =a$. 
\item \textbf{Existence of Inverse} : For any $a\in G$ there exists a unique element $b\in G$(denoted by $a^{-1}$) such that $a*b =b*a =e$. 

\end{itemize}
\end{definition}

\begin{example}
\begin{itemize}
\item $S_n$ forms a group under composition.
\item $(\mathbb{Z}_5,+)$ is a group.
\end{itemize}
\end{example}

From now on, we will use the letter $X$ to denote a graph and $G$ to denote a group.

\begin{remark} 
Let $(G,*)$ be a group. Let $H\subseteq G$ such that $(H,*)$ also forms a group. We say $H$ is a {\em subgroup} of $G$ and denote by $H\leq G$.   
\end{remark}


\begin{exercise}
For any graph X, the set $Aut(X)$ forms a group under the composition operation. That is, $Aut(G)\leq M$ where $M=(S_n,\circ)$ is the permutation group.
\end{exercise}

Let $(G,+)$ be a finite group and $g\in G$ be an element. Let $g^2 =  g*g ,g^3 =  g*g*g $. Similarly $g^k = \underbrace{g*g*\cdots * g}_\text{k times}$. Now consider the set $H=  \{g,g^2,g^3,\ldots\}$. Since $(G,*)$ is a finite group there must exist a $k$ such that $g^k=g$ in $H$.

\begin{lemma}
Let $(G,+)$ be a finite group and $g\in G$ be an element. Let $H=\{g,g^2,g^3,\ldots\}$ be a set of elements. The unique identity $e$ of $G$ is in $H$.
\end{lemma}
\begin{proof}
Since $(G,*)$ is a finite group there must exist a $k$ such that $g^k=g$ in $H$. By definition, $g^k = g^{k-1}*g = g$. As $(G,*)$ is a group, $g^{-1}$ exists in $G$.\\
Therefore,
\begin{center}
$g^{k-1}*g*g^{-1} = g*g^{-1} = e$ 
\end{center}
\end{proof}

\begin{definition}[Generator]
Let $(G,+)$ be a finite group and $g\in G$ be an element. We say an element $g\in G$ is a generator of the set $H$ if for every element $h\in H$ there exists a $m$ such that $h=g^m$.  (denoted by $H=<g>$).
\end{definition}

\begin{observation}$H = <g>$ is a subgroup of $G$. That is, $H\leq G$.
\end{observation}

\begin{example}
\begin{itemize}
\item $<1> = (\mathbb{Z}_5,+)$
\end{itemize}
\end{example}



\begin{definition}An group $(G,*)$ that can be generated by a single element is called a {\em cyclic group}. For instance, $(\mathbb{Z}_5,+)$.
\end{definition}

Not every group is cyclic. For instance $S_3$ is not cyclic.

<<<<<<< HEAD
\begin{definition}(Generating set.)
Let $S\subseteq G$ be the set $\{u_1,\ldots,u_k\}$. $S$ is said to be {\em generating} if $<S>=G$.  
\end{definition}

Having observed that $Aut(X)$ could have potentially be of exponenetial size, it is natural to look for generating sets of small size.

Given a graph $X=(V,E)$ where $|V|=n$, does $Aut(X)$ have a generating set of size $\poly(n)$?

\subsection{Lagrange's theorem}

Let $(G,*)$ be a group. Let $H\leq G$. For any $g\in G$, define the right coset of $H$ in $G$ to be
\begin{center}
$Hg= \{hg \mid h\in H \}$
\end{center}
 

Let $g_1,g_2\in G$ and $Hg_1,Hg_2$ be the corresponding right cosets.  Are there elements that belong to more than one coset of $H$ in $G$? Is $Hg_1\cap Hg_2 \neq \phi$ ? If yes, then the set of right cosets of $H$ in $G$ form a partition of the ground set of $G$. In that case, how many such cosets are required to cover the entire set $G$ ? Let us answer these two questions.

\begin{lemma}
\label{lag1}
Let $g_1,g_2\in G$ and $Hg_1 = \{hg_1\mid h\in H\},Hg_2 = \{hg_2\mid h\in H\}$. Then 
\begin{center}
$Hg_1=Hg_2$ or $Hg_1\cap Hg_2 = \phi$.
\end{center}
\end{lemma}
\begin{proof}
If $g_1=g_2$, then by definition $Hg_1=Hg_2$. Therefore let $g_1\neq g_2$. We will prove : If $Hg_1\cap Hg_2 \neq \phi$ then $Hg_1 = Hg_2$. Let $Hg_1\cap Hg_2 \neq \phi, g\in Hg_1\cap Hg_2$ we will show 
\begin{itemize}
\item[(i)] $Hg_1\subseteq Hg_2$ ; and
\item[(ii)] $Hg_2\subseteq Hg_1$.
\end{itemize}

Since $g\in Hg_1$ we know that there exists a $h_1\in H$ such that $g=h_1g_1$. Similarly $g\in Hg_2$ suggests that there exists a $h_2\in H$ such that $g=h_2g_2$.
\begin{center}
$h_1g_1=h_2g_2=g$
\end{center}   
As $(H,*)$ is a group by itself, $h_1^{-1}$ and $h_2^{-1}$ exists.
\begin{equation}
\label{eqn1}
g_1= h_1^{-1}h_2g_2
\end{equation}

\begin{equation}
\label{eqn2}
g_2 = h_2^{-1}h_1g_1
\end{equation}

\begin{itemize}
\item[(i)] $Hg_1\subseteq Hg_2$ \\
Let $g'\in Hg_1$. This implies there exists a $h'\in H$ such that $g'=h'g_1$. Therefore,
\begin{align*}
g'&= h'g_1\\
g'&= h'(h_1^{-1}h_2g_2) \hspace*{10mm} [\text{By equation \eqref{eqn1}}]
\end{align*} 
By closure property in $(H,*)$, we have $h''=h'h_1^{-1}h_2\in H$. Therefore $g'=h''g_2$, $g'\in Hg_2$.
\item[(ii)] $Hg_2\subseteq Hg_1$ \\
Let $g'\in Hg_2$. This implies there exists a $h'\in H$ such that $g'=h'g_2$. Therefore,
\begin{align*}
g'&= h'g_2\\
g'&= h'(h_2^{-1}h_1g_1) \hspace*{10mm} [\text{By equation \eqref{eqn2}}]
\end{align*} 
By closure property in $(H,*)$, we have $h''=h'h_2^{-1}h_1\in H$. Therefore $g'=h''g_1$, $g'\in Hg_1$.
\end{itemize}
\end{proof}

\begin{lemma}
\label{lag2}
For every $g\in G$, $|Hg|=|H|$.
\end{lemma}
\begin{proof}
By construction, for every element in $H$ there exists an element in $Hg$. So $|Hg|\leq |H|$. 

Let us show : $|H|\leq |Hg|$.Suppose not, $|Hg|<|H|$. Then there exists $h_1,h_2\in H, h_1\neq h_2$ such that $h_1g = h_2g$. Since $(G,*)$ is a group, $g^{-1}$ exists. We have $h_1gg^{-1} = h_2gg^{-1}$ which implies $h_1=h_2$, a contradiction.
\end{proof}

\begin{theorem}(Lagrange's theorem.)
\label{lag-thm}
Let $(G,*)$ be a group and $H\leq G$. Then $|H|$ divides $|G|$.
\end{theorem}
\begin{proof}
Direct consequence of Lemmas \ref{lag1} and \ref{lag2}
\end{proof}


\begin{observation}
\label{obs-genset}
Let $H=<g>$ and $H'=<H,g'>$ where $g'\in G\backslash H$. Then $H\leq H'\leq G$. We have $g\in H'\backslash H$, therefore $|H'|>|H|$. $H'\leq H$. By Theorem \ref{lag-thm}, $|H'|\geq 2|H|$. This shows that every group has a generating set of size $\log|G|$.
\end{observation}


\begin{remark}
We know that $Aut(X)\leq S_n$ for any graph $X=(V,E)$. By Observation \ref{obs-genset} $Aut(X)$ has a generating set of size $\log|S_n| = log(n!) \in \mathcal{O}(n\log n)$ by Stirling's approximation.
\end{remark}  


\newpage \setcounter{section}{0}
\Lecture{Jayalal Sarma M.N.}{Aug 13, 2013}{5}{Multivariate Multipolynomial Division and Applications}{stud}
\noindent
Preamble

\section{Another Example Application - Geometric Theorem Proving}

\section{An Informal View}

\section{Algebraic Preliminaries}

\newpage \setcounter{section}{0}
\Lecture{Jayalal Sarma M.N.}{Aug 20, 2013}{6}{From Dickson's Lemma to Hilbert Basis Theorem}{stud}
           
\noindent
Preamble

\section{From Multivariate Polynomial Division Algorithm to Ideal Membership Problem}
\section{Proof of Hilbert Basis Theorem}
\section{Grobner Conditions for Basis}
\section{Ideal Membership Problem with Grobner basis as Input}

\newpage \setcounter{section}{0}
\Lecture{Jayalal Sarma M.N.}{Aug 21, 2013}{7}{Buchberger's Algorithm}   
\noindent
Preamble

\section{Constructing Counter-Examples to Grobner condition}

\section{$S$-polynomials and Buchberger's Criterion}

\section{Buchberger's Algorithm - Correctness \& Termination}
\subsection{Ascending Chain Condition for Ideals}

\section{Proof of Buchberger's criterion.}
\subsection{A structure Lemma for counter examples for Grobner condition}
\newpage \setcounter{section}{0}
\Lecture{Jayalal Sarma M.N.}{Feb 6, 2013}{8}{Proof of Buchberger's criterion.}{stud}

\noindent
Preamble

\section{A Structure Lemma}

\section{A Proof by Contradiction}

\newpage \setcounter{section}{0}
\Lecture{Jayalal Sarma M.N.}{Aug 28, 2013}{9}{Minimality, Elimination Theory}   

\noindent
Preamble

Minimal and Reduced Grobner Basis. Ideal Equality Problem. Uniqueness of Reduced Grobner basis. Elimination Theory and Elimination Ideals. Grobner Basis for the Elimination Ideals. Relating to the Robotics Arms problem.


\section{Minimal and Reduced Grobner Basis}

\section{Uniqueness of Reduced Grobner basis}

\section{Elimination Theory and Elimination Ideals}

\section{Grobner Basis for the Elimination Ideals}
\newpage \setcounter{section}{0}
\Lecture{Jayalal Sarma M.N.}{Feb 6, 2013}{10}{An application of Monomial ordering}{stud}

\noindent
Preamble

Applications of Grobner Basis. 3-coloring via Grobner basis. Testing membership in Kernel and Image of Ring Homomorphisms. Applications to Integer Programming.


\section{Applications of Grobner Basis}

\subsection{3-coloring via Grobner Basis}

\section{Integer Programming}
\subsection{Formulation as polynomials}
\subsection{$\mathbb{F}$-algebra homomorphism}
\subsection{Kernel}
\subsection{Testing membership in the image}
\subsection{Bringing in optimisation}


\newpage \setcounter{section}{0}
\Lecture{Jayalal Sarma M.N.}{Sep 4, 2013}{11}{Integer Programming using Grobner Basis}   

\noindent

Proof of the characterization of the Image of the k-algebra homomorphism. Observation about the Grobner basis in the special case of integer programming. Defining the monomial ordering to bring in optimization. Proof of optimality of the solution.

\section{Characterizing the Image}

\section{Observations on the Grobner Basis}

\section{Bringing in Optimality}

\section{Remarks about generalizations}
\newpage \setcounter{section}{0}
\Lecture{Jayalal Sarma M.N.}{Sep 3, 2013}{12}{From Root Finding to Factorization}{stud}

\noindent
\textit{Shorter Lecture:}
From Root finding to factorization of polynomials. Why or when is a polynomial completely factorizable over the underlying ring/field? Why should they be unique factorizable? Informal answers, and directions to explore.

\section{Number of Roots}

\section{A Linear Algebraic Method - The companion Matrix}

\section{From Roots to Factorization}

\section{Informal Answers and the road ahead}

\newpage \setcounter{section}{0}
\Lecture{Jayalal Sarma M.N.}{Sep 4, 2013}{13}{Unique Factorization Domains}{stud}

\noindent
Irreducible and Prime Elements in an Integral Domain. Primes are irreducible. When is it that all irreducibles are primes? A proof that this is exactly when the domain is a UFD.
A field is a UFD. Every principal Ideal Domain (example: F[x] and Z) is a UFD.
What about rings like F[x1,x2], and Z[x]?
Gauss's theorem : If R is a UFD, then so is R[x].
Proof of the theorem using the characterization about irreducibles in R[x]. The case when the irreducibles are from R itself (Gauss's Lemma).

\section{Irreducible vs Prime Elements of an Integral Domain}

\section{A characterization of Unique Factorizability}

\section{A Non-trivial Application - All Principal Ideal Domains are Unique Factorization Domains}

\section{Gauss's Theorem}
\subsection{All constant primes in $R$ are primes in $R[x]$}
\subsection{All non-constant irreducibles in $\Z[x]$ are irreducibles in $\Q[x]$}
All primes in $\Q[x]$ are primes in $\Z[x]$

\newpage \setcounter{section}{0}
\Lecture{Jayalal Sarma M.N.}{Sep 10, 2013}{14}{Irreducibility}   

\noindent
Continuing the proof of Gauss's theorem : The case when the irreducibles are from the R[x]. Moving the argument to the Field of Fractions.
Conclusion : Two tasks are well-framed.
\begin{itemize}
\item How do we detect irreducibility?
\item How do we factorize into irreducibile factors?
\end{itemize}

Eisenstein criterion for irreducibility. 

\section{Completing Gauss's Theorem}

\section{A Remark about Field of Fractions}

\section{Eisenstein criterion for irreducibility}
\newpage \setcounter{section}{0}
\Lecture{Jayalal Sarma}{Aug 26, 2015}{15}{Group Intersection Problem (Contd)}{Jayalal Sarma}{$\alpha$}{Jayalal Sarma}


\newpage \setcounter{section}{0}
\Lecture{Jayalal Sarma M.N.}{Sep 10, 2013}{16}{More on Fields}   

\noindent

Quick introduction to vector spaces. Viewing field extensions as vector spaces. Characterestic of a field. Sizes of fields. Constructing extensions and uniqueness of fields of a given size (up to isomorphism).

\section{Field Extensions as Vector Spaces}

A vector space over a field $\F$ is a set $V$ with two kinds of operations - addition and scalar multiplications - satisfying the following properties. Elements of $V$ are called vectors and elements of $\F$ are called scalars.
\begin{itemize}
\item $(V,+)$ forms an abelian group.
\item If $\alpha$ is a scalar, and $v$ is a vector, then $\alpha v$ is a vector.
\item If $\alpha$ is a scalar, and $u$ and $v$ are vectors, then $\alpha(u+v)$ is the same vector as $\alpha u+ \alpha v$.
\item If $\alpha_1$, $\alpha_2$ are scalars, and $v$ is a vector, then $\alpha_1(\alpha_2 v)$ is the same vector as $(\alpha_1 \alpha_2)v$ where $\alpha_1\alpha_2$ is the multiplication in $\F$.
\end{itemize}

Some easy examples are the set of points in $\R \times \R$. Set of polynomials of degree $d$ over a field $\F$ forms a vector space with the natural notion of addition and multiplication.

Let $E$ be a field and $F$ be a subfield of it. One can view $E$ as a vector space over $F$. To see this, view the elements of $F$ as scalars and the elements of $E$ as vectors in the above definition.

\subsection{Linear Independence, Basis, and Dimension}

\begin{definition}
A set of vectors $v_1, v_2, \ldots v_k$ are said to be \textit{linearly independent}, if:
\[ \alpha_1v_1 + \alpha_2 v_2 + \ldots + \alpha_k v_k = 0 \Longrightarrow \alpha_1 = 0 \land \alpha_2 = 0 \land \ldots \land \alpha_k = 0 \]
\end{definition}
For any set $S$ of vectors, the set of vectors spanned by it, denoted by  {\sc Span}$(S)$ is the set of vectors that can be expressed as the linear combination of vectors in $S$. A set $S$ is said to be a basis of a vector space, if $S$ itself is linearly independent and the {\sc Span}$(S)$ is the whole space.

We need two observations about the basis of a vector spaces and basis. 

All basis of a vector space are of the same size. Suppose there is an $S$ and an $S'$ which forms the basis of the same vector space $V$, and $|S| > |S'|$. Since $S$ and $S'$ are subsets of $V$ itself, the elements of $S'$ \jsay{This needs to be completed.}

Since all basis of a vector space are of the same size - it must be the case that.
\subsection{Minimal Polynomials - viewing adjoining as a vector space}

\section{Characterestic of Rings \& Fields}

Consider the following Ring homomorphism from $\Z$ to a ring $R$.
\[ \phi : \Z \to R \]
where, $\forall a \in \Z$, $\phi(a) = |a|.1$ if $a > 0$, and $\phi(a) = |a|.(-1)$ if $a < 0$. where $n.1$ is simply a notation for adding the identity of the ring $R$, $n$ times to itself.

We will first check that it is a ring homomorphism. \jsay{Yet to be written}

Let $I$ be the kernel of this map. Since $\Z$ is a principal ideal domain, $I$ is singly genereated and the generator is simply the least number in absolute value. Let $\ell$ be the generator of the ideal. We know that the ideal $I$ is simply $\ell \Z$. This $\ell$ is called the characterestic of the ring $R$. In other words, {\em characterestic of a ring $R$ with identity is simply the smallest number of times one needs to add $1$ to get to 0}. Indeed, it is possible that adding the identity to itself never gets to $0$ of the ring. In this case $I = {0}$ and $\ell=0$ - we say that the characterstic of the ring is $0$.

We explore more properties that can be derived from this $\ell$. 
Let $R'$ be the image of this homomorphism. We know that $R'$ is a subring of $R$. Consider the quotient ring $\Z/\ell\Z$ which is same as $\Z_\ell$. By the first isomorphism theorem we have the following:
$\Z_\ell \cong R'$. In other words, if the characterestic of a ring $R$ is $\ell$, then there is an isomorphic copy of $\Z_\ell$ sitting inside $R$ as a subring.

Now we turn to characterestic of a field. First of all let us argue that it can only be zero or a prime number.

\begin{lemma}
The characterestic of a field is either a prime number or zero.
\end{lemma}
\begin{proof}
Let $F$ be a field. Suppose the characterestic is not a prime and is $\ell \in \Z$. Assume for the constradiction that $\ell$ is not prime. $\ell = p.q$ where $p, q < n$. Indeed, $\ell$ is least integer such that $\ell.n = 0$. Hence $p.1 \ne 0$ and $q.1 \ne 0$. Since $\phi$ is a homomorphism, associated with the $\ell$ that we discussed above, $\phi(pq) = \phi(p)\phi(q)$. Since the LHS is $0$, $\phi(p)$ and $\phi(q)$ forms zero divisors in $\F$. Thus, we have arrived at a contradiction and hence the lemma.
\end{proof}

\begin{corollary}
Any finite field must have a subfield whose order is a prime number.
\end{corollary}
\begin{proof}
Let $F$ be a finite field. We first argue that the characterestic cannot be zero. If it is zero, then we know that the ideal $I$ in the above discussion is the zero ideal and hence the quotient ring is $\Z$ itself.
Hence, $\exists R' \subseteq F$ such that $Z \cong R'$, which implies that $\F$ must have infinite cardinality. Thus, characterestic can only be a prime number. Thus, an isomorphic copy of $Z_p$ for some prime $p$ must be present in every field.
\end{proof}

\section{Sizes of Finite Fields}

We combine the ideas developed in the previous two sections to conclude that the sizes of finite fields cannot be arbitrary.

\begin{lemma}
The size of any finite field is of always of the form $p^d$ for some prime $p$ and a non-negative integer $d$.
\end{lemma}
\begin{proof}
$\Z_p$ (for some prime $p$) appears a subfield(up to isomorphism) of any finite field. Let $d$ be the dimension of $F$ as a vector space over $\Z_p$. That is, there is a subset $S \subseteq F$ with $|S| = d$, which forms the basis of $F$ over $\Z_p$. Let us say that $S = \{ a_1, a_2, \ldots a_d \}$. Indeed, each vector (each element of $a \in \F$ can be viewed as a $d$-tuple $(\alpha_1, \alpha_2, \ldots, \alpha_d)$ such that $a = \alpha_1a_1 + \alpha_2a_2 + \ldots + \alpha_d a_d$. Can two tuples represent the same $a$? No, because it would mean then that $\sum_i \alpha_i a_i = \sum_i \alpha' a_i$. This contradicts the fact that $S$ is linearly independent (since it forms a basis of $\F$). Hence there are precisely $p^d$ tuples possible, each of them representing distinct elements of $\F$ as a vector space over $\Z_p$. Hence the size of the field $\F$ must be exactly $p^d$.
\end{proof}

\section{Constructing Field Extensions}

For any $p$ and $d$, is there a field of size $p^d$. We will answer this question positively.

Consider a polynomial $p(x)$ of degree $d$ that is irreducible over $\Z_p$. Let $a$ be a root of such a polynomial. Clearly $a \notin \Z_p$. Consider the field $\Z_p/\langle p \rangle$. This is isomorphic to $\Z_p(a)$ which also is a vector space over $\Z_p$. 

We argue that the size of this finite field is precisely $p^d$. First of all, we argue that any element of the space $\Z_p(a)$ can be viewed as a linear combination of elements in $\{1, a, a^2, \ldots, a^{d-1}\}$. We do not need an $a^d$ in this expression - indeed, it can be written as a combination of the other elements of lesser power since $p(a) = 0$. Suppose there is a linear combination of $(1, a, a^2, \ldots, a^{d-1})$ that goes to zero, that is $a$ is a root of the polynomial of lesser degree than $p(a)$. But then there is a polynomial of degree less than $p(x)$ which has $a$ as the root.

\section{Uniqueness of Fields up to isomorphism}

We will be greedy, for any $p$ and $d$, are there two non-isomorphisc fields of size $p^d$? We will answer this question negatively. So, we can always talk about {\em the} field of size $p^d$.
\jsay{Define splitting field etc.}
\begin{lemma}
The splitting field of a polynomial are always isomorphic to each other.
\end{lemma}
\begin{proof}
\jsay{Yet to be written}
\end{proof}

\begin{lemma}
For any field $\F$, there is a polynomial whose splitting field is $\F$.
\end{lemma}
\begin{proof}
Let $|\F| = k$. Consider the multiplicative group $\F - \{0\}$. Let $g$ be an element in  this group. We know by Lagrange's theorem, $g^{k-1} = 1$ where $1$ is the multiplicative identity. Thus for all $g \in \F$, $g^k = g$. Thus all of them are roots of the polynomial $x^k - x$, as a polynomial in $\F[x]$. Since this polynomial can have at most $k$ roots, the polynomial completely splits in $\F$ and it does not split in any subfield of $\F$. Hence $\F$ is the splitting field of the polynomial $x^k - x$. 
\end{proof}

By combining the above two lemmas, we get the main point of this section. That finite fields of a fixed size must be isomorphic.
\newpage \setcounter{section}{0}
\Lecture{Jayalal Sarma M.N.}{Sep 18, 2013}{17}{Warming up to Berlekamp's Factorization Algorithm} 

\noindent

Back to Factorization problem. A starting idea to use Fermat's little theorem to extract product of linear factors. Reduction to Squarefree case.
Frobenius map $(x^q = x)$ and the sub-algebra of the quotient ring $F[x]/f$. Dimension of the sub-algebra when f is irreducible.
\newpage \setcounter{section}{0}
\Lecture{Jayalal Sarma M.N.}{Sep 10, 2013}{18}{Berlekamp's Lemma}

\noindent

Chinese remaindering. Berlekamp algebra W and its dimension. From an element in W to factorization - Berlekamp's Lemma.
\newpage \setcounter{section}{0}
\Lecture{Jayalal Sarma M.N.}{Sep 25, 2013}{19}{Berlekamp's Factorization Algorithm} 

\noindent

Writing a set of linear equations and the Berlekamp Matrix. The $O((qn^3)(n^2)(qn^2))$ time algorithm. Reducing the first factor of q by faster exponentiation. Removing the second factor of q. Identifying the minimal polynomial for the $g(x)$.
\newpage \setcounter{section}{0}
\Lecture{Jayalal Sarma M.N.}{Oct 1, 2013}{20}{Berlekamp's Factorization Algorithm}{stud}

\noindent

Computing the minimal polynomial of g(x). Factorizing the minimal polynomial by Rabin's factorization method. Discussions on effect of choosing g(x) in W, at random.

Berlekamp's Algorithm.

\section{The Berlekamp Subalgebra $\mathbb{W}$}

\section{From Number of Irreducible Factors to Dimension}

\section{Using $\mathbb{W}$ for factorization}

\section{Constructing a basis for $\mathbb{W}$ - a linear algebraic approach}


\end{document}
