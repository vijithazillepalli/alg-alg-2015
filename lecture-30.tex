\Lecture{Jayalal Sarma}{Sep 23, 2015}{30}{Term Ordering}{Naga Varun}{$\beta$}{Ramya C}
We will first complete Eucledian algorithm and proof of lemma 30.1 from the previous lecture and then look at term ordering.
\begin{lemma}
$$I = <f_1, f_2>, f_1 \neq 0\;and\;f_2\neq 0$$
then,$$I=<gcd(f_1,f_2)>$$
\end{lemma}
\begin{proof}
We know from Theorem 29.6 that for any $g$ of least degree in $I$, $<g>=I$ holds true. We also saw that there is exactly one $g$ of least degree in $I$ s.t $lc(g)=1$(where $lc(g)$ is the leading coefficient of $g$). We'll prove that this $g$ is indeed $gcd(f_{1},f_{2})$, thus proving $<gcd(f_{1},f_{2})>=I$.
$$<g>=I\;\;\&\;\;f_{1},f_{2}\in I$$ 
$$\therefore \exists\ g_{1},g_{2}\in \mathbb{F}[x]\;\;s.t\;\;f_{1}=g_{1}.g\;\;\&\;\;f_{2}=g_{2}.g$$ 
$$\therefore g|f_{1}\;\;\&\;\;g|f_{2}$$.
$$g\in I\;\;\&\;\;I=<f_{1},f_{2}>$$
$$\therefore\exists u_{1},u_{2}\in \mathbb{F}[x]\;s.t\;g=u_{1}.f_{1}+u_{2}.f_{2}$$
$$\therefore h|f_{1}\;\;\&\;\;h|f_{2} \implies h|g$$
and we chose $lc(g)=1$
therefore, from the defn. of $gcd$, $g=gcd(f_{1},f_{2})$
\end{proof}
\section*{Eucledian algorithm:}
Since $<f_{1},f_{2}>=<f_{1}-qf_{2},f_{2}>\forall q\in \mathbb{F}[x]$, from Lemma 30.1  $gcd(f_{1},f_{2})=gcd(f_{1}-qf_{2},f_{2})$.
\begin{algorithm}
\caption{Eucledian algorithm for finding the gcd of two Univariate polynomials}\label{div-euclid}
\begin{algorithmic}[1]
\Procedure{GCD}{ Input : Polynomials $f$ and $g$ (where $deg(g)\leq deg(f)$ and $f,g\neq0$)}
\State \emph{loop}:
\If{$g\neq 0$}
\State Use Algorithm 6 to get $q$ and $r$ s.t $f=qg+r$ where $deg(r)< deg(g)$.
\State $f=g\;\&\;g=r$
\State \textbf{goto} \emph{loop}
\EndIf
\State $f=f/lc(f)$
\State \textbf{return} $f$
\EndProcedure
\end{algorithmic}
\end{algorithm}
\section*{Term Ordering:}
%formal defn. of term ordering
%well-ordering
%examples of term-ordering
%over $\mathbb{F}[x_1,....x_n]$

\begin{definition}(Power Products)
Set of all power products($T^n$) over $\mathbb{F}[x_1,....x_n]$ is defined as follows:
$$T^n=\{x_{1}^{\alpha_{1}}..x_{i}^{\alpha_{i}}..x_{n}^{\alpha_{n}}\;|\; \alpha_{i}\in \mathbb{N},\;1\leq i\leq n\}$$
\end{definition}
We will represent a power product/monomial $x_{1}^{\alpha_{1}}..x_{i}^{\alpha_{i}}..x_{n}^{\alpha_{n}}$ by $x^{\alpha}$ 
\begin{definition}(Total Order)
An ordering relationship $<$ over $T^n*T^n$ is called a Total Order iff it satisfies the following properties:

$ \forall x^{\alpha},x^{\beta},x^{\gamma} \in T^n$

$$ x^{\alpha} < x^{\beta}\; and \;x^{\beta} < x^{\alpha} \implies x^{\alpha} = x^{\beta} (antisymmetry)$$
$$ x^{\alpha} < x^{\beta} \;and\; x^{\beta} < x^{\gamma} \implies x^{\alpha} < x^{\gamma} (transitivity)$$
$$x^{\alpha} < x^{\beta} \;or\; x^{\beta} < x^{\alpha} (totality)$$.

Note here that $x^{\alpha} < x^{\alpha}$ from $totality$ property.
\end{definition}

\begin{definition}[Well Ordering] 
An ordering relationship $<$ over $T^n*T^n$ is called a Well Ordering iff it's a Total Order and has a property that every non-empty subset of $T^n$ has a least element in this ordering.
\end{definition}

\begin{definition}[Term Ordering] \label{def:term-order}
An ordering relationship $<$ over $T^n*T^n$ is called a Term Ordering iff it's a Total Order and satisfies the following properties:
$$1<x^{\alpha}\;\; \forall x^{\alpha}\in T^n$$
$$x^{\alpha}<x^{\beta}\implies x^{\alpha}x^{\gamma}<x^{\beta}x^{\gamma}\;\;\forall x^{\alpha},x^{\beta},x^{\gamma}\in T^n$$
\end{definition}

\begin{exercise}
Prove that every Term Ordering is also a Well Ordering ( Hint: Use Hilbert's basis theorem).
\end{exercise}
\section*{Examples:}
The following examples define ordering relationships $<$ over $T^{n}*T^{n}$
\subsection*{lex ordering:}
$$x^{\alpha},x^{\beta} \in T^{n}$$
$$x^{\alpha}=x_{1}^{\alpha_{1}}....x_{n}^{\alpha_{n}}$$
$$x^{\alpha}<x^{\beta}\; iff \;(x^{\alpha}=x^{\beta}) or ( \;for \;first \;i \;s.t\; \alpha_{i}\neq \beta_{i} , \alpha_{i}<\beta_{i})$$.
It's an exercise to prove that this is a Total Order. We'll prove that this is a term ordering.

Let us take $x^{\alpha}=1\;\&\;x^{\beta}\neq1$
$$\therefore\exists 1\leq i\leq n\;s.t\;\beta_{i}>0 \;\&\;(\beta_{j}=0 \;\forall 1\leq j<i)$$
$$since\;\forall 1\leq j<i \;\alpha_{j}=\beta_{j}=0\;\&\;0=\alpha_{i}<\beta_{i},\;$$
$$1<x^{\beta}$$
$1<1$ is trivially true.
$$\therefore1<x^{\beta}\;\forall\;x^{\beta}\in T^{n}$$

$$x^{\alpha}=x^{\beta} \implies x^{\alpha}x^{\gamma}=x^{\beta}x^{\gamma}\;which\;implies\;x^{\alpha}x^{\gamma}<x^{\beta}x^{\gamma}$$
$$x^{\alpha}<x^{\beta} \;\&\;x^{\alpha}\neq x^{\beta}\implies \exists 1\leq i\leq n\;s.t\;\beta_{i}>\alpha_{i} \;\&\;(\beta_{j}=\alpha_{j} \;\forall 1\leq j<i)$$ 
$$\beta_{i}+\gamma_{i}>\alpha_{i}+\gamma_{i} \;\&\;(\beta_{j}+\gamma_{i}=\alpha_{i}+\gamma_{i} \;\forall 1\leq j<i)$$
$$x^{\alpha}x^{\gamma}=x_{1}^{\alpha_{1}}....x_{n}^{\alpha_{n}}.x_{1}^{\gamma_{1}}....x_{n}^{\gamma_{n}}=x_{1}^{\alpha_{1}+\gamma_{1}}....x_{n}^{\alpha_{n}+\gamma_{n}}=x^{\alpha+\gamma}\;(Commutative \;property \;of \;\mathbb{F}[x_{1}....x_{n}] \;is \;used \;here) $$
$$\therefore x^{\alpha}<x^{\beta}\implies x^{\alpha}x^{\gamma}<x^{\beta}x^{\gamma}\; \forall x^{\alpha},x^{\gamma},x^{\beta}\in T^{n}$$
\subsection*{degree lex ordering:}
$x^{\alpha}<x^{\beta}$ if $\sum\alpha_{i}<\sum\beta_{i}$ or ($\sum\alpha_{i}=\sum\beta_{i}\; and\; x^{\alpha}<x^{\beta}\; w.r.t \;lex\; ordering)$
\subsection*{revlex ordering:}
$x^{\alpha}<x^{\beta}$ if $\sum\alpha_{i}<\sum\beta_{i}$ or ($\sum\alpha_{i}=\sum\beta_{i}\; and\; x^{\alpha}>x^{\beta}\; w.r.t\; lex \;ordering)$.
$$$$
We can also prove that 'degree lex ordering' and 'revlex ordering' are Term Orderings.
\begin{exercise}
The usual $<$ ordering of $\mathbb{Z}_{\ge 0}$ has some nice (trivial to state) properties.
\begin{itemize}
\item ~there are only finitely many integers between any two integers. 
\item ~$\alpha > 0$ for all nonzero $\alpha \in \mathbb{Z}$. 
\end{itemize}
Let $<$ be a total order on $\mathbb{Z}_{\ge 0}^n$ that satisifes: $\forall \alpha, \beta$,  $\alpha > \beta \implies \alpha+\gamma > \beta+\gamma$.
Check if any of the above conditions are necessary and sufficient $<$  to be a monomial ordering?
\end{exercise}

