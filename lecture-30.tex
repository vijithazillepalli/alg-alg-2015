\Lecture{Jayalal Sarma}{Sep 23, 2015}{30}{Term Ordering}{Naga Varun}{$\beta$}{Ramya C}
We will first complete proof of the following lemma from the previous lecture and then look at the Euclidean algorithm.

\begin{lemma}
\label{gcd}
Let $f_1 \neq 0$ and $f_2\neq 0$. Let $I = <f_1, f_2>$. Then
$$I=<gcd(f_1,f_2)>$$
\end{lemma}
\begin{proof}
We know from Theorem 29.6 that for any $g$ of least degree in $I$, $I=<g>$ holds. We also saw that there is exactly one $g$ of least degree in $I$ such that $lc(g)=1$ (where $lc(g)$ is the leading coefficient of $g$). We will prove that this $g$ is indeed $gcd(f_{1},f_{2})$, thus proving $<gcd(f_{1},f_{2})>=I$.

Let $I=<g>$ and $f_{1},f_{2}\in I$. Since $f_{1},f_{2}\in I$ there exists $g_{1},g_{2}\in \mathbb{F}[x]$ such that $f_{1}=g_{1}.g$ and $f_{2}=g_{2}.g$. Therefore $g\mid f_{1}$ and $g\mid f_{2}$. \\

Since $g\in I$ there exists $u_{1},u_{2}\in \mathbb{F}[x]$ such that $g=u_{1}.f_{1}+u_{2}.f_{2}$. Therefore if $h\mid f_{1}$ and 
$h\mid f_{2}$ then $h\mid g$. We chose $lc(g)=1$.\\
By the definition of $gcd$, $g=gcd(f_{1},f_{2})$
\end{proof}

Having seen that the $gcd$ is the generator of the ideal we will now see how to compute the gcd of the set of polynomials using the familiar Euclidean algorithm.


\section*{Euclidian algorithm:}
\begin{observation}
For all $q\in \mathbb{F}[x]$ observe that
$<f_{1},f_{2}>=<f_{1}-qf_{2},f_{2}>$
\end{observation}

From Lemma \ref{gcd} we have $gcd(f_{1},f_{2})=gcd(f_{1}-qf_{2},f_{2})$.


\begin{algorithm}
\caption{Euclidean algorithm for finding the gcd of two Univariate polynomials}
\label{euclid}
\begin{algorithmic}[1]
\Procedure{GCD}{ Input : Polynomials $f$ and $g$ (where $deg(g)\leq deg(f)$ and $f,g\neq0$)}
\State \emph{loop}:
\If{$g\neq 0$}
\State Divide $f$ by $g$ to get $q$ and $r$ s.t $f=qg+r$ where $deg(r)< deg(g)$.
\State $f=g$
\State $g=r$
\State \textbf{goto} \emph{loop}
\EndIf
\State $f=\frac{f}{lc(f)}$
\State \textbf{return} $f$
\EndProcedure
\end{algorithmic}
\end{algorithm}



\section*{Term Ordering:}
%formal defn. of term ordering
%well-ordering
%examples of term-ordering
%over $\mathbb{F}[x_1,....x_n]$


Having completed the univariate case, let us now get to the multivariate case.


\begin{definition}(Power Products)
Set of all power products $(T^n)$ over $\mathbb{F}[x_1,\dots,x_n]$ is defined as follows:
$$T^n=\{x_1^{\alpha_{1}}\cdot x_2^{\alpha_{2}}\cdots x_{n}^{\alpha_{n}}\;|\; \alpha_{i}\in \mathbb{N},\;1\leq i\leq n\}$$
\end{definition}
Observe that a power product is a multilinear monomial.

We will represent a power product/monomial $x_1^{\alpha_{1}}\cdot x_2^{\alpha_{2}}\cdots x_{n}^{\alpha_{n}}$ by $x^{\alpha}$ 


\begin{definition}(Total Order)
An ordering relationship $<$ over $T^n \times T^n$ is called a total Order iff it satisfies the following properties:
$ \forall x^{\alpha},x^{\beta},x^{\gamma} \in T^n$
\begin{itemize}
\item \textbf{Antisymmetry :}$ x^{\alpha} < x^{\beta}\; and \;x^{\beta} < x^{\alpha} \implies x^{\alpha} = x^{\beta}$
\item \textbf{Transitivity : }$ x^{\alpha} < x^{\beta} \;and\; x^{\beta} < x^{\gamma} \implies x^{\alpha} < x^{\gamma} $
\item \textbf{ Totality :} $x^{\alpha} < x^{\beta} \;or\; x^{\beta} < x^{\alpha}$
\end{itemize}

Note that $x^{\alpha} < x^{\alpha}$ from $totality$ property.
\end{definition}

\begin{definition}(Well Ordering)
An ordering relationship $<$ over $T^n\times T^n$ is called a Well Ordering iff it is a Total Order and has a property that every non-empty subset of $T^n$ has a least element in this ordering.
\end{definition}

\begin{definition}(Term Ordering)
An ordering relationship $<$ over $T^n\times T^n$ is called a Term Ordering iff it is a Total Order and satisfies the following properties:
\begin{itemize}
\item $1<x^{\alpha}$ for all $x^{\alpha}\in T^n$
\item For all $x^{\alpha},x^{\beta},x^{\gamma}\in T^n$ we have $x^{\alpha}<x^{\beta}\implies x^{\alpha}x^{\gamma}<x^{\beta}x^{\gamma}$
\end{itemize}
\end{definition}

\begin{exercise}
Prove that every Term Ordering is also a Well Ordering.\\
Hint: Use Hilbert's basis theorem.
\end{exercise}




\section*{Examples:}
The following examples define ordering relationships $<$ over $T^{n}\times T^{n}$
\subsection*{Lex ordering:}
Let $x^{\alpha},x^{\beta} \in T^{n}$ where
$x^{\alpha}=x_{1}^{\alpha_{1}}\cdots x_{n}^{\alpha_{n}}$ and $x^{\beta}=x_{1}^{\beta_{1}}\cdots x_{n}^{\beta_{n}}$. We say 
$x^{\alpha}<x^{\beta}$ iff $x^{\alpha}=x^{\beta}$ or for the first $i$ such that $\alpha_{i}\neq \beta_{i}, \alpha_{i}<\beta_{i}$.
It's an exercise to prove that this is a Total Order. 

We will prove that this is a term ordering.

Let $x^{\alpha}=1\;\&\;x^{\beta}\neq1$. Therefore
there exists $1\leq i\leq n$ such that $\beta_{i}>0$ and $\beta_{j}=0$ for all $1\leq j<i$. Since 
for all $1\leq j<i, \alpha_{j}=\beta_{j}=0$ and
$0=\alpha_{i}<\beta_{i}$ we have $1<x^{\beta}$.
$1<1$ is trivially true. Therefore
$1<x^{\beta}$ for every $x^{\beta}\in T^n$.\\


If $x^{\alpha}=x^{\beta}$ then $x^{\alpha}x^{\gamma}=x^{\beta}x^{\gamma}$. Therfore by definition of lex ordering, $x^{\alpha}x^{\gamma}<x^{\beta}x^{\gamma}$.

If $x^{\alpha}<x^{\beta}$ and $x^{\alpha}\neq x^{\beta}$ then there exists $1\leq i\leq n$ such that $\beta_{i}>\alpha_{i}$ and $\beta_{j}=\alpha_{j}$ for all $1\leq j<i$. Since $\beta_{i}+\gamma_{i}>\alpha_{i}+\gamma_{i}$ and $(\beta_{j}+\gamma_{j}=\alpha_{j}+\gamma_{j}$ for all $1\leq j<i)$, by commutative property of $\mathbb{F}[x_{1},\ldots,x_{n}]$ we have
\begin{center}
$x^{\alpha}x^{\gamma}=x_{1}^{\alpha_{1}}\cdots x_{n}^{\alpha_{n}}\cdot x_{1}^{\gamma_{1}}\cdots x_{n}^{\gamma_{n}}=x_{1}^{\alpha_{1}+\gamma_{1}}\cdots x_{n}^{\alpha_{n}+\gamma_{n}}=x^{\alpha+\gamma}$
\end{center}
Therefore,
$x^{\alpha}<x^{\beta}\implies x^{\alpha}x^{\gamma}<x^{\beta}x^{\gamma}\; \forall x^{\alpha},x^{\gamma},x^{\beta}\in T^{n}$


\subsection*{Degree lex ordering:}
Let $x^{\alpha},x^{\beta} \in T^{n}$ where
$x^{\alpha}=x_{1}^{\alpha_{1}}\cdots x_{n}^{\alpha_{n}}$ and $x^{\beta}=x_{1}^{\beta_{1}}\cdots x_{n}^{\beta_{n}}$. We say $x^{\alpha}<x^{\beta}$ iff 
\begin{itemize}
\item $\sum\alpha_{i}<\sum\beta_{i}$ 
\item [or]
\item $\sum\alpha_{i}=\sum\beta_{i}$ and $x^{\alpha}<x^{\beta}$ with respect to the lex ordering. 
\end{itemize}

\subsection*{Revlex ordering:}
Let $x^{\alpha},x^{\beta} \in T^{n}$ where
$x^{\alpha}=x_{1}^{\alpha_{1}}\cdots x_{n}^{\alpha_{n}}$ and $x^{\beta}=x_{1}^{\beta_{1}}\cdots x_{n}^{\beta_{n}}$. We say 
$x^{\alpha}<x^{\beta}$ iff
\begin{itemize}
\item $\sum\alpha_{i}<\sum\beta_{i}$ 
\item [or]
\item  $\sum\alpha_{i}=\sum\beta_{i}$ and $x^{\alpha}>x^{\beta}$ with respect to the lex ordering. 
\end{itemize}

We can also prove that 'degree lex ordering' and 'revlex ordering' are Term Orderings.
