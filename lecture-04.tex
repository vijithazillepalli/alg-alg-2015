\Lecture{Jayalal Sarma M.N.}{Aug 7, 2013}{4}{Graphs, Groups and Generators}
\noindent
Preamble


\section{Graph Isomorphism}


\begin{defn}(Graph Isomorphism.)
Let $G_1=(V_1,E_1)$ and $G_2=(V_2,E_2)$ be graphs. We say $G_1\stackrel{\sim}{=}G_2$(read as $G_1$ is \em isomorphic to $G_2$) if there exists a bijection $\sigma : V_1\rightarrow V_2$ such that $\forall (u,v) \in V_1\times V_2$ we have
\begin{center}
$(u,v)\in E_1 \iff (\sigma(u),\sigma(v))\in E_2$
\end{center}
\end{defn}

In other words, we say a graph $G_1$ is isomorphic to $G_2$ if there exists a relabeling of the vertices in $G_1$  such that the the adjacency and non-adjacency relationships in $G_2$ is preserved. 
\begin{obs}
If $|V_1|\neq |V_2|$ we have that $G_1$ is not isomorphic to $G_2$.
\end{obs}

The graph isomorphism problem is stated as follows. 
\begin{center}
\fbox{
\begin{minipage}{10 cm}\textbf{PROBLEM : GRAPH ISOMORPHISM}\\
\textbf{Input} : $G_1=(V_1,E_1), G_2=(V_2,E_2)$\\
\textbf{Output} : Decide if $G_1\stackrel{\sim}{=}G_2$ or not.
\end{minipage}
}
\end{center}

A natural question to ask in this setting is that if there is an isomorphism from a graph $G$ to itself. 

Let $[n]=\{1,2,\ldots,n\}$. Let $S_n$ denote the set of all permuatations from the set $[n]$ to $[n]$.

\begin{defn}(Graph Automorphism.) 
Let $G=(V,E)$ be a graph. An automorphism of $G$ is a bijection $\sigma:V\rightarrow V$ such that $\sigma(G)= G$. Let
\begin{center}
$Aut(G)= \{\sigma | \sigma\in S_n \text{~and~} \sigma(G)= G \}$
\end{center}
be the set of all automorphisms of $G$. 
\end{defn}



The graph automorphism problem is stated as follows. 
\begin{center}
\fbox{
\begin{minipage}{10 cm}\textbf{PROBLEM : GRAPH AUTOMORPHISM}\\
\textbf{Input} : A graph $G=(V,E)$\\
\textbf{Output} : Construct $Aut(G)$.
\end{minipage}
}
\end{center}



\begin{obs}
Let $\tau:[n]\rightarrow[n]$ be the identity permuataion. That is, for all $i\in[n],\tau(i)=i$. Then by definition $\tau\in Aut(G)$ for any graph $G=(V,E)$. This identity permuataion $\tau$ is in $Aut(G)$. 
\end{obs}



Are there other permutations from $[n]$ to $[n]$ that are in the set $Aut(G)$ ? Formally the graph rigidity problem is stated as follows.


\begin{center}
\fbox{
\begin{minipage}{10 cm}\textbf{PROBLEM : GRAPH RIGIDITY}\\
\textbf{Input} : A graph $G=(V,E)$\\
\textbf{Output} : Decide if $Aut(G)$ is trivial or not.
\end{minipage}
}
\end{center}

Is the set $Aut(G)$ just a set or does it have more algebraic structure ? 

\begin{ex}
The set $Aut(G)$ forms a group under the composition operation.
\end{ex}

\section{Euclidean Algorithm}

\section{Termination \& Correctness}

\section{Solution to Ideal Membership Problem}

\section{Monomial Ideals}

\section{Dickson's Lemma}