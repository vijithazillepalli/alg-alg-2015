\Lecture{Jayalal Sarma}{October 12 2015}{38}{Univariate Polynomial
Factorization and Unique Factorization Domains}{K Dinesh}{$\gamma$}{K Dinesh}

Recall that our aim is to obtain a solution to the system of polynomial
equations $f_i(x_1,\ldots,x_n) = 0$ for $i \in [k]$ where $f_i \in \polyring$.
Let $I = I \cap \F[x_i,\ldots, x_n]$.

We saw that given $f_1,\ldots,f_k$, with respect to some term ordering and $G$
as its Grobner basis,  it is possible to obtain the ideal $I_i = I \cap
\F[x_i,\ldots, x_n]$ (called the elimination ideal). We saw in last lecture
that the generator set of $I_i$, is precisely all polynomials in $G \cap
\F[x_i,\ldots, x_n]$. Using elimination ideals, we can express the equations
$f_i$ as another polynomial $f_i'$ which is only in variables $x_i,
\ldots,x_n$ such that the ideals generated by the new set of polynomials is
same as $I$. Hence it suffices to solve the following system of equations. 
\begin{align*}
	f_1'(x_1,\ldots,x_n) & = 0 \\
	\vdots \\
	f_{n-1}'(x_{n-1},x_n)  & = 0 \\
	f_n'(x_n) & = 0
\end{align*}

Now if we can solve for the univariate case $f_n'(x_n) = 0$, then the solution 
can be substituted back in $f_{n-1}'(x_{n-1},x_n) = 0$ to get a univariate 
polynomial in $x_{n-1}$ and is solved similarly. This highlights the
need for algorithms for solving univariate polynomials.

Natural questions that comes up are : when does a univariate polynomial admit
a solution ? Note that if we can factorize a polynomial we can get it
solution. Hence we need to understand more general question : when can a
polynomial be factorized ? when is it unique ? can we characterize
polynomials that cannot be factored ?

\section{Units, Irreducibles and Primes}
Recall that in the ring of integers ($\Z,+, \cdot)$ we have a very
good understanding about the notion of factorization, primes etc.
We want to generalize these notion to an arbitrary ring and ask
questions : when is an element factorizable, what are irreducibles,
what are primes etc?

Recall the definitions of algebraic structures from earlier lectures (See
section~\ref{sec:alg-defs}). 
\begin{definition}
	A ring $(R,+,\cdot)$ is an algebraic structure with $(R,+)$ being an
	Abelian group, $(R,\cdot)$ is a monoid. Also the operation $\cdot$
	distributes over $+$.  An integral domain is a ring where no two
	non-zero elements can multiply to give a zero ($\forall a, b \in R,
	a\cdot b = 0 \implies a =0 \text{ or } b = 0$).  A field is an
	integral domain where multiplicative inverse exists for every non-zero
	element.
\end{definition}
Note that $\Z$ is an integral domain (For brevity, we will skip the 
operations ($+, \cdot$) involved from now on).

We also argued, by a pigeon hole argument, that every finite integral domain is
field (Theorem~\ref{thm:finite-id-is-field}). Integral domains have the
following useful cancellation property.  
\begin{lemma} \label{lem:id-cancel}
	For an integral domain $R$, with $c,d,e \in R$. If 
	$c\cdot d = c\cdot e$ and $c \ne 0$, then $d=e$
\end{lemma}
\begin{proof}
	Since $c\cdot d = c \cdot e$, $c\cdot (d-e) = 0$. Since $c \ne 0$ and $R$ is
	an integral domain, it must be that $d-e = 0$ or $d=e$.
\end{proof}

Our aim is to see properties of $\Z$ and generalise them
to integral domains and this is the motivation behind all the definitions
below. Let $R$ denote an integral domain in the following
definitions.
\begin{definition}[Unit]
	A unit in $R$ is an element which has a multiplicative inverse.
\end{definition}
For example in $\Z$, units are $1,-1$. Consider $\Z[i] = \{a + bi ~|~ a,b \in
\Z \}$\footnote{Verify that this is indeed an integral domain}.  The units are
$\{1,-1,i,-i\}$. Here $i$ denotes the square root of $-1$.
\begin{definition}[Irreducible]
	An element $p \in R$ is irreducible if whenever $p$ can be expressed
	as $a \cdot b$ where $a,b \in R$ either $a$ or $b$ is a unit.
\end{definition}
For example, in $\Z$, $7 = (-1)(-7)$. Hence $7$ is irreducible. Similarly $2$
is irreducible in $\Z$ but $2 = (1+i)(1-i)$ in $\Z[i]$. Hence $2$ is not
irreducible in $\Z[i]$.

\begin{definition}[Factorization]
	Let $f \in R$. A factorization of $f$ into irreducibles is an
	expression of the form \[ f = c\prod_i p_i^{\alpha_i}\] where $c$ is a
	unit in $R$, $p_i \in R$ are irreducibles and $\alpha_i \in \Z^{\ge
	0}$
\end{definition}

Note that $\Z$ has the property that any factorization is always unique.
\begin{definition}[Unique factorization Domain]
	An integral domain $R$ is said to be a unique factorization domain
	(UFD) if for every $f \in R$, $f$ has a unique factorization upto
	muliplication by units.
\end{definition}
We need to work over UFDs since only then factorization exists and is well
defined.  Our aim is to characterise UFD and show that some of the rings that
we have seen $\Z[x], \polyring$ are indeed UFDs. Hence it makes sense to ask
for factorization of polynomials in these rings.

Similar to the notion of primes in $\Z$, let us define the notion of primes
for integral domains.
\begin{definition}[Prime]
	An element $p \in R$ is said to be a prime if for all $a,b \in R$,
	$p \mid a\cdot b \implies p \mid a$ or $p \mid b$.
\end{definition}

\section{Characterization of UFD}
We now characterize UFDs. Firstly, we show that in a UDF, every prime is 
an irreducible.
\begin{lemma}
	For a UFD $R$, every prime in $R$ is an irreducible in $R$
\end{lemma}
\begin{proof}
	Let $p$ be a prime in $R$. We want to show that if $p=a\cdot b$ for
	some $a,b \in R$, then $a$ or $b$ is a unit.

	Let $p =a\cdot b$ for some $a, b \in R$. Since $p$ is a prime, it must
	be that $p \mid a$ or $p \mid b$. Suppose $p \mid a$ and $c \in R$ be
	such that $a = p\cdot  c$. Hence $p =a \cdot b$ implies, $p = p \cdot
	b \cdot c$. Hence by cancellation property of integral domains
	(Lemma~\ref{lem:id-cancel}), we get $b\cdot c = 1$. Hence $b$ has an
	inverse and it a unit. If $p \mid b$, then a similar argument shows
	that $a$ is a unit.
\end{proof}
So a natural question to ask is when are irreducibles also primes. It turns
out that this is true iff $R$ is a UFD. 
\begin{theorem}
	Let $R$ be an integral domain. Then,
	\begin{enumerate}
		\item $R$ is a UFD implies all irreducibles in $R$ are also
			prime.  
		\item If factorization exists for every $f \in R$ and all
			irreducibles in $R$ are primes in $R$, then $R$ is a
			UFD.
	\end{enumerate}
	\label{thm:ufd-characterisation}
\end{theorem}
\begin{proof}
	(Proof of (1)) Suppose $R$ is a UFD. Let $f \in R$ be an irreducible.
	Let $a,b \in R$ such that $f \mid a\cdot b$. Note that $a$ and $b$
	both cannot be units. Since $a,b \in R$ and $R$
	is a UFD, $a,b$ can be uniquely written as $a = u_1 \prod_i
	p_i^{\alpha_i}$ and $b = u_2 \prod_j q_j^{\beta_j}$ where $u_1,u_2$ are
	units. We need to show that if $f \mid a\cdot b$ then $f$ appears at
	least once in the factorization of $a$ or $b$. We prove this by
	contradiction.

	Suppose that $f$ did not appear as an irreducible factor in $a$ or
	$b$. Then consider $c = ab/f$. Since $c \in R$ and $R$ is a UFD, it
	has a unique factorization $c = u \prod_k r_k^{\gamma_k}$ where $u$ is
	a unit. Hence $ab =fc = f \cdot u\prod_k r_k^{\gamma_k} $. From the
	factorization of $a$ and $b$ we get $ab = u_1u_2 \prod_i p_i^{\alpha_i} 
	\prod_j q_j^{\beta_j}$. But this factorization of $ab$ does not have the
	irreducible $f$ appearing as a factor while the earlier factorization
	has $f$ appearing as an irreducible factor. This gives that $ab \in R$
	has two different factorization contradicting the fact that $R$ is a
	UFD.

	(Proof of (2)) Suppose all irreducibles in $R$ are primes in $R$. Since
	we are given that every element is factorizable in $R$, we just need
	to prove its uniqueness. 

	Let $f \in R$ and let \[ f = c \left( \prod_i p_i^{\alpha_i} \right) = d
	\left( \prod_j q_j^{\beta_j} \right) \]
	where $c,d$ are units. Let $p_i$ be one irreducible. Since $p_i \mid
	f$, $p_i \mid d \prod_j q_j^{\beta_j}$. Since $p_i$ is an irreducible
	and irreducibles are primes  by hypothesis, we get that $p_i \mid q_j$
	for some $j$. Hence $q_j = p_i u$. Since both $p_i, q_j$ are
	irreducibles, $u$ must be a unit. Repeating the process will
	remove all copies of $p_i$ from $\prod_i p_i^{\alpha_i}$ and all
	copies of $q_j$ from $\prod_j q_j^{\beta_j}$ and we get that for $p_i$
	and $q_j$, their exponents $\alpha_i$ and $\beta_j$ are same. We
	repeat this for all the $p_i$ to get that the set of irreducibles
	$p_i$ and $q_i$ are the same.
\end{proof}


\Lecture{Jayalal Sarma}{October 13 2015}{39}{Principal ideal domain, relation
to Unique factorization domain and Gauss's lemma}{K Dinesh}{$\beta$}{K Dinesh}

Last lecture, we saw that to prove that an integral domain is a UFD, we need
to argue that
\begin{enumerate}
	\item factorization exists for every $f \in R$
	\item all irreducibles are primes
\end{enumerate}

Our goal in current and the next lecture is to show that $\polyring$ is a UFD.
Recall that we showed that $\F[x]$ is a principal ideal domain (PID)
(Theorem~\ref{thm:pid-univariate}).  Following exercise shows that
$\F[x_1,\ldots,x_n]$ for $n > 1$ is not a PID. Also not all ideals in $\Z[x]$
as principal.

\begin{exercise-prob}[See Problem Set 2 (problem~\ref{pid-props})]
\begin{show-ps2}{pid-props}
We showed in class that every ideal of $\mathbb{F}[x]$ is principal (generated by one element). We will show two observations about it.
\begin{enumerate}[(a)]
\item This is special to the case of polynomials in one variable. Consider the ideal $\langle x,y \rangle \subset \mathbb{F}[x,y]$. Prove that $I$ is not a principal ideal. (Hint : If $x = fg$ where $f,g \in \mathbb{F}[x,y]$ prove that $f$ or $g$ is a constant.)
\item This is special to the case of fields. Let $I$ be a subset of $\mathbb{Z}[x]$ consisting of all polynomials with an even constant term, i.e. for $p(x) = a_0 + a_1x+a_2x^2+\ldots+a_nx^n \in \mathbb{Z}[x]$, $p \in I$ if and only if $a_0$ is even. Show that $I$ is an ideal of $\mathbb{Z}[x]$ but not a principal ideal.
\end{enumerate}
\end{show-ps2}
\end{exercise-prob}

We also saw that $\Z$ is a UFD. It is also a PID. 
\begin{lemma} \label{lem:z-is-pid}
	The set of all integers form a PID.
\end{lemma}
\begin{proof}
	Let $I$ be any ideal in $\Z$. Let $J$ denote the set of all positive
	elements in $I$. We show that all elements in $J$ must be a multiple
	of a non-zero integer therby showing that $I$ is also generated from a
	single element. 
	
	By well ordering of positive integers, there is a smallest non-zero
	element $a$ in $J$. We claim that $a$ generates $J$. Suppose it is not
	the case, then there exists an $i \in J$ such that $a \nmid i$. Then $
	j= i \mod a$ satisfy $0 <j <a$. Also $j \in J$ as $j = i -ak$ for some
	integer $k$. But this contradicts minimality of $a$.
\end{proof}

In general, we show the following,
\begin{theorem} \label{thm:pid-is-ufd}
	For an integral domain $R$, if $R$ is a PID then $R$ is a UFD.
\end{theorem}
\begin{theorem} \label{thm:ufd-extn}
	For an integral domain $R$, if $R$ is a UFD, then $R[x]$ is also a
	UFD.
\end{theorem}

The above two theorems tells that factorization is well defined in
$\Z[x_1,\ldots,x_n]$ and $\F[x_1,\ldots,x_n]$.
\begin{corollary}
	$\polyring$ is a UFD. This is because, $\F[x]$ is a PID which by
		Theorem~\ref{thm:pid-is-ufd} is also UFD. Applying
		Theorem~\ref{thm:ufd-extn}, we get that $\polyring$ for any $n
		\ge 1$ is also a UFD. We have shown in 
		Lemma~\ref{lem:z-is-pid} that $\Z$ is a PID. By similar
		argument, we get that $\Z[x_1,\ldots,x_n]$ for $n \ge
		1$ is also a UFD.
\end{corollary}

We now prove the two theorems.
\section{Every PID is an UFD}

\begin{proof}[Proof of Theorem~\ref{thm:pid-is-ufd}]
	Given that $R$ is a PID. To show that $R$ is a UFD, we need to prove
	the following.  
	\begin{description}
	\item[Every irreducible in $R$ is a prime : ] 
	Let $p \in R$ and $p$ be irreducible. We need to show that $p$ is a
	prime. Suppose for $a,b \in R$, $p \mid ab$ and $p \nmid b$. We show
	that $p \mid a$. 

	Consider the ideal $I = \langle p, b \rangle$. Since $I$ is an ideal
	in $R$ and $R$ is a PID, there exists a $c \in R$ such that $I =
	\langle c \rangle$. Hence $c \mid p$ and $c \mid b$. Hence $p = cd$ for
	some $d \in R$. 

	Note that $c$ must be a unit. Suppose $c$ is not a unit. Since $p$ is
	irreducible, $d$ must be a unit. Hence $c = pu$ for unit $u = d^{-1}$.
	Since $c \mid b$, we get that $p \mid b$ which is contradiction. Hence
	$c$ must be a unit and since $c \in I$, there exists $\alpha, \beta
	\in R$ such that $c = \alpha p + \beta b$. Since $c$ is a unit,
	multiplying by $c^{-1}$, we get $1 =  \alpha'p+\beta' b$ for 
	$\alpha' = \alpha c^{-1}$ and $\beta' = \beta c^{-1}$. Now $a = \alpha'
	pa + \beta' ba$. Since $p \mid ab$, we get that $p  \mid
za	\alpha' pa+\beta' ba$. Hence $p \mid a$.
	\item [Factorisation exists in $R$ : ] We show that every element in $R$
	has an irreducible factor. If we show this, then the process of
	factorisation cannot happen infintely and this would prove that
	factorisation exists.
		
	Suppose there exists an $f \in R$ such that $f$ does not have an
	irreducible factor. Let $f_1$ be a non-trivial factor and $f_1 \mid
	f$. Since $f$ does not have an irreducible factor, $f_1$ can also not
	have an irreducible factor. Now we can repeat the argument on $f_1$ to
	get an $f_2$ and in general an infinite sequence of factors
	$f_2,f_3,\ldots$ with $f_{i+1} \mid f_i$. If we define $I_i$ to be the
	ideal $\langle f_i \rangle $ for $i \ge 1$ with $I_0 = \langle f
	\rangle$ we have the following infinte chain of increasing ideals :
	$I_0 \subset I_1 \subset I_2 \ldots$. Since $f_i$s are non-trivial
	factors, the above chain is strict. Hence by Hilbert's Theorem
	(Theorem~\ref{thm:ascending-chain-condition}), $R$ is not finitely
	generated. But this is a contradiction as $R$ is a PID and every ideal
	is finitely generated. Hence all $f \in R$ has an irreducible factor.
\end{description}
\end{proof}

\section{Gauss's Lemma}
We now prove the Theorem~\ref{thm:ufd-extn}. Let $R$ be a UFD. 
To show that $R[x]$ is a UFD, we need to show (similar to the previous proof)
that (1) every irreducible in $R[x]$ is prime and (2) factorisation exists for
every $f \in R$.


