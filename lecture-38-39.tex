\Lecture{Jayalal Sarma}{October 12 2015}{38}{Univariate Polynomial
Factorization and Unique Factorization Domains}{K Dinesh}{$\beta$}{K Dinesh}

Recall that our aim is to obtain a solution to $f_i(x_1,\ldots,x_n) = 0$ for
$i \in [k]$ where $f_i \in \polyring$.

We saw that given $f_1,\ldots,f_k$, with respect to some term ordering and $I
= \langle f_1,\ldots,f_k \rangle$ with $G$ as its Grobner basis,  
it is possible to obtain the ideal $I_i = I \cap \F[x_i,\ldots, x_n]$ 
(called the elimination ideal). We saw in last lecture that the 
generator set of $I_i$, is precisely all polynomials in $G \cap
\F[x_i,\ldots, x_n]$. Using elimination ideals, we can express the equations
$f_i$ as another polynomial $f_i'$ which is only in variables $x_i,
\ldots,x_n$. 
\begin{align*}
	f_1'(x_1,\ldots,x_n) & = 0 \\
	\vdots \\
	f_{n-1}'(x_{n-1},x_n)  & = 0 \\
	f_n'(x_n) & = 0
\end{align*}

Now if we can solve for the univariate case $f_n'(x_n) = 0$, then the solution 
can be substituted back in $f_{n-1}'(x_{n-1},x_n) = 0$ to get a univariate 
polynomial in $x_{n-1}$ and is solved similarly. This highlights the
need for algorithms for solving univariate polynomials.

Natural questions that comes up are : when does a univariate polynomial admit
a solution ? Note that if we can factorize a polynomial we can get it
solution. Hence we need to understand more general question : when can a
polynomial be factorized ? when is it unique ? can we characterize
polynomials that cannot be factored ?

\section{Units, Irreducible and Prime}
Recall that in the ring of integers ($\Z,+, \cdot)$ we have a very
good understanding about the notion of factorization, primes etc.
We want to generalize this notion to an arbitrary ring and ask
questions on when is an element factorizable, what are irreducibles,
what are primes etc?

Recall the definitions of algebraic structures from earlier lectures. 
\begin{definition}
	A ring $(R,+,\cdot)$ is an algebraic structure with $(R,+)$ being an
	Abelian group, $(R,\cdot)$ is a monoid. Also the operation $\cdot$
	distributes over $+$.  An integral domain is a ring where no two
	non-zero elements can multiply to give a zero ($\forall a, b \in R,
	a\cdot b = 0 \implies a =0 \text{ or } b = 0$).  A field is an
	integral domain where multiplicative inverse exists for every non-zero
	element.
\end{definition}
Note that $\Z$ is an integral domain (For brevity, we will skip the 
operations ($+, \cdot$) involved from now on).

We also argued, by a pigeon hole argument, that every finite integral domain is
field. Integral domains have the following useful cancellation property.
\begin{lemma} \label{lem:id-cancel}
	For an integral domain $R$, with $c,d,e \in R$. If 
	$c\cdot d = c\cdot e$ and $c \ne 0$, then $d=e$
\end{lemma}
\begin{proof}
	Since $c\cdot d = c \cdot e$, $c(d-e) = 0$. Since $c \ne 0$ and $R$ is
	an integral domain, it must be that $d-e = 0$ or $d=e$.
\end{proof}

Our aim is to see properties of the integral domain $\Z$ and generalise them
general integral domains. Let $R$ denote an integral domain in the following
definitions.
\begin{definition}[Unit]
	A unit in $R$ is an element which has a multiplicative inverse.
\end{definition}
For example in $\Z$, units are $1,-1$. Consider $\Z[i] = \{a + bi ~|~ a,b \in
\Z \}$\footnote{Verify that this is indeed an integral domain}.  The units are
$\{1,-1,i,-i\}$. 
\begin{definition}[Irreducible]
	An element $p \in R$ is irreducible if whenever $p$ can be expressed
	as $a \cdot b$ where $a,b \in R$ either $a$ or $b$ is a unit.
\end{definition}
For example, in $\Z$, $7 = (-1)(-7)$. Hence $7$ is irreducible. Similarly $2$
is irreducible in $\Z$ but $2 = (1+i)(1-i)$ in $\Z[i]$. Hence $2$ is not
irreducible in $\Z[i]$.

\begin{definition}[Factorization]
	Let $f \in R$. A factorization of $f$ into irreducibles is an
	expression of the form \[ f = c\prod_i p_i^{\alpha_i}\] where $c$ is a
	unit in $R$, $p_i \in R$ are irreducibles and $\alpha_i \in \Z^{\ge
	0}$
\end{definition}

Note that $\Z$ has the property that any factorization is always unique.
\begin{definition}[Unique factorization Domain]
	An integral domain $R$ is said to be a unique factorization domain
	(UFD) if for every $f \in R$, $f$ has a unique factorization upto
	muliplication by units.
\end{definition}
We need to work over UFDs since only then factorization is well defined. 
Our aim is to characterise UFD and show that some of the rings that we have
seen $\Z[x], \polyring$ are indeed UFDs. Hence it makes sense to ask for
factorization of polynomials in these rings.

Similar to the notion of primes in $\Z$, let us define the notion of primes
for integral domains.
\begin{definition}[Prime]
	An element $p \in R$ is said to be a prime if for all $a,b \in R$,
	$p \mid a\cdot b \implies p \mid a$ or $p \mid b$.
\end{definition}

\section{Characterization of UFD}
We now characterize UFDs. Firstly, we show that in a UDF, every prime is 
an irreducible.
\begin{lemma}
	For a UFD $R$, every prime in $R$ is an irreducible in $R$
\end{lemma}
\begin{proof}
	Let $p$ be a prime in $R$. We want to show that if $p=a\cdot b$ for
	some $a,b \in R$, then $a$ or $b$ is a unit.

	Let $p =a\cdot b$ for some $a, b \in R$. Since $p$ is a prime, it must
	be that $p \mid a$ or $p \mid b$. Suppose $p \mid a$ and $c \in R$ be
	such that $a = p\cdot  c$. Hence $p =a \cdot b$ implies, $p = p \cdot
	b \cdot c$. Hence by cancellation property of integral domains
	(Lemma~\ref{lem:id-cancel}), we get $b\cdot c = 1$. Hence $b$ has an
	inverse and it a unit. If $p \mid b$, then a similar argument shows
	that $a$ is a unit.
\end{proof}
So a natural question to ask is when are irreducibles also primes. It turns
out that this is true iff $R$ is a UFD. 
\begin{theorem}
	Let $R$ be an integral domain. Then,
	\begin{itemize}
		\item $R$ is a UFD implies all irreducibles in $R$ are also
			prime.
		\item If factorization exists in $R$ and all irreducibles are
			primes in $R$, then $R$ is a UFD.
	\end{itemize}
	\label{thm:ufd-characterisation}
\end{theorem}
\begin{proof}
	
\end{proof}


\Lecture{Jayalal Sarma}{October 13 2015}{39}{Principal ideal domain, relation
to Unique factorization domain and Gauss's lemma}{K Dinesh}{$\beta$}{K Dinesh}
